


\chapter{RandomNumbers}
\label{RandomNumbers} 


Random numbers are required in a number of different problem domains, such as  

\begin{itemize}
	\item numerics (simulation, Monte-Carlo integration)  
	\item games (non-deterministic enemy behavior) 
	\item security (key generation) 
	\item testing (random coverage in white-box tests)
\end{itemize}

The Boost Random Number Generator Library provides a framework for random number generators with well-defined 
properties so that the generators can be used in the demanding numerics and security domains. For a general introduction 
to random numbers in numerics, see  \cite{NumericalRecipes_2007}, Chapter 7.


Depending on the requirements of the problem domain, different variations of random number generators are appropriate:  

This is based on the Boost Random Library \cite{boost_random}.

\section{Definitions}

\subsection{Random Device}
Class random\_device models a non-deterministic random number generator . It uses one or more implementation-defined 
stochastic processes to generate a sequence of uniformly distributed non-deterministic random numbers. For those 
environments where a non-deterministic random number generator is not available, class random\_device must not be 
implemented. See \cite{Eastlake_1994} for further discussions. 

Implementation Note for Windows: On the Windows operating system, token is interpreted as the name of a cryptographic service provider. By default 
random\_device uses MS\_DEF\_PROV. 


\subsection{Uniform Random Number Generator}

A uniform random number generator is a NumberGenerator that provides a sequence of random numbers uniformly 
distributed on a given range. The range can be compile-time fixed or available (only) after run-time construction of the 
object.  
The tight lower bound of some (finite) set $S$ is the (unique) member $l$ in $S$, so that for all $v$ in $S$, $l \leq v$ holds. Likewise, the 
tight upper bound of some (finite) set $S$ is the (unique) member $u$ in $S$, so that for all $v$ in $S$, $v \leq u$ holds.  

For integer generators (i.e. integer T), the generated values $x$ fulfill $\text{min}() \leq x \leq \text{max}()$, for non-
integer generators (i.e. non-integer T), the generated values $x$ fulfill $\text{min}() \leq x < \text{max}()$.  


\subsection{Pseudo-Random Number Generator}

A pseudo-random number generator is a UniformRandomNumberGenerator which provides a deterministic sequence of 
pseudo-random numbers, based on some algorithm and internal state. Linear congruential and inversive 
congruential generators are examples of such pseudo-random number generators. Often, these generators are very 
sensitive to their parameters. In order to prevent wrong implementations from being used, an external testsuite should 
check that the generated sequence and the validation value provided do indeed match.  
\cite{Knuth_1997} gives an extensive overview on pseudo-random number generation. The descriptions for the specific generators contain additional 
references.  




\section{The Random Number Generator Interface}
\label{RandomNumberInterface}



\subsection{Sampling}
\label{Sampling} \index{Spreadsheet Procedures!Sampling}
This is a place holder reference for Excel Sampling.




\section{Random number generator algorithms}
\label{RandomNumberAlgorithms}

\vspace{0.3cm}
This library provides several pseudo-random number generators. The quality of a pseudo random number generator 
crucially depends on both the algorithm and its parameters. This library implements the algorithms as class templates with 
template value parameters, hidden in namespace boost::random. Any particular choice of parameters is represented 
as the appropriately specializing typedef in namespace boost.  

Pseudo-random number generators should not be constructed (initialized) frequently during program execution, for two 
reasons. First, initialization requires full initialization of the internal state of the generator. Thus, generators with a lot of 
internal state (see below) are costly to initialize. Second, initialization always requires some value used as a "seed" for the 
generated sequence. It is usually difficult to obtain several good seed values. For example, one method to obtain a seed is 
to determine the current time at the highest resolution available, e.g. microseconds or nanoseconds. When the pseudo-
random number generator is initialized again with the then-current time as the seed, it is likely that this is at a near-
constant (non-random) distance from the time given as the seed for first initialization. The distance could even be zero if 
the resolution of the clock is low, thus the generator re-iterates the same sequence of random numbers. For some 
applications, this is inappropriate.  

\vspace{0.3cm}
\begin{mpFunctionsExtract}
	\mpFunctionOne
	{SaveDefaultRngState? Boolean? a boolean value: TRUE if the state was successfully save, FALSE otherwise}
	{FName? String? A String, containing the full path of the file.}
\end{mpFunctionsExtract}

\vspace{0.6cm}
\begin{mpFunctionsExtract}
	\mpFunctionOne
	{LoadDefaultRngState? Boolean? a boolean value: TRUE if the state was successfully loaded, FALSE otherwise}
	{FName? String? A String, containing the full path of the file.}
\end{mpFunctionsExtract}

\vspace{0.3cm}
Note that all pseudo-random number generators described below are CopyConstructible and Assignable. Copying or 
assigning a generator will copy all its internal state, so the original and the copy will generate the identical sequence of 
random numbers. Often, such behavior is not wanted. 

The following table gives an overview of some characteristics of the generators. The cycle length is a rough estimate of 
the quality of the generator; the approximate relative speed is a performance measure, higher numbers mean faster random 
number generation.  

\subsection{Minimal Standard}
The specialization minstd\_rand0 was originally suggested in \cite{Lewis_1969}

It is examined more closely together with minstd\_rand in \cite{Park_1988}.


The specialization minstd\_rand was suggested in \cite{Park_1988}.


\subsection{rand48}
Class rand48 models a pseudo-random number generator . It uses the linear congruential algorithm with the parameters a = 0x5DEECE66D, c = 0xB, m = 2**48. It delivers identical results to the lrand48() function available on some systems (assuming lcong48 has not been called).

It is only available on systems where uint64\_t is provided as an integral type, so that for example static in-class constants and/or enum definitions with large uint64\_t numbers work. 




\subsection{Ecuyer 1988}
The specialization ecuyer1988 was suggested in \cite{LEcuyer_1988}




\subsection{Knuth b}
The specialization knuth\_b is specified by the C++ standard. It is described in \cite{Knuth_1981}



\subsection{Kreutzer 1986}
the specialization kreutzer1986 was suggested in \cite{Kreutzer_1986}



\subsection{Tauss 88}
The specialization taus88 was suggested in \cite{LEcuyer_1996}



\subsection{Hellekalek 1995}
The specialization hellekalek1995 was suggested in \cite{Hellekalek_1995}





\subsection{Mersenne-Twister 11213b}
The specializations mt11213b and mt19937 are from \cite{Matsumoto_1998}



\subsection{Mersenne-Twister 19937}
The specializations mt11213b and mt19937 are from \cite{Matsumoto_1998}


\subsection{Mersenne-Twister 19937 64}
The specializations mt11213b and mt19937 are from \cite{Matsumoto_1998}. adapted for 64 bit.
The recursion is similar but different, so the output is totally different from the 32-bit versions. 


\subsection{Lagged Fibonacci Generators}
The specializations lagged\_fibonacci607 ... lagged\_fibonacci44497 use well tested lags. See \cite{Brent_1992a}

The lags used here can be found in \cite{Brent_1992b}.




\subsection{Ranlux Generators}
The ranlux family of generators are described in \cite{Luescher_1994}.

The levels are given in \cite{James_1994}.







\section{Random number distributions}
\label{RandomNumberDistributions}


\subsection{Uniform, small integer}
discrete uniform distribution on a small set of integers (much smaller than the range of the underlying generator) .

The distribution function uniform\_smallint models a random distribution . On each invocation, it returns a random integer value uniformly distributed in the set of integer numbers $\{min, min+1, min+2, ..., max\}$. It assumes that the desired range $(max-min+1)$ is small compared to the range of the underlying source of random numbers and thus makes no attempt to limit quantization errors.

Let $r_{out} = (max-min+1)$ be the desired range of integer numbers, and let $r_{base}$ be the range of the underlying source of random numbers. Then, for the uniform distribution, the theoretical probability for any number $i$ in the range $r_{out}$ will be $p_{out}=\frac{1}{r_{out}}$. Likewise, assume a uniform distribution on  for the underlying source of random numbers, i.e. . Let  denote the random distribution generated by uniform\_smallint. Then the sum over all i in  of  shall not exceed .

The template parameter IntType shall denote an integer-like value type.




[Note] 
Note


The property above is the square sum of the relative differences in probabilities between the desired uniform distribution  and the generated distribution . The property can be fulfilled with the calculation , as follows: Let . The base distribution on  is folded onto the range . The numbers $i < r$ have assigned  numbers of the base distribution, the rest has only . Therefore,  for $i < r$ and  otherwise. Substituting this in the above sum formula leads to the desired result. 




Note: The upper bound for  is . Regarding the upper bound for the square sum of the relative quantization error of , it seems wise to either choose  so that  or ensure that  is divisible by . 


\subsection{Uniform, integer}
discrete uniform distribution on a set of integers; the underlying generator may be called several times to gather enough randomness for the output.

The class template uniform\_int\_distribution models a random distribution . On each invocation, it returns a random integer value uniformly distributed in the set of integers $\{min, min+1, min+2, ..., max\}$.

The template parameter IntType shall denote an integer-like value type.


\subsection{Uniform, 01}
continuous uniform distribution on the range $[0,1)$; important basis for other distributions.

The distribution function uniform\_01 models a random distribution . On each invocation, it returns a random floating-point value uniformly distributed in the range $[0..1)$.

The template parameter RealType shall denote a float-like value type with support for binary operators $+$, $-$, and $/$.

Note: The current implementation is buggy, because it may not fill all of the mantissa with random bits. I'm unsure how to fill a (to-be-invented) boost::bigfloat class with random bits efficiently. It's probably time for a traits class. 


\subsection{Uniform, Real}
continuous uniform distribution on some range [min, max) of real numbers 

The class template uniform\_real\_distribution models a random distribution . On each invocation, it returns a random floating-point value uniformly distributed in the range $[min..max)$. 

\subsection{Discrete}
discrete distribution with specific probabilities (rolling an unfair dice). 

The class discrete\_distribution models a random distribution . It produces integers in the range $[0, n)$ with the probability of producing each value is specified by the parameters of the distribution. 

Constructs a discrete\_distribution from a std::initializer\_list. If the initializer\_list is empty, equivalent to the default constructor. Otherwise, the values of the initializer\_list represent weights for the possible values of the distribution. For example, given the distribution
\begin{verbatim}
* discrete_distribution<> dist{1, 4, 5};
* 
\end{verbatim}

The probability of a 0 is 1/10, the probability of a 1 is 2/5, the probability of a 2 is 1/2, and no other values are possible. 



\subsection{Piecewise constant}
Constructs a piecewise\_constant\_distribution from two iterator ranges containing the interval boundaries and the interval weights. If there are less than two boundaries, then this is equivalent to the default constructor and creates a single interval, $[0, 1)$.

The values of the interval boundaries must be strictly increasing, and the number of weights must be one less than the number of interval boundaries. If there are extra weights, they are ignored.

For example,
\begin{verbatim}

* double intervals[] = { 0.0, 1.0, 4.0 };
* double weights[] = { 1.0, 1.0 };
* piecewise_constant_distribution<> dist(
*     &intervals[0], &intervals[0] + 3, &weights[0]);

\end{verbatim}

The distribution has a 50\% chance of producing a value between 0 and 1 and a 50\% chance of producing a value between 1 and 4. 



\subsection{Piecewise linear}
Constructs a piecewise\_linear\_distribution from two iterator ranges containing the interval boundaries and the weights at the boundaries. If there are fewer than two boundaries, then this is equivalent to the default constructor and creates a distribution that produces values uniformly distributed in the range [0, 1).

The values of the interval boundaries must be strictly increasing, and the number of weights must be equal to the number of interval boundaries. If there are extra weights, they are ignored.

For example,
\begin{verbatim}
* double intervals[] = { 0.0, 1.0, 2.0 };
* double weights[] = { 0.0, 1.0, 0.0 };
* piecewise_constant_distribution<> dist(
*     &intervals[0], &intervals[0] + 3, &weights[0]);
* 
\end{verbatim}

produces a triangle distribution. 




\subsection{Triangle}
Instantiations of triangle\_distribution model a random distribution. 

A triangle\_distribution has three parameters, $a$, $b$, and $c$, which are the smallest, the most probable and the largest values of the distribution respectively. 


\subsection{Uniform on Sphere}
Instantiations of class template uniform\_on\_sphere model a random distribution . Such a distribution produces random numbers uniformly distributed on the unit sphere of arbitrary dimension dim. 






\chapter{Special Functions (based on Boost)}
\label{BoostSpecialFunctionsBoost} % So I can \ref{altrings3} later.
%\lipsum[2-3]


Boost: \cite{boost_multiprecision}


The standard referencea are \cite{NIST}, and \cite{Temme1996}, and \cite{NumericalRecipes_2007}

See also \cite{Gil2007} and \cite{Gil2011}


See also  \cite{Cuyt_2008}


\section{Gamma and Beta Functions}
\label{GammaFunctionBoost}

Detailed review of the gamma function can be found in \cite{pugh_2004} and \cite{Luschny2012}.

The implementation is based on \cite{boost_math}


\subsection{\texorpdfstring{$\text{Gamma function }\Gamma(x)$}{TGamma}}

\begin{mpFunctionsExtract}
	\mpFunctionOne
	{TgammaBoost? mpNum? the gamma function for $x \neq 0, -1, -2,\ldots$.}
	{x? mpNum? A real number.}
\end{mpFunctionsExtract}

\vspace{0.3cm}
The gamma function for $x \neq 0, -1, -2,\ldots$ is defined by
\begin{equation}
	\Gamma(x)  = \int_{0}^{\infty} t^{x-1} e^{-t} dt \quad (x>0),
\end{equation}
and by analytic continuation if $x<0$, using the reflection formula
\begin{equation}
	\Gamma(x) \Gamma(1-x)  = \pi / \sin(\pi x).
\end{equation}




\subsection{\texorpdfstring{$\text{Logarithm of }\Gamma(x)$}{Lgamma}}

\begin{mpFunctionsExtract}
	\mpFunctionOne
	{LgammaBoost? mpNum? the logarithm of the gamma function.}
	{x? mpNum? A real number.}
\end{mpFunctionsExtract}

\vspace{0.3cm}
This function computes $\ln|\Gamma(x)|$ for $x \neq 0, -1, -2, \ldots$. If $x<0$ the function uses the logarithm of the reflection formula.


\subsection{\texorpdfstring{$\text{Auxiliary function }\Gamma(x) / \Gamma(x+\delta)$}{TgammaDeltaRatio}}

\begin{mpFunctionsExtract}
	\mpFunctionTwo
	{TgammaDeltaRatioBoost? mpNum?  the ratio of gamma functions.}
	{x? mpNum? A real number.}
	{$\delta$? mpNum? A real number.}
\end{mpFunctionsExtract}

\vspace{0.3cm}
This functions returns the ratio of gamma functions in the form
\begin{equation}
	\frac{\Gamma(a)}{\Gamma(a+\delta)}
\end{equation}
Note that the result is calculated accurately even when $\delta$ is small compared to $a$: indeed even if $a+\delta \approx a$. The function is typically used when $a$ is large and $\delta$ is very small.




\subsection{\texorpdfstring{$\text{Digamma function }\psi(x)$}{Digamma}}
%\label{DiGammaFunction}

\begin{mpFunctionsExtract}
	\mpFunctionOne
	{DigammaBoost? mpNum? the digamma function for $x \neq 0, -1, -2,\ldots$.}
	{x? mpNum? A real number.}
\end{mpFunctionsExtract}

\vspace{0.3cm}
The digamma or $\psi$ function is defined as
\begin{equation}
	\psi(x) = \frac{d(\ln \Gamma(x))}{dx} = \frac{\Gamma'(x)}{\Gamma(x)}, \quad x \neq 0, -1, -2,\ldots
\end{equation}
If $x<0$ it is transformed to positive values with the reflection formula
\begin{equation}
	\psi(1-x)=\psi(x) + \pi \cot(\pi x)
\end{equation}
and for $0<x<12$ the recurrence formula
\begin{equation}
	\psi(x+1)=\psi(x) + \frac{1}{x}
\end{equation}





\subsection{Ratio of Gamma Functions}

\begin{mpFunctionsExtract}
	\mpFunctionTwo
	{TgammaratioBoost? mpNum?  the ratio of gamma functions.}
	{a? mpNum? A real number.}
	{b? mpNum? A real number.}
\end{mpFunctionsExtract}

\vspace{0.3cm}
This functions returns the ratio of gamma functions in the form
\begin{equation}
	\frac{\Gamma(a)}{\Gamma(b)} 
\end{equation}



\subsection{Normalised incomplete gamma functions}
\label{IncompleteGammaFunction}

\begin{mpFunctionsExtract}
	\mpFunctionTwo
	{GammaPBoost? mpNum? the normalised incomplete gamma function $P(a,x)$.}
	{a? mpNum? A real number.}
	{x? mpNum? A real number.}
\end{mpFunctionsExtract}

\vspace{0.3cm}
The normalised incomplete gamma function $P(a,x)$ is defined as
\begin{equation}
	P(a,x)=\frac{1}{\Gamma(a)} \int_0^x t^{a-1} e^{-t}dt
\end{equation}
for $a \geq 0$ and $x \geq 0$.

\vspace{0.6cm}
\begin{mpFunctionsExtract}
	\mpFunctionTwo
	{GammaQBoost? mpNum? the normalised incomplete gamma function $Q(a,x)$.}
	{a? mpNum? A real number.}
	{x? mpNum? A real number.}
\end{mpFunctionsExtract}


Boost references are \cite{Temme_1979} and \cite{Temme_1994}



\vspace{0.3cm}
The normalised incomplete gamma function $Q(a,x)$ is defined as
\begin{equation}
	Q(a,x)=\frac{1}{\Gamma(a)} \int_x^{\infty} t^{a-1} e^{-t}dt
\end{equation}
for $a \geq 0$ and $x \geq 0$.



\subsection{Non-Normalised incomplete gamma functions}

\begin{mpFunctionsExtract}
	\mpFunctionTwo
	{NonNormalisedGammaPBoost? mpNum? the non-normalised incomplete gamma function $\Gamma(a,x)$.}
	{a? mpNum? A real number.}
	{x? mpNum? A real number.}
\end{mpFunctionsExtract}

\vspace{0.3cm}
The non-normalised incomplete gamma function $\Gamma(a,x)$ is defined as
\begin{equation}
	\Gamma(a,x)= \int_0^x t^{a-1} e^{-t}dt
\end{equation}
for $a \geq 0$ and $x \geq 0$.

\vspace{0.6cm}
\begin{mpFunctionsExtract}
	\mpFunctionTwo
	{NonNormalisedGammaQBoost? mpNum? the non-normalised incomplete gamma function $\gamma(a,x)$.}
	{a? mpNum? A real number.}
	{x? mpNum? A real number.}
\end{mpFunctionsExtract}

\vspace{0.3cm}
The non-normalised incomplete gamma function $\gamma(a,x)$ is defined as
\begin{equation}
	\gamma(a,x)= \int_x^{\infty} t^{a-1} e^{-t}dt
\end{equation}
for $a \geq 0$ and $x \geq 0$.


Note: in Boost, the functions are referred to as TgammaLower and TgammaUpper.


\subsection{Inverse normalised incomplete gamma functions}

\begin{mpFunctionsExtract}
	\mpFunctionTwo
	{GammaPinvBoost? mpNum? the inverse of the normalised incomplete gamma function $P(a,x)$.}
	{a? mpNum? A real number.}
	{p? mpNum? A real number.}
\end{mpFunctionsExtract}

\vspace{0.3cm}
This function returns the inverse normalised incomplete gamma function, i.e. it calculates $x$ with $P(a,x) = p$. The input parameters are $a>0$, $p \geq 0$,  and $p+q=1$. 

\vspace{0.6cm}
\begin{mpFunctionsExtract}
	\mpFunctionTwo
	{GammaQinvBoost? mpNum? the inverse of the normalised incomplete gamma function $Q(a,x)$.}
	{a? mpNum? A real number.}
	{q? mpNum? A real number.}
\end{mpFunctionsExtract}


\vspace{0.3cm}
This function returns the inverse normalised incomplete gamma function, i.e. it calculates $x$ with $Q(a,x) = q$. The input parameters are $a>0$, $q \geq 0$, and $p+q=1$. 

\vspace{0.6cm}
\begin{mpFunctionsExtract}
	\mpFunctionTwo
	{GammaPinvaBoost? mpNum? the parameter $a$ of the normalised incomplete gamma function $P(a,x)$, such that $P(a,x) = p$.}
	{x? mpNum? A real number.}
	{p? mpNum? A real number.}
\end{mpFunctionsExtract}


\vspace{0.6cm}
\begin{mpFunctionsExtract}
	\mpFunctionTwo
	{GammaQinvaBoost? mpNum? the parameter $a$ of the normalised incomplete gamma function $Q(a,x)$, such that $Q(a,x) = q$.}
	{x? mpNum? A real number.}
	{q? mpNum? A real number.}
\end{mpFunctionsExtract}








\subsection{Derivative  of the normalised incomplete gamma function}

\begin{mpFunctionsExtract}
	\mpFunctionTwo
	{GammaPDerivativeBoost? mpNum? the partial derivative with respect to $x$ of the incomplete gamma function $P(a,x)$.}
	{a? mpNum? A real number.}
	{x? mpNum? A real number.}
\end{mpFunctionsExtract}

\vspace{0.3cm}
The partial derivative with respect to $x$ of the incomplete gamma function $P(a,x)$ is defined as:
\begin{equation}
	\frac{\partial}{\partial x}P(a,x) = \frac{e^{-x} x^{a-1}}{\Gamma(a)}.
\end{equation}




\section{Factorials and Binomial Coefficient}

\subsection{Factorial}

\begin{mpFunctionsExtract}
	\mpFunctionOne
	{FactorialBoost? mpNum? the factorial $n! = \Gamma(n+1) = n \times (n-1) \times \cdots \times 1$.}
	{n? mpNum? An integer.}
\end{mpFunctionsExtract}



\subsection{Double Factorial}

\begin{mpFunctionsExtract}
	\mpFunctionOne
	{DoubleFactorialBoost? mpNum? the double factorial $n!!$.}
	{n? mpNum? An integer.}
\end{mpFunctionsExtract}

\vspace{0.3cm}
For even $n<0$ the result is $\infty$. For positive $n$ the double factorial is defined as

\begin{equation}
	n!!=\begin{cases}
		1 \cdot 3 \cdot 5 \cdots n  & \text{ if } n \text{ is odd.}\\
		2 \cdot 4 \cdot 6 \cdots n  & \text{ if } n \text{ is even.}
	\end{cases}
\end{equation}




\subsection{Rising Factorial}

\begin{mpFunctionsExtract}
	\mpFunctionTwo
	{RisingFactorialBoost? mpNum? the rising factorial of $x$ and $i$.}
	{n? mpNum? An integer.}
	{i? mpNum? An integer.}
\end{mpFunctionsExtract}

\vspace{0.3cm}
Returns the rising factorial of $x$ and $i$:
\begin{equation}
	\text{RisingFactorial}(n, i) = n(n+1)(n+2)(n+3) \cdots(n+i-1)
\end{equation}
or
\begin{equation}
	(n)_i = \frac{\Gamma(n+i)}{\Gamma(n)}.
\end{equation}
Note that both n and i can be negative as well as positive.



\subsection{Falling Factorial}

\begin{mpFunctionsExtract}
	\mpFunctionTwo
	{FallingFactorialBoost? mpNum? the falling factorial of $x$ and $i$.}
	{n? mpNum? An integer.}
	{i? mpNum? An integer.}
\end{mpFunctionsExtract}

\vspace{0.3cm}
The falling factorial of $x$ and $i$ is defined as:
\begin{equation}
	\text{FallingFactorial}(n, i) = n(n-1)(n-2)(n-3)...(n-i+1)
\end{equation}
Note that this function is only defined for positive $i$, hence the unsigned second argument. Argument $n$ can be either positive or
negative however.



\subsection{Binomial coefficient}

\begin{mpFunctionsExtract}
	\mpFunctionTwo
	{BinomialCoefficientBoost? mpNum? the binomial coefficient.}
	{n? mpNum? An integer.}
	{k? mpNum? An integer.}
\end{mpFunctionsExtract}

\vspace{0.3cm}
The binomial coefficient ("$n$ choose $k$") is defined as
\begin{equation}
	\binom{n}{k} = \frac{n(n-1) \cdots (n-k+1)}{k(k-1) \cdots (1)}
\end{equation}
for $k \geq 0$.





\section{Beta Functions}



\subsection{Beta function B(a, b)}
\label{BetaFunctionBoost}

\begin{mpFunctionsExtract}
	\mpFunctionTwo
	{BetaBoost? mpNum? the beta function.}
	{a? mpNum? A real number.}
	{b? mpNum? A real number.}
\end{mpFunctionsExtract}

\vspace{0.3cm}
The beta function is defined as
\begin{equation}
	B(a,b) = \frac{\Gamma(a)\Gamma(b)}{\Gamma(a+b)}
\end{equation}
where $\Gamma(\cdot)$ denotes the Gamma function (see section \ref{GammaFunctionBoost}).
The reference is \cite{Didonato_1992}


%
%
%\subsection{Logarithm of  B(a, b)}
%
%\begin{mpFunctionsExtract}
%\mpFunctionTwo
%{LnBetaBoost? mpNum? the logarithm of the beta function $\ln B(a,b)|$ with $a,b \neq 0,-1,-2,\ldots$.}
%{a? mpNum? A real number.}
%{b? mpNum? A real number.}
%\end{mpFunctionsExtract}
%
%
%
%The alogorithm is implemented as in \cite{DiDonato_1987}
%
%\subsection{Normalised incomplete beta functions}
%\label{sec:IncompleteBetaFunctionBoost}
%\label{sec:NormalisedIncompleteBetaFunction}
%\nomenclature{$I_x(a,b)$}{Normalised incomplete beta function}
%
%
%\begin{mpFunctionsExtract}
%\mpFunctionThree
%{IBetaBoost? mpNum? the normalised incomplete beta function.}
%{a? mpNum? A real number.}
%{b? mpNum? A real number.}
%{x? mpNum? A real number.}
%\end{mpFunctionsExtract}
%
%\vspace{0.3cm}
%This function returns the normalised incomplete beta function $I_x(a,b)$ for $a>0$, $b>0$, and $0 \leq x \leq 1$:
%\begin{equation}
%I_x(a,b) = \frac{B_x(a,b)}{B(a,b)}, \quad B_x(a,b) = \int_0^x t^{a-1} (1-t)^{b-1} dt.
%\end{equation}
%There are some special cases
%\begin{equation}
%I_0(a,b)=0, \quad I_1(a,b)=1, \quad I_x(a,1)=x^a,
%\end{equation}
%and the symmetry relation $I_x(a,b)=1-I_{1-x}(b,a)$, which is used for $x>a/(a+b)$.
%
%
%\vspace{0.6cm}
%\begin{mpFunctionsExtract}
%\mpFunctionThree
%{IBetacBoost? mpNum? the normalised complement of the incomplete beta function, $1 - I_x(a,b)$.}
%{a? mpNum? A real number.}
%{b? mpNum? A real number.}
%{x? mpNum? A real number.}
%\end{mpFunctionsExtract}
%
%
%
%
%\subsection{Non-Normalised incomplete beta functions}
%
%\begin{mpFunctionsExtract}
%\mpFunctionThree
%{IBetaNonNormalizedBoost? mpNum? the non-normalised incomplete beta function.}
%{a? mpNum? A real number.}
%{b? mpNum? A real number.}
%{x? mpNum? A real number.}
%\end{mpFunctionsExtract}
%
%\vspace{0.3cm}
%This function returns the non-normalised incomplete beta function $B_x(a,b)$ for $a>0$, $b>0$, and $0 \leq x \leq 1$:
%\begin{equation}
%B_x(a,b) = \int_0^x t^{a-1} (1-t)^{b-1} dt.
%\end{equation}
%There are some special cases
%\begin{equation}
%B_0(a,b)=0, \quad B_1(a,b)=B(a,b), \quad B_x(a,1)= \frac{x^a}{a}, \quad B_x(1,b)= \frac{1-(1-x)^b}{b},
%\end{equation}
%and the relation $B_{1-x}(a,b)=B(a,b)-B_x(b,a)$, which is used if $x>a/(a+b)$. 
%
%
%\vspace{0.6cm}
%\begin{mpFunctionsExtract}
%\mpFunctionThree
%{IBetacNonNormalizedBoost? mpNum? the non-normalised complement of the incomplete beta function, $1 - B_x(a,b)$.}
%{a? mpNum? A real number.}
%{b? mpNum? A real number.}
%{x? mpNum? A real number.}
%\end{mpFunctionsExtract}
%
%
%
%\subsection{Inverse normalised incomplete beta functions}
%
%
%\begin{mpFunctionsExtract}
%\mpFunctionThree
%{IBetaInvBoost? mpNum? the inverse of the normalised incomplete beta function $I_x(a,b)$.}
%{a? mpNum? A real number.}
%{b? mpNum? A real number.}
%{p? mpNum? A real number.}
%\end{mpFunctionsExtract}
%
%\vspace{0.3cm}
%This function calculates $x$ such that $I_x(a,b) = p$. The input parameters are $a, b>0$, $p \geq 0$,  and $p+q=1$. 
%
%\vspace{0.6cm}
%\begin{mpFunctionsExtract}
%\mpFunctionThree
%{IBetacInvBoost? mpNum? the inverse of the complement of the normalised incomplete beta function $1 - I_x(a,b)$.}
%{a? mpNum? A real number.}
%{b? mpNum? A real number.}
%{q? mpNum? A real number.}
%\end{mpFunctionsExtract}
%
%
%\vspace{0.3cm}
%This function calculates $x$ such that $1 - I_x(a,b) = q$. The input parameters are $a, b>0$, $q \geq 0$, and $p+q=1$. 
%
%\vspace{0.6cm}
%\begin{mpFunctionsExtract}
%\mpFunctionThree
%{IBetaInvaBoost? mpNum? the parameter $a$ of the normalised incomplete beta function $I_x(a,b)$, such that $I_x(a,b) = p$.}
%{x? mpNum? A real number.}
%{b? mpNum? A real number.}
%{p? mpNum? A real number.}
%\end{mpFunctionsExtract}
%
%
%\vspace{0.6cm}
%\begin{mpFunctionsExtract}
%\mpFunctionThree
%{IBetacInvaBoost? mpNum? the parameter $a$ of the complement of the normalised incomplete beta function $1-I_x(a,b)$, such that $1-I_x(a,b) = q$.}
%{x? mpNum? A real number.}
%{b? mpNum? A real number.}
%{q? mpNum? A real number.}
%\end{mpFunctionsExtract}
%
%
%\vspace{0.6cm}
%\begin{mpFunctionsExtract}
%\mpFunctionThree
%{IBetaInvbBoost? mpNum? the parameter $b$ of the normalised incomplete beta function $I_x(a,b)$, such that $I_x(a,b) = p$.}
%{x? mpNum? A real number.}
%{a? mpNum? A real number.}
%{p? mpNum? A real number.}
%\end{mpFunctionsExtract}
%
%
%\vspace{0.6cm}
%\begin{mpFunctionsExtract}
%\mpFunctionThree
%{IBetacInvbBoost? mpNum? the parameter $b$ of the complement of the normalised incomplete beta function $1-I_x(a,b)$, such that $1-I_x(a,b) = q$.}
%{x? mpNum? A real number.}
%{a? mpNum? A real number.}
%{q? mpNum? A real number.}
%\end{mpFunctionsExtract}
%
%
%
%
%
%\subsection{Derivative of the Normalised Incomplete beta Function}
%\label{sec:DerivativeNormalisedIncompleteBetaFunction}
%\nomenclature{$I'_x(a,b)$}{Derivative of the normalised incomplete beta function}
%
%\begin{mpFunctionsExtract}
%\mpFunctionThree
%{IBetaDerivativeBoost? mpNum? the partial derivative with respect to $x$ of the incomplete beta function.}
%{x? mpNum? A real number.}
%{a? mpNum? A real number.}
%{b? mpNum? A real number.}
%\end{mpFunctionsExtract}
%
%\vspace{0.3cm}
%The partial derivative with respect to $x$ of the incomplete beta function is defined as:
%\begin{equation}
%\frac{\partial}{\partial x}I_x(a,b) = \frac{(1-x)^{b-1} x^{a-1}}{B(a,b)}.
%\end{equation}
%
%
%
\section{Error Function and Related Functions}

\subsection{Error Function erf}
\label{Boost Error Function erf}

\begin{mpFunctionsExtract}
	\mpFunctionOne
	{ErfBoost? mpNum? the value of the error function.}
	{x? mpNum? A real number.}
\end{mpFunctionsExtract}


\vspace{0.3cm}
The error function is defined by
\begin{equation}
	\text{erf}(x) = \frac{2}{\sqrt{\pi}} \int_0^x e^{-x^2} dt,
\end{equation}



\subsection{Complementary Error Function}
\label{Boost Complementary Error Function}

\begin{mpFunctionsExtract}
	\mpFunctionOne
	{ErfcBoost? mpNum? the value of the complementary error function.}
	{x? mpNum? A real number.}
\end{mpFunctionsExtract}



\subsection{Inverse Function of erf}
\label{Boost Inverse Function of erf}

\begin{mpFunctionsExtract}
	\mpFunctionOne
	{ErfInvBoost? mpNum? the functional inverse of $\text{erf}(x)$}
	{x? mpNum? A real number.}
\end{mpFunctionsExtract}

\vspace{0.3cm}
Tthe functional inverse of $\text{erf}(x)$ is defined by
\begin{equation}
	\text{erf(erf\_inv}(x)) = x, \quad -1<x<1.
\end{equation}




\subsection{Inverse Function of erfc}
\label{Boost Inverse Function of erfc}

\begin{mpFunctionsExtract}
	\mpFunctionOne
	{ErfcInvBoost? mpNum? the functional inverse of $\text{erfc}(x)$}
	{x? mpNum? A real number.}
\end{mpFunctionsExtract}

\vspace{0.3cm}
The functional inverse of $\text{erfc}(x)$ is defined by
\begin{equation}
	\text{erfc(erfc\_inv}(x)) = x, \quad 0<x<2.
\end{equation}



\newpage
\section{Polynomials}



%\newpage
\subsection{Legendre Polynomials/Functions}


\begin{mpFunctionsExtract}
	\mpFunctionTwo
	{LegendrePBoost? mpNum? $P_l(x)$, the Legendre functions of the first kind.}
	{l? mpNum? An Integer.}
	{x? mpNum? A real number.}
\end{mpFunctionsExtract}

\vspace{0.3cm}
These functions return $P_l(x)$, the Legendre functions of the first kind, also called Legendre polynomials if $l \geq 0$ and $|x| \leq 1$. The Legendre polynomials are orthogonal on the interval $(−1, 1)$ with $w(x) = 1$. If $l \geq 0$ the function uses the recurrence relation for varying degree from [1, 8.5.3]:
\begin{IEEEeqnarray}{rCl} \label{eq:LegendrePBoost}
	P_0 (x) & = & 1 \\
	P_1 (x) & = & x \nonumber \\ 
	(l+1)P_{l+1} (x)& = & (2l+1) P_{l}(x) - l P_{l-1}(x).  \nonumber
\end{IEEEeqnarray}
and for negative $l$ the result is $P_l(x) = P_{-l-1}(x)$.


\vspace{0.6cm}
\begin{mpFunctionsExtract}
	\mpFunctionFour
	{LegendrePNextBoost? mpNum? the Legendre function of the first kind of degree $l+1$, using the results for degree $l$ and $l-1$.}
	{l? mpNum? An Integer. The degree of the last polynomial calculated.}
	{x? mpNum? A real number. The abscissa value.}
	{Pl? mpNum? A real number. The value of the polynomial evaluated at degree $l$.}
	{Plm1? mpNum? A real number. The value of the polynomial evaluated at degree $l-1$.}
\end{mpFunctionsExtract}

\vspace{0.3cm}
This function implements the recursion relation given in equation \ref{eq:LegendrePBoost}




%\newpage
\subsection{Associated Legendre Polynomials/Functions}

\begin{mpFunctionsExtract}
	\mpFunctionThree
	{AssociatedLegendrePlmBoost? mpNum? $L^m_n (x)$, the associated Legendre polynomials of degree $l \geq 0$ and order $m \geq 0$.}
	{l? mpNum? An Integer.}
	{m? mpNum? An Integer.}
	{x? mpNum? A real number.}
\end{mpFunctionsExtract}

\vspace{0.3cm}
This function returns $L^m_n (x)$, the associated Legendre polynomials of degree $l \geq 0$ and order $m \geq 0$, defined for $m>0, |x|<1$ as
\begin{equation}
	P^m_l (x) = (-1)^m (1-x^2)^{m/2} \frac{d^m}{dx^m} P_{l} (x).
\end{equation}

The following recursion relation holds:
\begin{equation} \label{eq:AssociatedLegendrePlmBoost}
	(l-m+1)P^m_{l+1} (x) =  (2l+1) x P^m_{l}(x) - (l+m+1) P^m_{l-1}(x). 
\end{equation}



\vspace{0.6cm}
\begin{mpFunctionsExtract}
	\mpFunctionFive
	{AssociatedLegendrePlmNextBoost? mpNum? the Legendre function of the first kind of degree $l+1$, using the results for degree $l$ and $l-1$.}
	{l? mpNum? An Integer. The degree of the last polynomial calculated.}
	{m? mpNum? An Integer. The order of the Associated Polynomial.}
	{x? mpNum? A real number. The abscissa value.}
	{Pl? mpNum? A real number. The value of the polynomial evaluated at degree $l$.}
	{Plm1? mpNum? A real number. The value of the polynomial evaluated at degree $l-1$.}
\end{mpFunctionsExtract}

\vspace{0.3cm}
This function implements the recursion relation given in equation \ref{eq:AssociatedLegendrePlmBoost}



%\newpage
\subsection{Legendre Functions of the Second Kind}

\begin{mpFunctionsExtract}
	\mpFunctionTwo
	{LegendreQBoost? mpNum? $Q_l(x)$, the Legendre functions of the second kind of degree $l \geq 0$ and $|x| \neq 1$.}
	{l? mpNum? An Integer.}
	{x? mpNum? A real number.}
\end{mpFunctionsExtract}

\vspace{0.3cm}
These functions return $Q_l(x)$, the Legendre functions of the second kind of degree $l \geq 0$ and $|x| \neq 1$, defined as
\begin{IEEEeqnarray}{rCl} 
	Q_0 (x) & = & \frac{1}{2} \ln \frac{1+x}{1-x}  \\
	Q_1 (x) & = & \frac{x}{2} \ln \frac{1+x}{1-x} -1\nonumber \\ 
	(k+1)Q_{k+1} (x)& = & (2k+1) x Q_{k}(x) - k Q_{k-1}(x).  \nonumber
\end{IEEEeqnarray}





\subsection{Laguerre Polynomials}

\begin{mpFunctionsExtract}
	\mpFunctionTwo
	{LaguerreLBoost? mpNum? $L_n (x)$, the Laguerre polynomials of degree $n \geq 0$.}
	{n? mpNum? An Integer.}
	{x? mpNum? A real number.}
\end{mpFunctionsExtract}

\vspace{0.3cm}
This function returns $L_n (x)$, the Laguerre polynomials of degree $n \geq 0$. The Laguerre polynomials are just special cases of the generalized Laguerre polynomials
\begin{equation}
	L_n (x) = L^{(0)}_n (x).
\end{equation}
The standard recurrence formulas are used:
\begin{IEEEeqnarray}{rCl} \label{eq:LaguerreLBoost}
	L_0 (x) & = & 1 \\
	L_1 (x) & = & -x+1 \nonumber \\ 
	nLn (x)& = & (2n-1-x) L_{n-1}(x) - (n-1)  L_{n-2}(x).  \nonumber
\end{IEEEeqnarray}

\vspace{0.6cm}
\begin{mpFunctionsExtract}
	\mpFunctionFour
	{LaguerreLNextBoost? mpNum? the Laguerre polynomial of the first kind of degree $n+1$, using the results for degree $n$ and $n-1$.}
	{n? mpNum? An Integer. The degree of the last polynomial calculated.}
	{x? mpNum? A real number. The abscissa value.}
	{Ln? mpNum? A real number. The value of the polynomial evaluated at degree $n$.}
	{Lnm1? mpNum? A real number. The value of the polynomial evaluated at degree $n-1$.}
\end{mpFunctionsExtract}

\vspace{0.3cm}
This function implements the recursion relation given in equation \ref{eq:LaguerreLBoost}





\subsection{Associated Laguerre Polynomials}

\begin{mpFunctionsExtract}
	\mpFunctionThree
	{AssociatedLaguerreBoost? mpNum? $L^m_n (x)$, the associated Laguerre polynomials of degree $n \geq 0$ and order $m \geq 0$.}
	{n? mpNum? An Integer.}
	{m? mpNum? An Integer.}
	{x? mpNum? A real number.}
\end{mpFunctionsExtract}

\vspace{0.3cm}
This function returns $L^m_n (x)$, the associated Laguerre polynomials of degree $n \geq 0$ and order $m \geq 0$, defined as
\begin{equation}
	L^m_n (x) = (-1)^m \frac{d^m}{dx^m} L_{n+m} (x).
\end{equation}
The standard recurrence formulas are used:
\begin{IEEEeqnarray}{rCl} \label{eq:AssociatedLaguerreLNextBoost}
	L^{a}_0 (x) & = & 1 \\
	L^{a}_1 (x) & = & -x+1+a \nonumber \\ 
	nL^{a}_n (x)& = & (2n+a-1-x) L^{a}_{n-1}(x) - (n+a-1)  L^{a}_{n-2}(x).  \nonumber
\end{IEEEeqnarray}




\vspace{0.6cm}
\begin{mpFunctionsExtract}
	\mpFunctionFive
	{AssociatedLaguerreLNextBoost? mpNum? the associated Laguerre polynomial of the first kind of degree $n+1$, using the results for degree $n$ and $n-1$.}
	{n? mpNum? An Integer. The degree of the last polynomial calculated.}
	{m? mpNum? An Integer. The order of the Associated Polynomial.}
	{x? mpNum? A real number. The abscissa value.}
	{Ln? mpNum? A real number. The value of the polynomial evaluated at degree $n$.}
	{Lnm1? mpNum? A real number. The value of the polynomial evaluated at degree $n-1$.}
\end{mpFunctionsExtract}

\vspace{0.3cm}
This function implements the recursion relation given in equation \ref{eq:AssociatedLaguerreLNextBoost}







\subsection{Hermite Polynomials}

\begin{mpFunctionsExtract}
	\mpFunctionTwo
	{HermiteHBoost? mpNum? $H_n(x)$, the Hermite polynomial of degree $n \geq 0$.}
	{n? mpNum? An Integer.}
	{x? mpNum? A real number.}
\end{mpFunctionsExtract}

\vspace{0.3cm}
These functions return $H_n(x)$, the Hermite polynomial of degree $n \geq 0$. The $H_n$ are orthogonal on the interval $(-\infty, \infty)$, with respect to the weight function $w(x) = e^{-x^2}$.
They are computed with the standard recurrence formulas [1, 22.7.13]:
\begin{IEEEeqnarray}{rCl}  \label{eq:HermiteHBoost}
	H_0 (x) & = & 1 \\ 
	H_1 (x) & = & 2x \nonumber \\ 
	H_n (x)& = & 2x H_{n-1}(x) - 2(n-1)  H_{n-2}(x).  \nonumber
\end{IEEEeqnarray}


\vspace{0.6cm}
\begin{mpFunctionsExtract}
	\mpFunctionFour
	{HermiteHNextBoost? mpNum? the Hermite polynomial of degree $n+1$, using the results for degree $n$ and $n-1$.}
	{n? mpNum? An Integer. The degree of the last polynomial calculated.}
	{x? mpNum? A real number. The abscissa value.}
	{Hn? mpNum? A real number. The value of the polynomial evaluated at degree $n$.}
	{Hnm1? mpNum? A real number. The value of the polynomial evaluated at degree $n-1$.}
\end{mpFunctionsExtract}

\vspace{0.3cm}
This function implements the recursion relation given in equation \ref{eq:HermiteHBoost}





\subsection{Spherical Harmonic Functions}

\begin{mpFunctionsExtract}
	\mpFunctionFour
	{SphericalHarmonicBoost? mpNum? the real and imaginary parts of the spherical harmonic function $Y_{lm}(\theta, \phi)$.}
	{l? mpNumList[2]? An Integer.}
	{m? mpNum? An Integer.}
	{$\theta$? mpNum? A real number.}
	{$\phi$? mpNum? A real number.}
\end{mpFunctionsExtract}

\vspace{0.3cm}
The procedures return the real and imaginary parts of the spherical harmonic function $Y_{lm}(\theta, \phi)$. These functions are closely related to the associated Legendre polynomials:
\begin{equation}
	Y_{lm}(\theta, \phi) = \sqrt{\frac{(2l+1) (l-m)!}{4 \pi (l+m)!}} P^m_l (\cos(\theta)) e^{i m \phi}
\end{equation}


\section{Bessel Functions of Real Order}
\label{BesselFunctionsBoostReal}

\subsection{\texorpdfstring{$\text{Bessel Function }J_{\nu}(x)$}{Bessel Function Jnu}}


\begin{mpFunctionsExtract}
	\mpFunctionTwo
	{BesselJBoost? mpNum? $J_{\nu}(z)$, the Bessel function of the first kind of real order $\nu$.}
	{x? mpNum? A real number.}
	{$\nu$? mpNum? A real number.}
\end{mpFunctionsExtract}


\vspace{0.3cm}
$J_{\nu}(z)$, the Bessel function of the first kind of order $\nu$, is defined as
\begin{equation}
	J_{\nu}(x)  = \left(\tfrac{1}{2}x\right)^{\nu}  \sum_{k=0}^\infty (-1)^k \frac{(x^2 / 4)^k}{k! \Gamma(\nu+k+1)}
\end{equation}





\subsection{\texorpdfstring{$\text{Bessel Function }Y_{\nu}(x)$}{Ynux}}

\begin{mpFunctionsExtract}
	\mpFunctionTwo
	{BesselYBoost? mpNum? $Y_{\nu}(z)$, the Bessel function of the second kind of order $\nu$.}
	{x? mpNum? A real number.}
	{$\nu$? mpNum? A real number.}
\end{mpFunctionsExtract}


\vspace{0.3cm}
$Y_{\nu}(z)$, the Bessel function of the second kind of order $\nu$, is defined as
\begin{equation}
	Y_{\nu}(x)  = \frac{J_{\nu}(x) \cos(\nu \pi) - J_{-\nu}(x)}{ \sin(\nu \pi)}
\end{equation}





\section{Modified Bessel Functions of Real Order}
\label{ModifiedBesselFunctionsBoostReal}


\subsection{\texorpdfstring{$\text{Bessel Function }I_{\nu}(x)$}{Bessel Function Inu}}

\begin{mpFunctionsExtract}
	\mpFunctionTwo
	{BesselIBoost? mpNum? the modified Bessel function $I_{\nu}(z)$ of the first kind of order $\nu$.}
	{x? mpNum? A real number.}
	{$\nu$? mpNum? A real number.}
\end{mpFunctionsExtract}


\vspace{0.3cm}
This function returns the modified Bessel function $I_{\nu}(z)$ of the first kind of order $\nu$, defined as
\begin{equation}
	I_{\nu}(z)  = \frac{z}{2}  \sum_{j=0}^\infty \frac{(z^2 / 4)^j}{j! \Gamma(\nu+j+1)}
\end{equation}





\subsection{\texorpdfstring{$\text{Bessel Function }K_{\nu}(x)$}{Bessel Function Knu}}

\begin{mpFunctionsExtract}
	\mpFunctionTwo
	{BesselKBoost? mpNum? $K_{\nu}(x)$, the modified Bessel function of the second kind of order $\nu$.}
	{x? mpNum? A real number.}
	{$\nu$? mpNum? A real number.}
\end{mpFunctionsExtract}


\vspace{0.3cm}
This function returns $K_{\nu}(z)$, the modified Bessel function of the second kind of order $\nu$, defined as
\begin{equation}
	K_{\nu}(x)  = \frac{\pi}{2} \frac{I_{-\nu}(x)) - I_{\nu}(x)}{ \sin(\nu \pi)}
\end{equation}





\section{Spherical Bessel Functions}

\subsection{\texorpdfstring{$\text{Spherical Bessel function }j_n(x)$}{jjnx}}

\begin{mpFunctionsExtract}
	\mpFunctionTwo
	{BesselSphericaljBoost? mpNum? $j_n(x)$, the spherical Bessel function of the 1st kind, order $n$.}
	{x? mpNum? A real number.}
	{$\nu$? mpNum? A real number.}
\end{mpFunctionsExtract}

\vspace{0.3cm}
The function $j_n(x)$, the spherical Bessel function of the 1st kind, order $n$, is defined as
\begin{equation}
	j_n(x) = \sqrt{\tfrac{1}{2}\pi/x} J_{n+\tfrac{1}{2}}(x), \quad (x\leq 0), \quad \text{and } j_n(-x)=(-1)^n j_n(x).
\end{equation}




\subsection{\texorpdfstring{$\text{Spherical Bessel function }y_n(x)$}{yynx}}

\begin{mpFunctionsExtract}
	\mpFunctionTwo
	{BesselSphericalyBoost? mpNum? $j_n(x)$, the spherical Bessel function of the 1st kind, order $n$.}
	{x? mpNum? A real number.}
	{$\nu$? mpNum? A real number.}
\end{mpFunctionsExtract}

\vspace{0.3cm}
The function $y_n(x)$, the spherical Bessel function of the second kind, order $n$, $x \neq 0$, is defined as
\begin{equation}
	y_n(x) = \sqrt{\tfrac{1}{2}\pi/x} Y_{n+\tfrac{1}{2}}(x), \quad (x > 0), \quad \text{and } y_n(-x)=(-1)^{n+1} y_n(x).
\end{equation}



%
%\subsection{\texorpdfstring{$\text{Modified spherical Bessel function }i_n(x)$}{Iinx}}
%\begin{tabular}{p{481pt}}
%\toprule
%\textsf{Function \textbf{BesselSph1i}($\boldsymbol{a}\ As\ mpNum$, $\boldsymbol{b}\ As\ mpNum$) As mpNum}\index{Multiprecision Functions!BesselSph1i} \\
%\bottomrule
%\end{tabular}
%
%\vspace{0.3cm}
%This function returns $i_n(x)$, the modified spherical Bessel function of the 1st (and 2nd) kind, order $n$.
%Except for $n = 0$ where the value $\text{sinh}(x)/x$ is used, the result is calculated just from the  definition [1, 10.1.1]:
%\begin{equation}
%i_n(x) = \sqrt{\tfrac{1}{2}\pi/x} I_{n+\tfrac{1}{2}}(x), \quad (x\geq 0), \quad \text{and } i_n(-x)=(-1)^n i_n(x).
%\end{equation}
%with the reflection formula from [30, 10.47.16]. Note that $i_n$ is named $i^{(1)}_n$ in the NIST
%handbook [30] and restricted to $n \geq 0$, the modified spherical Bessel function of the 2nd kind is then defined as
%\begin{equation}
%i^{(2)}_n = \sqrt{\tfrac{1}{2}\pi/x} I_{-n-\tfrac{1}{2}}(x), 
%\end{equation}
%which can be expressed by $i_n$, i.e.
%\begin{equation}
%i^{(1)}_n = i_n(x), \quad i^{(2)}_n = i_{-n-1}(x), \quad (\text{for }n \geq 0).
%\end{equation}
%
%
%\subsection{\texorpdfstring{$\text{Exponentially scaled Bessel function }i_{n,e}(x)$}{Iinex}}
%\begin{tabular}{p{481pt}}
%\toprule
%\textsf{Function \textbf{BesselSph1ie}($\boldsymbol{a}\ As\ mpNum$, $\boldsymbol{b}\ As\ mpNum$) As mpNum}\index{Multiprecision Functions!BesselSph1ie} \\
%\bottomrule
%\end{tabular}
%
%\vspace{0.3cm}
%These functions return $i_n(x) \exp(−|x|)$, the exponentially scaled modified spherical Bessel function of the 1st/2nd kind, order $n$. For $n = 0$ the result is
%\begin{equation}
%i_{0.e}(x)=-\frac{\text{expm1}(-2|x|)}{2|x|},
%\end{equation}
%otherwise $I_{\nu,e}(x)$, the exponentially scaled modified Bessel function of the 1st kind, is used
%\begin{equation}
%i_{n,e}(x) = \sqrt{\tfrac{1}{2}\pi/x} I_{n+\tfrac{1}{2},e}(x), \quad (x\geq 0), \quad \text{and } i_{n,e}(-x)=(-1)^n i_{n,e}(x).
%\end{equation}
%
%\subsection{\texorpdfstring{$\text{Modified spherical Bessel function }k_n(x)$}{kknx}}
%\begin{tabular}{p{481pt}}
%\toprule
%\textsf{Function \textbf{BesselSph1k}($\boldsymbol{a}\ As\ mpNum$, $\boldsymbol{b}\ As\ mpNum$) As mpNum}\index{Multiprecision Functions!BesselSph1k} \\
%\bottomrule
%\end{tabular}
%
%\vspace{0.3cm}
%This function returns $k_n(x)$, the modified spherical Bessel function of the 3rd kind, order $n$, $x>0$.
%Except for $n = 0$ where the value $\tfrac{1}{2}\pi \exp(-x)/x$ is used, the result is calculated just from the  definition [1, 10.1.1]:
%\begin{equation}
%k_n(x) = \sqrt{\tfrac{1}{2}\pi/x} K_{n+\tfrac{1}{2}}(x), \quad (x> 0).
%\end{equation}
%
%
%
%
%\subsection{\texorpdfstring{$\text{Exponentially scaled  Bessel function }k_{n,e}(x)$}{kkinex}}
%\begin{tabular}{p{481pt}}
%\toprule
%\textsf{Function \textbf{BesselSph1ke}($\boldsymbol{a}\ As\ mpNum$, $\boldsymbol{b}\ As\ mpNum$) As mpNum}\index{Multiprecision Functions!BesselSph1ke} \\
%\bottomrule
%\end{tabular}
%
%\vspace{0.3cm}
%Thise function returns $k_n(x) \exp(x)$, the exponentially scaled modified spherical Bessel function of the 3rd kind, order $n$. For $n = 0$ the result is $\pi /(2x)$, otherwise $K_{\nu,e}(x)$, the exponentially scaled modified Bessel function of the 2nd kind, is used
%\begin{equation}
%k_{n,e}(x) = \sqrt{\tfrac{1}{2}\pi/x} K_{n+\tfrac{1}{2},e}(x), \quad (x> 0).
%\end{equation}
%







\section{Hankel Functions}
\label{HankelFunctions}


\subsection{Hankel Function of the First Kind}

\begin{mpFunctionsExtract}
	\mpFunctionTwo
	{cplxHankel1Boost? mpNum? the Hankel function of the first kind $H_{\nu}^{(1)}(x)$.}
	{x? mpNum? A real number.}
	{$\nu$? mpNum? A real number.}
\end{mpFunctionsExtract}

\vspace{0.3cm}
This routine returns the Hankel function of the first kind $H_{\nu}^{(1)}(x)$, defined as
\begin{equation}
	H_{\nu}^{(1)}(x) = J_{\nu}(x) + iY_{\nu}(x).
\end{equation}



\subsection{Hankel Function of the Second Kind}

\begin{mpFunctionsExtract}
	\mpFunctionTwo
	{cplxHankel2Boost? mpNum? the Hankel function of the second kind $H_{\nu}^{(2)}(x)$.}
	{x? mpNum? A real number.}
	{$\nu$? mpNum? A real number.}
\end{mpFunctionsExtract}

\vspace{0.3cm}
This routine returns the Hankel function of the second kind $H_{\nu}^{(2)}(x)$, defined as
\begin{equation}
	H_{\nu}^{(2)}(x) = J_{\nu}(x) - iY_{\nu}(x).
\end{equation}



\subsection{Spherical Hankel Function of the First Kind}

\begin{mpFunctionsExtract}
	\mpFunctionTwo
	{cplxHankelSph1Boost? mpNum? the spherical Hankel function of the first kind $h_{\nu}^{(1)}(x)$.}
	{x? mpNum? A real number.}
	{$\nu$? mpNum? A real number.}
\end{mpFunctionsExtract}

\vspace{0.3cm}
This routine returns the spherical Hankel function of the first kind $h_{\nu}^{(1)}(x)$, defined as
\begin{equation}
	h_{\nu}^{(1)} = \sqrt{\frac{\pi}{2x}} H_{\nu+\tfrac{1}{2}}^{(1)}(x).
\end{equation}


\subsection{Spherical Hankel Function of the Second Kind}

\begin{mpFunctionsExtract}
	\mpFunctionTwo
	{cplxHankelSph2Boost? mpNum? the spherical Hankel function of the second kind $h_{\nu}^{(2)}(x)$.}
	{x? mpNum? A real number.}
	{$\nu$? mpNum? A real number.}
\end{mpFunctionsExtract}

\vspace{0.3cm}
This routine returns the spherical Hankel function of the second kind $h_{\nu}^{(2)}(x)$, defined as
\begin{equation}
	h_{\nu}^{(2)} = \sqrt{\frac{\pi}{2x}} H_{\nu+\tfrac{1}{2}}^{(2)}(x).
\end{equation}









\section{Airy Functions}
\label{AiryFunctionsBoost}


In this section let $z = (2/3)|x|^{3/2}$.  For large negative $x$ the Airy functions and the Scorer function $Gi(x)$ have asymptotic expansions oscillating with $\cos(z + \pi/4)$ or $\sin(z + \pi/4)$, see Abramowitz and Stegun [1, 10.4.60, 10.4.64, 10.4.87]; therefore the phase information becomes totally unreliable for $x < −(2/eps x)^{2/3}$, and
the relative error increases strongly for $x$ less than the square root.

\subsection{\texorpdfstring{$\text{Airy Function Ai}(x)$}{Aix}}

\begin{mpFunctionsExtract}
	\mpFunctionOne
	{AiryAiBoost? mpNum? the Airy function $\text{Ai}(x)$.}
	{x? mpNum? A real number.}
\end{mpFunctionsExtract}

\vspace{0.3cm}
The Airy function $\text{Ai}(x)$ is defined as
\begin{equation}
	\text{Ai}(x) = \frac{1}{\pi} \sqrt{\frac{x}{3}} K_{1/3}(z), \quad (x>0)
\end{equation}
\begin{equation}
	\text{Ai}(x) = \frac{1}{3^{2/3}\Gamma(2/3)} , \quad (x=0)
\end{equation}
\begin{equation}
	\text{Ai}(x) = \tfrac{1}{2} \sqrt{-x} \left(J_{1/3}(z) - \frac{1}{\sqrt{3}} Y_{1/3}(z) \right), \quad (x<0)
\end{equation}




\subsection{\texorpdfstring{$\text{Airy Function Ai}'(x)$}{AiPx}}

\begin{mpFunctionsExtract}
	\mpFunctionOne
	{AiryAiDerivativeBoost? mpNum? the Airy function $\text{Ai}'(x)$.}
	{x? mpNum? A real number.}
\end{mpFunctionsExtract}

\vspace{0.3cm}
This routine returns the Airy function $\text{Ai}'(x)$, defined as
\begin{equation}
	\text{Ai}'(x) = \frac{x}{\pi\sqrt{3}}  K_{2/3}(z), \quad (x>0)
\end{equation}
\begin{equation}
	\text{Ai}'(x) = \frac{1}{-(3^{2/3)}\Gamma(1/3)} , \quad (x=0)
\end{equation}
\begin{equation}
	\text{Ai}'(x) = -\frac{x}{2} \left(J_{2/3}(z) + \frac{1}{\sqrt{3}} Y_{2/3}(z) \right), \quad (x<0)
\end{equation}




\subsection{\texorpdfstring{$\text{Airy Function Bi}(x)$}{Bix}}

\begin{mpFunctionsExtract}
	\mpFunctionOne
	{AiryBiBoost? mpNum? the Airy function $\text{Bi}(x)$.}
	{x? mpNum? A real number.}
\end{mpFunctionsExtract}

\vspace{0.3cm}
This routine returns the Airy function $\text{Bi}(x)$, defined as
\begin{equation}
	\text{Bi}(x) = \sqrt{x} \left(\frac{2}{\sqrt{3}}I_{1/3}(z) + \frac{1}{\pi} K_{1/3}(z) \right), \quad (x>0)
\end{equation}
\begin{equation}
	\text{Bi}(x) = \frac{1}{3^{1/6}\Gamma(2/3)} , \quad (x=0)
\end{equation}
\begin{equation}
	\text{Bi}(x) = -\tfrac{1}{2} \sqrt{-x} \left(\frac{1}{\sqrt{3}} J_{1/3}(z) + Y_{1/3}(z) \right), \quad (x<0)
\end{equation}




\subsection{\texorpdfstring{$\text{Airy Function Bi}'(x)$}{BiPx}}

\begin{mpFunctionsExtract}
	\mpFunctionOne
	{AiryBiDerivativeBoost? mpNum? the Airy function $\text{Bi}'(x)$.}
	{x? mpNum? A real number.}
\end{mpFunctionsExtract}

\vspace{0.3cm}
This routine returns the Airy function $\text{Bi}'(x)$, defined as
\begin{equation}
	\text{Bi}'(x) = x \left(\frac{2}{\sqrt{3}}I_{2/3}(z) + \frac{1}{\pi} K_{2/3}(z) \right), \quad (x>0)
\end{equation}
\begin{equation}
	\text{Bi}(x) = \frac{3^{1/6}}{\Gamma(1/3)} , \quad (x=0)
\end{equation}
\begin{equation}
	\text{Bi}(x) = -\tfrac{x}{2} \left(\frac{1}{\sqrt{3}} J_{2/3}(z) - Y_{2/3}(z) \right), \quad (x<0)
\end{equation}


\section{Carlson-style Elliptic Integrals}
\label{EllipticIntegralCarlsonBoost}


The Carlson style elliptic integrals are a complete alternative group to the classical
Legendre style integrals. They are symmetric and the numerical calculation is usually
performed by duplication as described in  \cite{Carlson_1994} and  \cite{Carlson_1995}.

\subsection{Degenerate elliptic integral RC}

\begin{mpFunctionsExtract}
	\mpFunctionTwo
	{CarlsonRCBoost? mpNum? the value of the of Carlson's degenerate elliptic integral $R_C$.}
	{x? mpNum? A real number.}
	{y? mpNum? A real number.}
\end{mpFunctionsExtract}

\vspace{0.3cm}
This function computes the value of the of Carlson's degenerate elliptic integral $R_C$ for $x \geq 0$, $y \neq 0$:
\begin{equation}
	R_C(x,y)=R_F(x,y,y)=\frac{1}{2} \int_0^\infty (t+x)^{-1/2} (t+y)^{-1} dt.
\end{equation}


\subsection{Integral of the 1st kind RF}

\begin{mpFunctionsExtract}
	\mpFunctionThree
	{CarlsonRFBoost? mpNum? the value of the of Carlson's elliptic integral $R_F$ of the first kind.}
	{x? mpNum? A real number.}
	{y? mpNum? A real number.}
	{z? mpNum? A real number.}
\end{mpFunctionsExtract}

\vspace{0.3cm}
This function computes the value of the of Carlson's elliptic integral $R_F$ of the first kind
\begin{equation}
	R_F(x,y,z)=\frac{1}{2} \int_0^\infty ((t+x)(t+y)(t+z))^{-1/2}  dt.
\end{equation}
with $x, y, z \geq 0$, at most one may be zero. 


\subsection{Integral of the 2nd kind RD}

\begin{mpFunctionsExtract}
	\mpFunctionThree
	{CarlsonRDBoost? mpNum? the value of the of Carlson's elliptic integral $R_D$ of the second kind.}
	{x? mpNum? A real number.}
	{y? mpNum? A real number.}
	{z? mpNum? A real number.}
\end{mpFunctionsExtract}

\vspace{0.3cm}
This function computes the value of the of Carlson's elliptic integral $R_D$ of the second kind
\begin{equation}
	R_D(x,y,z) = R_J(x,y,z,z)  =\frac{3}{2} \int_0^\infty ((t+x)(t+y))^{-1/2} (t+z)^{-3/2}  dt.
\end{equation}
with $z>0$, $x, y \geq 0$, at most one of $x,y$ may be zero. 



\subsection{Integral of the 3rd kind RJ}

\begin{mpFunctionsExtract}
	\mpFunctionFour
	{CarlsonRJBoost? mpNum? the value of the of Carlson's elliptic integral $R_J$ of the third kind.}
	{x? mpNum? A real number.}
	{y? mpNum? A real number.}
	{z? mpNum? A real number.}
	{r? mpNum? A real number.}
\end{mpFunctionsExtract}

\vspace{0.3cm}
This function computes the value of the of Carlson's elliptic integral $R_J$ of the third kind
\begin{equation}
	R_J(x,y,z,r)  =\frac{3}{2} \int_0^\infty ((t+x)(t+y)(t+z))^{-1/2} (t+r)^{-1}  dt.
\end{equation}
with  $x, y, z \geq 0$, at most one of may be zero, and $r \neq 0$. 


\section{Legendre-style Elliptic Integrals}
\label{EllipticIntegralLegendreBoost}

\subsection{Complete elliptic integral of the 1st kind}


\begin{mpFunctionsExtract}
	
	\mpFunctionOne
	{CompleteLegendreEllint1Boost? mpNum? the value of the complete elliptic integral of the first kind.}
	{k? mpNum? A real number.}
\end{mpFunctionsExtract}

\vspace{0.3cm}
This function computes the value of the complete elliptic integral of the first kind
$K(k)$ with $|k| < 1$
\begin{equation}
	K(k)=\int_0^{\pi/2} \frac{dt}{\sqrt{1-k^2 \sin^2 t}}.
\end{equation}



\subsection{Complete elliptic integral of the 2nd kind}

\begin{mpFunctionsExtract}
	\mpFunctionOne
	{CompleteLegendreEllint2Boost? mpNum? the value of the complete elliptic integral of the second kind.}
	{k? mpNum? A real number.}
\end{mpFunctionsExtract}

\vspace{0.3cm}
This function computes the value of the complete elliptic integral of the second kind
$E(k)$ with $|k| \leq 1$
\begin{equation}
	E(k)=\int_0^{\pi/2} \sqrt{1-k^2 \sin^2 t}.
\end{equation}





\subsection{Complete elliptic integral of the 3rd kind}

\begin{mpFunctionsExtract}
	\mpFunctionTwo
	{CompleteLegendreEllint3Boost? the value of the complete elliptic integral of the third kind.}
	{$\nu$? mpNum? A real number.}
	{k? mpNum? A real number.}
\end{mpFunctionsExtract}

\vspace{0.3cm}
This function computes the value of the complete elliptic integral of the third kind
$\Pi(\nu,k)$ with $|k| < , \nu \neq 1$
\begin{equation}
	\Pi(\nu,k)=\int_0^{\pi/2} \frac{dt}{(1-\nu \sin^2 t)\sqrt{1-k^2 \sin^2 t}}.
\end{equation}



\subsection{Legendre elliptic integral of the 1st kind}
\label{Legendre elliptic integral of the 1st kind Boost}

\begin{mpFunctionsExtract}
	\mpFunctionTwo
	{LegendreEllint1Boost? mpNum? the value of the incomplete Legendre elliptic integral of the first kind.}
	{$\phi$? mpNum? A real number.}
	{k? mpNum? A real number.}
\end{mpFunctionsExtract}

\vspace{0.3cm}
This function computes the value of the incomplete Legendre elliptic integral of the first kind
\begin{equation}
	F(\phi, k)=\int_0^{\phi} \frac{dt}{\sqrt{1-k^2 \sin^2 t}}.
\end{equation}
with $|k \sin \phi| \leq 1$.



\subsection{Legendre elliptic integral of the 2nd kind}

\begin{mpFunctionsExtract}
	\mpFunctionTwo
	{LegendreEllint2Boost? mpNum? the value of the incomplete Legendre elliptic integral of the second kind.}
	{$\phi$? mpNum? A real number.}
	{k? mpNum? A real number.}
\end{mpFunctionsExtract}

\vspace{0.3cm}
This function computes the value of the incomplete Legendre elliptic integral of the second kind 
\begin{equation}
	E(\phi, k)=\int_0^{\phi} \sqrt{1-k^2 \sin^2 t}.
\end{equation}
with $|k \sin \phi| \leq 1$.



\subsection{Legendre elliptic integral of the 3rd kind}

\begin{mpFunctionsExtract}
	\mpFunctionThree
	{LegendreEllint3Boost? mpNum? the value of the incomplete Legendre elliptic integral of the third kind.}
	{$\phi$? mpNum? A real number.}
	{$\nu$? mpNum? A real number.}
	{k? mpNum? A real number.}
\end{mpFunctionsExtract}

\vspace{0.3cm}
This function computes the value of the incomplete Legendre elliptic integral of the third kind
\begin{equation}
	\Pi(\phi, \nu,k)=\int_0^{\phi} \frac{dt}{(1-\nu \sin^2 t)\sqrt{1-k^2 \sin^2 t}}.
\end{equation}
with $|k \sin \phi| \leq 1$.



\section{Jacobi Elliptic Functions}
\label{EllipticIntegralBoost}

These procedures return the Jacobi elliptic functions sn, cn, dn for argument $x$ and
complementary parameter $m_c$. A convenient implicit definition of the functions is
\begin{equation}
	x = \int_0^{\text{sn}} \frac{dt}{\sqrt{(1-t^2)(1-k^2 t^2)}}, \quad \text{sn}^2 + \text{cn}^2 = 1,  \quad k^2 \text{sn}^2 + \text{cn}^2 = 1
\end{equation}
with $k^2 = 1 - m_c$. There are a lot of equivalent definition of the Jacobi elliptic functions, e.g. with the Jacobi amplitude function (see e.g. \cite{NIST} [30, 22.16.11/12]) 
\begin{center}
	$\text{sn}(x, k) = \sin(\text{am}(x, k))$, 
	
	$\text{cn}(x, k) = \cos(\text{am}(x, k))$, 
\end{center}

or with Jacobi theta functions (cf. [\cite{NIST}, 22.2]).
\subsection{Jacobi elliptic function sn}

\begin{mpFunctionsExtract}
	\mpFunctionTwo
	{JacobiSNBoost? mpNum? the Jacobi elliptic function $\text{sn}(x, k)$.}
	{x? mpNum? A real number.}
	{k? mpNum? A real number.}
\end{mpFunctionsExtract}



\subsection{Jacobi elliptic function cn}

\begin{mpFunctionsExtract}
	\mpFunctionTwo
	{JacobiCNBoost? mpNum? the Jacobi elliptic function $\text{cn}(x, k)$.}
	{x? mpNum? A real number.}
	{k? mpNum? A real number.}
\end{mpFunctionsExtract}




\subsection{Jacobi elliptic function dn}

\begin{mpFunctionsExtract}
	\mpFunctionTwo
	{JacobiDNBoost? mpNum? the Jacobi elliptic function $\text{dn}(x, k)$.}
	{x? mpNum? A real number.}
	{k? mpNum? A real number.}
\end{mpFunctionsExtract}


\section{Zeta Functions}
\label{ZetaFunctionsBoost}

\subsection{\texorpdfstring{$\text{Riemann }\zeta(s)\text{ function}$}{RiemannZeta}}
\label{RiemannZetaBoost}


\begin{mpFunctionsExtract}
	\mpFunctionOne
	{RiemannZetaBoost? mpNum? the Riemann zeta function.}
	{s? mpNum? A real number.}
\end{mpFunctionsExtract}

\vspace{0.3cm}
The Riemann zeta function $\zeta(s)$ for $s \neq 1$ is defined as
\begin{equation}
	\zeta(s) = \sum_{k=1}^\infty \frac{1}{k^s}, \quad s>1.
\end{equation}
If $s<0$, the reflection formula is used:
\begin{equation}
	\zeta(s) = 2(2\pi)^{s-1} \sin\left(\tfrac{1}{2} \pi s\right) \Gamma(1-s) \zeta(1-s)
\end{equation}

\section{Exponential Integral and Related Integrals}


\subsection{Exponential Integral E1}
\label{Exponential Integral E1}

\begin{mpFunctionsExtract}
	\mpFunctionOne
	{ExponentialIntegralE1Boost? mpNum? the exponential integral $\text{E}_1(x)$.}
	{x? mpNum? A real number.}
\end{mpFunctionsExtract}

\vspace{0.3cm}
The exponential integral $\text{E}_1(x)$ for $x \neq 0$ is defined as
\begin{equation}
	\text{E}_1(x) =  \int_1^\infty \frac{e^{-xt}}{t} dt,
\end{equation}
For $x<0$ the integral is calculated as $\text{E}_1(x) = -\text{Ei}(-x)$. 




\subsection{Exponential Integral Ei}
\label{Exponential Integral Ei}

\begin{mpFunctionsExtract}
	\mpFunctionOne
	{ExponentialIntegralEiBoost? mpNum? the exponential integral $\text{Ei}(x)$.}
	{x? mpNum? A real number.}
\end{mpFunctionsExtract}

\vspace{0.3cm}
The exponential integral $\text{Ei}(x)$ for $x \neq 0$ is defined as
\begin{equation}
	\text{Ei}(x) = -PV \int_{-x}^\infty \frac{e^{-t}}{t} dt= PV \int_{-\infty}^x \frac{e^{t}}{t} dt,
\end{equation}
For $x<0$ the integral is calculated as $\text{Ei}(x) = -\text{E}_1(-x)$. 




\subsection{Exponential Integrals En}
\label{Exponential Integrals En}

\begin{mpFunctionsExtract}
	\mpFunctionTwo
	{ExponentialIntegralEnBoost? mpNum? the exponential integral  $\text{E}_n(x)$.}
	{x? mpNum? A real number.}
	{n? mpNum? A real number.}
\end{mpFunctionsExtract}

\vspace{0.3cm}
The exponential integrals $\text{E}_n(x)$ of integer order is defined as
\begin{equation}
	\text{E}_n(x) = \int_{1}^{\infty} \frac{e^{-xt}}{t^n} dt, \quad (n \geq 0).
\end{equation}
For $x<0$ the integral is calculated as $\text{Ei}(x) = -\text{E}_1(-x)$. 


\section{Basic Functions}

\begin{mpFunctionsExtract}
	\mpFunctionOne
	{SinPiBoost? mpNum? the value of the sine of $\pi x$, with $x$ in radians.}
	{x? mpNum? A real number.}
\end{mpFunctionsExtract}



\begin{mpFunctionsExtract}
	\mpFunctionOne
	{CosPiBoost? mpNum? the value of the cosine of $\pi x$, with $x$ in radians.}
	{x? mpNum? A real number.}
\end{mpFunctionsExtract}


\subsection{\texorpdfstring{$\text{Auxiliary Function }\ln(1+x)$}{lnp1}}

\begin{mpFunctionsExtract}
	\mpFunctionOne
	{Lnp1Boost? mpNum? the value of the function $\ln(1+x)$.}
	{x? mpNum? A real number.}
\end{mpFunctionsExtract}

In Boost, this function is called Log1p.



\subsection{\texorpdfstring{$\text{Auxiliary Function }e^{x}-1$}{expm1}}

\begin{mpFunctionsExtract}
	\mpFunctionOne
	{Expm1Boost? mpNum? the value of the function $\text{expm1}(x) = e^{x}-1$.}
	{x? mpNum? A real number.}
\end{mpFunctionsExtract}



\subsection{\texorpdfstring{$\text{Cube Root: }\sqrt[3]{x}$}{Cbrt}}

\begin{mpFunctionsExtract}
	\mpFunctionOne
	{CbrtBoost? mpNum? the absolute value of the cube root of $x, \sqrt[3]{x}$.}
	{x? mpNum? A real number.}
\end{mpFunctionsExtract}



\subsection{\texorpdfstring{$\text{Auxiliary Function }\sqrt{x+1}-1$}{sqrtp1m1}}

\begin{mpFunctionsExtract}
	\mpFunctionOne
	{Sqrtp1m1Boost? mpNum? the value of $\sqrt{x+1}-1$.}
	{x? mpNum? A real number.}
\end{mpFunctionsExtract}



\subsection{\texorpdfstring{$\text{Auxiliary Function }x^y-1$}{Powm1}}

\begin{mpFunctionsExtract}
	\mpFunctionTwo
	{Powm1Boost? mpNum? the value of $x^y-1, y \in  \mathbb{R}$.}
	{x? mpNum? A real number.}
	{y? mpNum? A real number.}
\end{mpFunctionsExtract}




\subsection{\texorpdfstring{$\text{Auxiliary Function }\sqrt{x^2+y^2}$}{Hypot}}

\begin{mpFunctionsExtract}
	\mpFunctionTwo
	{HypotBoost? mpNum? the value of $\sqrt{x^2+y^2}$.}
	{x? mpNum? A real number.}
	{y? mpNum? A real number.}
\end{mpFunctionsExtract}


\section{Sinus Cardinal Function and Hyperbolic Sinus Cardinal Functions}


\begin{mpFunctionsExtract}
	\mpFunctionOne
	{SincaBoost? mpNum? the sinus cardinal function}
	{x? mpNum? A real number.}
\end{mpFunctionsExtract}

\vspace{0.3cm}
The sinus cardinal function is defined as
\begin{equation}
	\text{sinc}_a(x) = \sin \left( \frac{\pi x}{a} \right) \frac{a}{\pi x}
\end{equation}



\subsection{\texorpdfstring{$\text{Hyperbolic Sinus Cardinal: Sinhc}{}_a(x)$}{Sinhca}}

\begin{mpFunctionsExtract}
	\mpFunctionOne
	{SinhcaBoost? mpNum? the hyperbolic sinus cardinal function.}
	{x? mpNum? A real number.}
\end{mpFunctionsExtract}

\vspace{0.3cm}
The hyperbolic sinus cardinal function is defined as
\begin{equation}
	\text{sinhc}_a(x) = \sinh \left( \frac{\pi x}{a} \right) \frac{a}{\pi x}
\end{equation}




\section{Inverse Hyperbolic Functions}


\subsection{\texorpdfstring{$\text{Hyperbolic Arc-cosine: acosh}(x)$}{acosh}}

\begin{mpFunctionsExtract}
	\mpFunctionOne
	{AcoshBoost? mpNum? the value of the hyperbolic arc-cosine  of $x$ in radians.}
	{x? mpNum? A real number.}
\end{mpFunctionsExtract}




\subsection{\texorpdfstring{$\text{Hyperbolic Arc-sine: asinh}(x)$}{asinh}}

\begin{mpFunctionsExtract}
	\mpFunctionOne
	{AsinhBoost? mpNum? the value of the hyperbolic arc-sine  of $x$ in radians.}
	{x? mpNum? A real number.}
\end{mpFunctionsExtract}



\subsection{\texorpdfstring{$\text{Hyperbolic Arc-tangent: atanh}(x)$}{atanh}}

\begin{mpFunctionsExtract}
	\mpFunctionOne
	{AtanhBoost? mpNum? the value of the hyperbolic arc-tangent  of $x$ in radians.}
	{x? mpNum? A real number.}
\end{mpFunctionsExtract}







\chapter{Distribution Functions}

\section{Introduction to Distribution Functions}
\label{DistributionFunctionsIntroduction} 

%\lipsum[1]

This is a citation~\citet{walck_2007}, and some more.

This is a citation~\citet{VanHauwermeiren_2009}, and some more.

This is a citation~\citet{Rinne_book_2008}, and some more.

This is a citation~\citet{Johnson_1994}, and some more.

This is a citation~\citet{Johnson_1995}, and some more.
\nomenclature{pmf}{probability mass function}%
\nomenclature{pdf}{probability density function}%
\nomenclature{CDF}{cumulative distribution function}%


See also \cite{Monahan_2011}

See also \cite{Lange_2010}

See also \cite{Chernick_2008}

See also \cite{Cheney_2008}


\subsection{Continuous Distribution Functions}

Continuous random number distributions are defined by a probability density function, $p(x)$, such that the probability of $x$ occurring in the infinitesimal range $x$ to $x +dx$ is $p\ dx$. The cumulative distribution function for the lower tail $P(x)$ gives the probability of a variate taking a value less than $x$, and the cumulative distribution function for the upper tail $Q(x)$ gives the probability of a variate taking a value greater than $x$. 

The upper and lower cumulative distribution functions are related by $P(x) + Q(x) = 1$ and satisfy $0 \leq P(x) \leq 1, 0 \leq Q(x) \leq 1$. The inverse cumulative distributions, $x = P-1(P)$ and $x = Q-1(Q)$ give the values of $x$ which correspond to a specific value of $P$ or $Q$. They can be used to find confidence limits from probability values. 



\subsection{Discrete Distribution Functions}

For discrete distributions the probability of sampling the integer value $k$ is given by $p(k)$. The cumulative distribution for the lower tail $P(k)$ of a discrete distribution is defined as the sum over the allowed range of the distribution less than or equal to $k$. The cumulative distribution for the upper tail of a discrete distribution $Q(k)$ is defined as the sum of probabilities for all values greater than $k$. These two definitions satisfy the identity $P(k) + Q(k) = 1$. If the range of the distribution is 1 to $n$ inclusive then $P(n) = 1$, $Q(n) = 0$ while $P(1) = p(1)$, $Q(1) = 1 - p(1)$. 


\newpage
\subsection{Commonly Used Function Types}
\label{Commonly Used Distribution Function Types}

\subsubsection{Functions returning pdf, CDF, and related information}
\label{Functions returning pdf, CDF, and related information}
These functions have the form \textsf{?Dist($x$; [Parameters;], OutputString)}.
Here 

"?" is a placeholder for the name of the distribution, 

"$x$" is the value for which we want to calculate the pdf, CDF etc, 

"[Parameters;]" denote any parameters (like degrees of freedom) of the distribution, and 

"OutputString" specifies the computed results which will be returned. This can be any of the following:

\begin{itemize}
	\item \textbf{pdf}: the probability density function
	\item \textbf{P}: the cumulative distribution function (CDF)
	\item \textbf{Q}: the complement of cumulative distribution function (CDF)
	\item \textbf{logpdf}: the logarithm of the probability density function
	\item \textbf{logP}: the logarithm of the cumulative distribution function (CDF)
	\item \textbf{logQ}: the logarithm of the complement of cumulative distribution function (CDF)
	\item \textbf{h}: hazard function
	\item \textbf{H}: cumulative hazard function
\end{itemize}


\vspace{0.3cm}
As an example, for Student's t-distribution, a "T" is used to specify the name of the distibution, and there is just one distribution parameter, $\nu$, the degrees of freedom. Therefore,  the function has the form

\vspace{0.3cm}
\textsf{TDist($x$ As nmNum; $\nu$ As mpNum, OutputString As String) As mpNumList}, 

\vspace{0.3cm}
and an actual call to the function, requesting the pdf, CDF, and the complement of the CDF for $x=2.3$ and $\nu=22$ could be

\lstset{language={[Visual]Basic}}
\begin{lstlisting}
Result = TDist(2.3, 22, "pdf + P + Q")
mp.Print Result
\end{lstlisting}
which produces the output

\begin{verbatim}
pdf: 0.434234342343434
P: 0.943453463453453
Q: 0.054564564564236
\end{verbatim}




\newpage
\subsubsection{Functions returning Quantiles}
\label{Functions returning Quantiles}
These functions have the form \textsf{?DistInv(Prob; [Parameters;], OutputString)}.
Here 

"?" is a placeholder for the name of the distribution, 

"Prob" sets the target values for $P$ and $Q$, 

"[Parameters;]" denote any parameters (like degrees of freedom) of the distribution, and 

"OutputString" specifies the computed results which will be returned. This can be any of the following:

\begin{itemize}
	\item \textbf{PInv}: the inverse of the cumulative distribution function (CDF). For discrete distribution, this will be outwardly rounded
	\item \textbf{QInv}: the inverse of the complement of the cumulative distribution function (CDF). For discrete distribution, this will be outwardly rounded
	\item \textbf{P}: the value of the cumulative distribution function (CDF), which has actually been achieved
	\item \textbf{Q}: the value of the complement of the cumulative distribution function (CDF), which has actually been achieved
\end{itemize}


\vspace{0.3cm}
As an example, for Student's t-distribution, a "T" is used to specify the name of the distibution, and there is just one distribution parameter, $\nu$, the degrees of freedom. Therefore,  the function has the form

\vspace{0.3cm}
\textsf{TDistInv($Prob$ As mpNum; $\nu$ As mpNum, OutputString As String) As mpNumList}, 

\vspace{0.3cm}
and an actual call to the function, requesting the inverse of the complement of the CDF for $Prob=0.01$ and $\nu=22$ could be

\lstset{language={[Visual]Basic}}
\begin{lstlisting}
Result = TDistInv(0,01, 22, "QInv")
mp.Print Result
\end{lstlisting}
which produces the output

\begin{verbatim}
QInv: 2.943453463453453
\end{verbatim}


\newpage
\subsubsection{Functions returning moments and related information}
\label{Functions returning moments and related information}
These functions have the form \textsf{?DistInfo([Parameters;], OutputString)}.
Here 

"?" is a placeholder for the name of the distribution, 

"[Parameters;]" denote any parameters (like degrees of freedom) of the distribution, and 

"OutputString" specifies the computed results which will be returned. This can be any of the following:

\begin{itemize}
	\item \textbf{range}: Returns the valid range of the random variable over distribution dist. 
	\item \textbf{support}: 	
	\item \textbf{mode}: Returns the mode of the distribution dist. This function may return a domain\_error if the distribution does not have a defined mode.
	\item \textbf{median}: Returns the median of the distribution dist.
	\item \textbf{mean}: Returns the mean of the distribution dist. This function may return a domain\_error if the distribution does not have a defi ned mean (for e xample the Cauchy distribution).
	\item \textbf{stdev}: Returns the standard deviation of distribution dist.
	This function may return a domain\_error if the distribution does not have a defined standard deviation.
	\item \textbf{variance}: Returns the variance of the distribution dist.
	This function may return a domain\_error if the distribution does not have a defi ned v ariance.
	\item \textbf{skewness}: Returns the skewness of the distribution dist.
	This function may return a domain\_error if the distribution does not have a defined skewness.
	\item \textbf{kurtosis}: Returns the 'proper' kurtosis (normalized fourth moment) of the distribution dist.
	\item \textbf{kurtosis excess}: Returns the kurtosis excess of the distribution dist. kurtosis excess = kurtosis - 3
\end{itemize}



\vspace{0.3cm}
As an example, for Student's t-distribution, a "T" is used to specify the name of the distibution, and there is just one distribution parameter, $\nu$, the degrees of freedom. Therefore,  the function has the form

\vspace{0.3cm}
\textsf{TDistInfo($\nu$ As mpNum, OutputString As String) As mpNumList}, 

\vspace{0.3cm}
and an actual call to the function, requesting the mean, varaince, skewness and kurtosis with $\nu=22$ could be

\lstset{language={[Visual]Basic}}
\begin{lstlisting}
Result = TDistInfo(22, "mean + variance + skewness + kurtosis")
mp.Print Result
\end{lstlisting}
which produces the output

\begin{verbatim}
mean: 0.434234342343434
variance: 0.943453463453453
skewness: 0.054564564564236
kurtosis: 0.6054564564564236
\end{verbatim}









\newpage
\subsubsection{Functions returning Sample Size estimates}
\label{Functions returning Sample Size estimates}
These functions have the form \textsf{?SampleSize(Alpha; Beta; ModifiedNoncentrality; [Parameters;],  OutputString)}.
Here 

"?" is a placeholder for the name of the distribution, 

"Alpha" specifies the confidence level (or Type I error), 

"Beta" specifies the Type I error (or 1 $-$ Power), 

"ModifiedNoncentrality" specifies the (modified) noncentrality parameter of the distribution in a form which does not depend on sample size (which may require a modification compared to the conventional form for stating the noncentrlaity parameter), 

"[Parameters;]" denote any additional parameters of the distribution (if any) which are not a function of the sample size, and 

"OutputString" specifies the computed results which will be returned. This can be any of the following:

\begin{itemize}
	\item \textbf{ExactN}: returns an "exact", i.e. typically non-integer sample size estimate 
	\item \textbf{UpperN}: upper integer sample size estimate
	\item \textbf{LowerN}: lower integer sample size estimate
	\item \textbf{UpperNPower}: actual power when using UpperN
	\item \textbf{LowerNPower}: actual power when using LowerN
\end{itemize}


\vspace{0.3cm}
As an example, for the noncentral  t-distribution, the prefix "NoncentralT" is used to specify the name of the distibution. The distribution parameter $\nu$, the degrees of freedom, which depends on the sample size, and is therefore not included in the parameter list of this function. The modified noncentrality parameter is called $\tilde{\rho} = \Delta/\sigma$. Therefore, the function has the form

\vspace{0.3cm}
\textsf{NoncentralTSampleSize($\alpha$ As mpNum, $\beta$ As mpNum, $\tilde{\rho}$ As mpNum, OutputString As String) As mpNumList}

\vspace{0.3cm}
and an actual call to the function, requesting an upper sample size estimate (and actual power) for $\alpha = 0.95$, $\beta=0.1$ , and $\tilde{\rho} = \Delta/\sigma = 0.6$   would be

\lstset{language={[Visual]Basic}}
\begin{lstlisting}
Result = NoncentralTSampleSize(0.95, 0.1, 0.6, "UpperN + UpperNPower")
mp.Print Result
\end{lstlisting}
which produces the output

\begin{verbatim}
UpperN: 26
UpperNPower: 0.92435435
\end{verbatim}



\newpage
\subsubsection{Functions related to noncentrality parameters}
\label{Functions related to noncentrality parameters}
These functions have the form \textsf{?Noncentrality(Alpha; Noncentrality; [Parameters;],  OutputString)}.
Here 

"?" is a placeholder for the name of the distribution, 

"Alpha" specifies the confidence level (or Type I error), 

"Noncentrality" specifies the noncentrality parameter of the distribution, 

"[Parameters;]" denote any additional parameters of the distribution, and 

"OutputString" specifies the computed results which will be returned. This can be any of the following:

\begin{itemize}
	\item \textbf{UpperCI}: upper confidence interval
	\item \textbf{LowerCI}: lower confidence interval
	\item \textbf{TwoSidedCI}: two-sided confidence interval
\end{itemize}


\vspace{0.3cm}
As an example, for the noncentral  t-distribution, the prefix "NoncentralT" is used to specify the name of the distibution. The noncentrality parameter is $\delta$, and the other distribution parameter is $\nu$, the degrees of freedom.  Therefore, the function has the form

\vspace{0.3cm}
\textsf{NoncentralTNoncentrality($\alpha$ As mpNum, $\delta$ As mpNum, $\nu$ As mpNum, OutputString As String) As mpNumList}

\vspace{0.3cm}
and an actual call to the function, requesting an upper confidence interval for $\delta$ with $\alpha = 0.95$, $\delta = 0.6$ and $\nu=22$   would be

\lstset{language={[Visual]Basic}}
\begin{lstlisting}
Result = NoncentralTNoncentrality(0.95, 0.6, 22, "UpperCI")
mp.Print Result
\end{lstlisting}
which produces the output

\begin{verbatim}
UpperCI: 0.7546534
\end{verbatim}



\newpage
\subsubsection{Functions returning Random numbers}
\label{Functions returning Random numbers}
These functions have the form \textsf{?DistRan(Size; [Parameters;], Generator, OutputString)}.
Here 

"?" is a placeholder for the name of the distribution, 

"Size" specifies the size of the random sample, 

"[Parameters;]" denote any parameters (like degrees of freedom) of the distribution, and 

"Generator" specifies the pseudo random generator which will be used to produce the random sample, 

"OutputString" specifies the computed results which will be returned. This can be any of the following:

\begin{itemize}
	\item \textbf{Unsorted}: produces unsorted output
	\item \textbf{Ascending}: output sorted in ascending order
	\item \textbf{Descending}: output sorted in descending order
	\item \textbf{Histogram($k$)}: output grouped in histogram format, with $k$ buckets
	\item \textbf{HistogramCDF($k$)}: cumulated output grouped in histogram format, with $k$ buckets
	
\end{itemize}


\vspace{0.3cm}
As an example, for Student's t-distribution, a "T" is used to specify the name of the distibution, and there is just one distribution parameter, $\nu$, the degrees of freedom. Therefore,  the function has the form

\vspace{0.3cm}
\textsf{TDistRan($Size$ As Integer; $\nu$ As mpNum, Generator As String, OutputString As String) As mpNumList}, 

\vspace{0.3cm}
and an actual call to the function, requesting a random sample of  $Size=10000$ of a t-distribution with $\nu=22$, using the default pseudo-random number generator, sorting output in ascending order could be

\lstset{language={[Visual]Basic}}
\begin{lstlisting}
Result = TDistRan(10000, 22, "Default", "Ascending")
mp.Plot Result
\end{lstlisting}
which produces the output

\begin{verbatim}
QInv: 2.943453463453453
\end{verbatim}




\section{Beta-Distribution}
\label{BetaDistribution}

\subsection{Definition}
\label{BetaDistributionDefinition}

If $X_1$ an $X_2$ is are independent random variables  following  $\chi^2$-distribution with $2a$ and $2b$ degrees of freedom respectively, 
then the distribution of the ratio $\frac{X_1}{X_1+X_2}$ is said to follow a Beta-distribution with  $a$ and $b$  degrees of freedom.

See \cite{Tretter_1979}


\subsection{Density and CDF}

\begin{mpFunctionsExtract}
	\mpFunctionFour
	{BetaDist? mpNumList? returns pdf, CDF and related information for the central Beta-distribution}
	{x? mpNum? A real number}
	{a? mpNum? A real number greater 0, representing the numerator  degrees of freedom}
	{b? mpNum? A real number greater 0, representing the denominator degrees of freedom}
	{Output? String? A string describing the output choices}
\end{mpFunctionsExtract}


\vspace{0.3cm}
See section \ref{Functions returning pdf, CDF, and related information} for the options for {\itshape\sffamily Output}. Algorithms and formulas are given in sections \ref{BetaDistributionDensity} and \ref{BetaDistributionCDF}.


%\vspace{0.3cm}
%
%The following functions are provided for compatibility with established spreadsheet functions

%\vspace{0.3cm}
%\begin{mpFunctionsExtract}
%\mpWorksheetFunctionThree
%{BETADIST? mpReal? the CDF and of the central Beta-distribution}
%{x? mpReal? A real number. The numeric value at which to evaluate the distribution}
%{a? mpNum? A real number greater 0, representing the numerator  degrees of freedom}
%{b? mpNum? A real number greater 0, representing the denominator degrees of freedom}
%\end{mpFunctionsExtract}

%\vspace{0.6cm}
%\begin{mpFunctionsExtract}
%\mpWorksheetFunctionFour
%{BETA.DIST? mpReal? the CDF and of the central Beta-distribution}
%{x? mpReal? A real number. The numeric value at which to evaluate the distribution}
%{a? mpNum? A real number greater 0, representing the numerator  degrees of freedom}
%{b? mpNum? A real number greater 0, representing the denominator degrees of freedom}
%{Cumulative ? Boolean? A logical value that determines the form of the function. If cumulative is TRUE, T.DIST returns the cumulative distribution function; if FALSE, it returns the probability density function}
%\end{mpFunctionsExtract}


\subsubsection{Density}
\nomenclature{$f_{\text{Beta}}(a,b,x)$}{pdf of the central Beta-distribution}
\label{BetaDistributionDensity}
The pdf of a variable following a central  Beta-distribution with $a$ and $b$ degrees of freedom is given by

\begin{equation}
	f_{\text{Beta}}(a,b,x) = \frac{1}{B(a,b)} x^{a-1}(1-x)^{b-1}
\end{equation}
where $B(a,b)$ denotes the beta function (see section \ref{BetaFunction}).

\subsubsection{CDF: General formulas}
\nomenclature{$F_{\text{Beta}}(a,b,x)$}{CDF of the central Beta-distribution}
\label{BetaDistributionCDF}
The cdf of a variable following a central  Beta-distribution with $a$ and $b$ degrees of freedom is given by

\begin{equation}
	\text{Pr}\left[X \le x\right] = F_{\text{Beta}}\left(a,b,x\right) =  \int_{0}^{x} f_{\text{Beta}}(a,b,t) dt
\end{equation}

\subsubsection{Exact cdf as continued fraction}
The following representation as continued fraction is used (Peizer 1968, .1428 and 1452):
\begin{equation}
	I(a,b,x)= \binom{n}{a} p^{b-1} q^a \frac{1}{(1+u_1/(v_1+u_2/(v2+u3/(v_3+ \cdots))))}, \quad \text{where } 
\end{equation}
\begin{equation*}
	p=(1-x), \quad q=x, \quad n=a+b-1, \quad u_1= \frac{-(b-1)q}{p}, \quad u_{2j}= \frac{j(n+j)q}{p},
\end{equation*}
\begin{equation*}
	u_{2j+1}= \frac{-(a+j)(b-j-1)q}{p}, \quad v_j=a+j, \quad j=1,2,\ldots
\end{equation*}


\subsection{Quantiles}

\begin{mpFunctionsExtract}
	\mpFunctionFour
	{BetaDistInv? mpNumList? returns quantiles and related information for the the central Beta-distribution}
	{Prob? mpNum? A real number between 0 and 1.}
	{m? mpNum? A real number greater 0, representing the numerator  degrees of freedom}
	{n? mpNum? A real number greater 0, representing the denominator degrees of freedom}
	{Output? String? A string describing the output choices}
\end{mpFunctionsExtract}

See section \ref{Functions returning Quantiles} for the options for  {\itshape\sffamily Prob} and {\itshape\sffamily Output}). 

%\vspace{0.3cm}
%
%The following functions are provided for compatibility with established spreadsheet functions
%
%\vspace{0.3cm}
%\begin{mpFunctionsExtract}
%\mpWorksheetFunctionThree
%{BETAINV? mpReal? the two-tailed inverse of the central Beta-distribution}
%{Prob? mpReal? A real number}
%{a? mpNum? A real number greater 0, representing the numerator  degrees of freedom}
%{b? mpNum? A real number greater 0, representing the denominator degrees of freedom}
%\end{mpFunctionsExtract}
%
%\vspace{0.6cm}
%\begin{mpFunctionsExtract}
%\mpWorksheetFunctionThree
%{BETA.INV? mpReal? the left-tailed inverse of the central Beta-distribution}
%{Prob? mpReal? A real number}
%{a? mpNum? A real number greater 0, representing the numerator  degrees of freedom}
%{b? mpNum? A real number greater 0, representing the denominator degrees of freedom}
%\end{mpFunctionsExtract}


\subsection{Properties}
\label{BetaDistributionProperties}


\begin{mpFunctionsExtract}
	\mpFunctionThree
	{BetaDistInfo? mpNumList? returns moments and related information for the central Beta-distribution}
	{a? mpNum? A real number greater 0, representing the degrees of freedom}
	{b? mpNum? A real number greater 0, representing the degrees of freedom}
	{Output? String? A string describing the output choices}
\end{mpFunctionsExtract}

\vspace{0.3cm}

See section \ref{Functions returning moments and related information} for the options for {\itshape\sffamily Output}. Algorithms and formulas are given in section \ref{tDistributionProperties}.



\subsubsection{Moments: algorithms and formulas}
The raw moments are given by:
\begin{equation}
	E^h(W) = \frac{\Gamma(a+h)\Gamma(a+b)}{\Gamma(a)\Gamma(a+b+h)}
\end{equation}

The raw moments of the power of a beta vairiable are given by:
\begin{equation}
	E^h(W^s) = \frac{\Gamma(a+hs)\Gamma(a+b)}{\Gamma(a)\Gamma(a+b+hs)}
\end{equation}


\subsubsection{Recurrences}

\begin{IEEEeqnarray}{rCl}
	I(a,b;x) & = & 1-I(b,a;1-x)  \\
	I(a,b;x) & = &  \binom{n}{a} x^a (1-x)^{b-1} + I(a+1,b-1; x)  \\
	I(a,b;x) & = &  \binom{n}{a} x^a (1-x)^{b} + I(a+1,b; x)  \\
	I(a,b+1;x) & = &  \binom{n}{a} x^a (1-x)^{b} + I(a,b; x)  \\
	I(a,b;x) & = &  \binom{n}{a+b} x^a (1-x)^{b} \frac{a}{a+b-x} + I(a+1,b+1; x)  \\
	I(a,b;x) & = &  F\left(2a,2b, \frac{nx}{m-mx}\right)
\end{IEEEeqnarray}



\subsection{Random Numbers}

\begin{mpFunctionsExtract}
	\mpFunctionFive
	{BetaDistRandom? mpNumList? returns random numbers following a central Beta-distribution}
	{Size? mpNum? A positive integer up to $10^7$}
	{a? mpNum? A real number greater 0, representing the numerator  degrees of freedom}
	{b? mpNum? A real number greater 0, representing the denominator degrees of freedom}
	{Generator? String? A string describing the random generator}
	{Output? String? A string describing the output choices}
\end{mpFunctionsExtract}

\vspace{0.3cm}

See section \ref{Functions returning Random numbers} for the options for  {\itshape\sffamily Size},  {\itshape\sffamily Generator} and {\itshape\sffamily Output}. Algorithms and formulas are given below.

\subsubsection{Random Numbers: algorithms and formulas}
\label{BetaDistRandomNumbers}
In order to obtain random numbers from a Beta distribution we first single out a few special cases.
For $p = 1$ and/or $q = 1$ we may easily solve the equation $F(x) = \xi$ where $F(x))$ is the cumulative function and $\xi$ a uniform random number between zero and one. In these cases

\begin{center}
	
	$p = 1 \Rightarrow x = 1 - \xi^{1/q}$
	
	$q = 1 \Rightarrow x = \xi^{1/q}$
	
\end{center}


For $p$ and $q$ half-integers we may use the relation to the chi-square distribution by forming the ratio $\frac{y_m}{y_m + y_n}$ with $y_m$ and $y_n$ two independent random numbers from chi-square distributions with $m =2p$ and $n = 2q$ degrees of freedom, respectively.

Yet another way of obtaining random numbers from a Beta distribution valid when $p$ and $q$ are both integers is to take the $l^{th}$ out of $k$ $(1 \leq l \leq k)$ independent uniform random numbers between zero and one (sorted in ascending order). Doing this we obtain a Beta distribution with parameters $p = l$ and $q = k + 1 - l$. Conversely, if we want to generate random numbers from a Beta distribution with integer parameters $p$ and $q$ we could use this technique with $l = p$ and $k = p+q-1$. This last technique implies that for low integer
values of $p$ and $q$ simple code may be used, e.g. for $p = 2$ and $q = 1$ we may simply take max$(\xi_1, \xi_2)$ i.e. the maximum of two uniform random numbers \citep{walck_2007}.






\section{Binomial Distribution}
\label{BinomialDistribution}

These functions return PMF and CDF of the (discrete) binomial distribution with
number of trials $n \geq 0$ and success probability $0 \leq p\leq 1$.



\subsection{Density and CDF}

\begin{mpFunctionsExtract}
	\mpFunctionFour
	{BinomialDist? mpNumList? returns pdf, CDF and related information for the central Binomial-distribution}
	{x? mpNum? The number of successes in trials.}
	{n? mpNum? The number of independent trials.}
	{p? mpNum? The probability of success on each trial}
	{Output? String? A string describing the output choices}
\end{mpFunctionsExtract}


\vspace{0.3cm}
See section \ref{Functions returning pdf, CDF, and related information} for the options for {\itshape\sffamily Output}. Algorithms and formulas are given in sections \ref{BinomialDistributionDensity} and \ref{BinomialDistributionCDF}.


%\vspace{0.3cm}
%
%The following functions are provided for compatibility with established spreadsheet functions
%
%\vspace{0.6cm}
%\begin{mpFunctionsExtract}
%\mpWorksheetFunctionFour
%{BINOMDIST? mpReal? pdf, CDF, and related information of the central Binomial-distribution}
%{x? mpNum? The number of successes in trials.}
%{n? mpNum? The number of independent trials.}
%{p? mpNum? The probability of success on each trial}
%{Cumulative ? Boolean? A logical value that determines the form of the function. If cumulative is TRUE, T.DIST returns the cumulative distribution function; if FALSE, it returns the probability density function}
%\end{mpFunctionsExtract}
%
%
%\vspace{0.6cm}
%\begin{mpFunctionsExtract}
%\mpWorksheetFunctionFour
%{BINOM.DIST? mpReal? the CDF and pdf of the central Binomial-distribution}
%{x? mpNum? The number of successes in trials.}
%{n? mpNum? The number of independent trials.}
%{p? mpNum? The probability of success on each trial}
%{Cumulative ? Boolean? A logical value that determines the form of the function. If cumulative is TRUE, T.DIST returns the cumulative distribution function; if FALSE, it returns the probability density function}
%\end{mpFunctionsExtract}


%\vspace{0.6cm}
%\begin{mpFunctionsExtract}
%\mpWorksheetFunctionFour
%{BINOM.DIST.RANGE? mpReal?  the probability that the number of successful trials will fall between x1 and x22}
%{n? mpNum? The number of independent trials.}
%{p? mpNum? The probability of success on each trial}
%{x1? mpNum? The number x1 of successes in trials.}
%{x2? mpNum? The number x2 of successes in trials.}
%\end{mpFunctionsExtract}



\subsubsection{Density}
\nomenclature{$f_{\text{Bin}}(n, k; p)$}{pmf of the  binomial distribution}
\label{BinomialDistributionDensity}

\begin{equation} 
	f_{\text{Bin}}(n, k; p) = \binom{n}{k} p^k (1-p)^{n-k} = f_{\text{Beta}}(k+1,n-k+1,p)/(n+1)
\end{equation}


\subsubsection{CDF}
\label{BinomialDistributionCDF}
\nomenclature{$F_{\text{Bin}}(n, k; p)$}{CDF of the binomial distribution}
\begin{equation} 
	F_{\text{Bin}}(n, k; p) = I_{1-p}(n-k,k+1) = ibeta(n-k,k+1,1-p)
\end{equation}



\subsection{Quantiles}


\begin{mpFunctionsExtract}
	\mpFunctionFour
	{BinomialDistInv? mpNumList? returns quantiles and related information for the the central binomial-distribution}
	{Prob? mpNum? A real number between 0 and 1.}
	{n? mpNum? The number of Bernoulli trials.}
	{p? mpNum? The probability of a success on each trial.}
	{Output? String? A string describing the output choices}
\end{mpFunctionsExtract}

\vspace{0.3cm}
See section \ref{Functions returning Quantiles} for the options for  {\itshape\sffamily Prob} and {\itshape\sffamily Output}). 

%\vspace{0.3cm}
%
%The following functions are provided for compatibility with established spreadsheet functions
%
%\vspace{0.3cm}
%\begin{mpFunctionsExtract}
%\mpWorksheetFunctionThree
%{CRITBINOM? mpReal? the smallest value for which the cumulative binomial distribution is greater than or equal to a criterion value.}
%{n? mpNum? The number of Bernoulli trials.}
%{p? mpNum? The probability of a success on each trial.}
%{Alpha? mpReal? The criterion value.}
%\end{mpFunctionsExtract}
%
%\vspace{0.6cm}
%\begin{mpFunctionsExtract}
%\mpWorksheetFunctionThree
%{BINOM.INV? mpReal? the smallest value for which the cumulative binomial distribution is greater than or equal to a criterion value.}
%{n? mpNum? The number of Bernoulli trials.}
%{p? mpNum? The probability of a success on each trial.}
%{Alpha? mpReal? The criterion value.}
%\end{mpFunctionsExtract}




\subsection{Properties}
\label{BinomialDistributionProperties}

\begin{mpFunctionsExtract}
	\mpFunctionThree
	{BinomialDistInfo? mpNumList? returns moments and related information for the central Binomial-distribution}
	{n? mpNum? The number of Bernoulli trials.}
	{p? mpNum? The probability of a success on each trial.}
	{Output? String? A string describing the output choices}
\end{mpFunctionsExtract}

\vspace{0.3cm}

See section \ref{Functions returning moments and related information} for the options for {\itshape\sffamily Output}. Algorithms and formulas are given in section \ref{tDistributionProperties}.



\subsubsection{Moments: algorithms and formulas}
\begin{equation} 
	\mu_r^{'} = \sum_{i=0}^r \binom{n}{i} \left(\sum_{j=0}^i \binom{i}{j} (-1)^j (i-j)^r\right)
\end{equation}

\begin{equation} 
	\mu_1 = np
\end{equation}

\begin{equation} 
	\mu_2 = np(1-p) = npq
\end{equation}

\begin{equation} 
	\mu_3 = npq(q-p)
\end{equation}

\begin{equation} 
	\mu_4 = 3(npq)^3 + npq(1-6pq)
\end{equation}





\subsection{Random Numbers}

\begin{mpFunctionsExtract}
	\mpFunctionFive
	{BinomialDistRandom? mpNumList? returns random numbers following a central Binomial-distribution}
	{Size? mpNum? A positive integer up to $10^7$}
	{n? mpNum? The number of Bernoulli trials.}
	{p? mpNum? The probability of a success on each trial.}
	{Generator? String? A string describing the random generator}
	{Output? String? A string describing the output choices}
\end{mpFunctionsExtract}

\vspace{0.3cm}

See section \ref{Functions returning Random numbers} for the options for  {\itshape\sffamily Size},  {\itshape\sffamily Generator} and {\itshape\sffamily Output}. Algorithms and formulas are given below.

\subsubsection{Random Numbers: algorithms and formulas}
In order to obtain random numbers from a Binomial distribution we first single out a few special cases.
For $p = 1$ and/or $q = 1$ we may easily solve the equation $F(x) = \xi$ where $F(x))$ is the cumulative function and $\xi$ a uniform random number between zero and one. In these cases

\begin{center}
	
	$p = 1 \Rightarrow x = 1 - \xi^{1/q}$
	
	$q = 1 \Rightarrow x = \xi^{1/q}$
	
\end{center}


For $p$ and $q$ half-integers we may use the relation to the chi-square distribution by forming the ratio $\frac{y_m}{y_m + y_n}$ with $y_m$ and $y_n$ two independent random numbers from chi-square distributions with $m =2p$ and $n = 2q$ degrees of freedom, respectively.

Yet another way of obtaining random numbers from a Beta distribution valid when $p$ and $q$ are both integers is to take the $l^{th}$ out of $k$ $(1 \leq l \leq k)$ independent uniform random numbers between zero and one (sorted in ascending order). Doing this we obtain a Beta distribution with parameters $p = l$ and $q = k + 1 - l$. Conversely, if we want to generate random numbers from a Beta distribution with integer parameters $p$ and $q$ we could use this technique with $l = p$ and $k = p+q-1$. This last technique implies that for low integer
values of $p$ and $q$ simple code may be used, e.g. for $p = 2$ and $q = 1$ we may simply take max$(\xi_1, \xi_2)$ i.e. the maximum of two uniform random numbers \citep{walck_2007}.




\section{Chi-Square Distribution}
\label{ChiSquareDistribution}

\subsection{Definition}
\label{ChiSquareDistributionDefinition}

Let $X_1, X_2, \ldots, X_n$ be independent and identically distributed random variables each following a normal distribution with mean zero and unit variance. Then $\chi^2 = \sum_{j=1}^n X_j$ is said to follow a $\chi^2$-distribution with $n$ degress of freedom. 


\subsection{Density and CDF}

\begin{mpFunctionsExtract}
	\mpFunctionThree
	{CDist? mpNumList? returns pdf, CDF and related information for the central $\chi^2$-distribution}
	{x? mpNum? A real number}
	{n? mpNum? A real number greater 0, representing the degrees of freedom}
	{Output? String? A string describing the output choices}
\end{mpFunctionsExtract}


\vspace{0.3cm}
See section \ref{Functions returning pdf, CDF, and related information} for the options for {\itshape\sffamily Output}. Algorithms and formulas are given in sections \ref{ChiSquareDistributionDensity} and \ref{sec:ChiSquareDistribution_cdf}.



%\vspace{0.3cm}
%The following functions are provided for compatibility with established spreadsheet functions
%
%\vspace{0.3cm}
%\begin{mpFunctionsExtract}
%\mpWorksheetFunctionThree
%{CHIDIST? mpReal? the CDF and of the central $\chi^2$-distribution}
%{x? mpReal? A real number. The numeric value at which to evaluate the distribution}
%{deg\_freedom? mpReal? An integer  greater 0, indicating the degrees of freedom}
%{Tails? Integer? Specifies the number of distribution tails to return. If tails = 1, TDIST returns the one-tailed distribution. If tails = 2, TDIST returns the two-tailed distribution.}
%\end{mpFunctionsExtract}
%
%\vspace{0.6cm}
%\begin{mpFunctionsExtract}
%\mpWorksheetFunctionThree
%{CHISQDIST? mpReal? the CDF and of the central $\chi^2$-distribution}
%{x? mpReal? A real number. The numeric value at which to evaluate the distribution}
%{deg\_freedom? mpReal? An integer  greater 0, indicating the degrees of freedom}
%{Tails? Integer? Specifies the number of distribution tails to return. If tails = 1, TDIST returns the one-tailed distribution. If tails = 2, TDIST returns the two-tailed distribution.}
%\end{mpFunctionsExtract}


%\vspace{0.6cm}
%\begin{mpFunctionsExtract}
%\mpWorksheetFunctionThree
%{CHISQ.DIST? mpReal? the CDF and of the central $\chi^2$-distribution}
%{x? mpReal? A real number. The numeric value at which to evaluate the distribution}
%{deg\_freedom? mpReal? An integer  greater 0, indicating the degrees of freedom}
%{Cumulative ? Boolean? A logical value that determines the form of the function. If cumulative is TRUE, T.DIST returns the cumulative distribution function; if FALSE, it returns the probability density function}
%\end{mpFunctionsExtract}
%
%\vspace{0.6cm}
%\begin{mpFunctionsExtract}
%\mpWorksheetFunctionTwo
%{CHISQ.DIST.RT? mpReal? the complement of the CDF and of the central $\chi^2$-distribution}
%{x? mpReal? A real number}
%{deg\_freedom? mpReal? An integer  greater 0, indicating the degrees of freedom}
%\end{mpFunctionsExtract}
%
%\vspace{0.6cm}
%\begin{mpFunctionsExtract}
%\mpWorksheetFunctionTwo
%{CHISQ.DIST.2T? mpReal? the two-sided CDF of the central $\chi^2$-distribution}
%{x? mpReal? A real number}
%{deg\_freedom? mpReal? An integer  greater 0, indicating the degrees of freedom}
%\end{mpFunctionsExtract}









\subsubsection{Density}
\label{ChiSquareDistributionDensity}
\nomenclature{$f_{\chi^2}(n, x)$}{pdf of the central chi-square distribution}
The density of a central chi-square variable with $n$ degrees of freedom is given by
\begin{equation}
	f_{\chi^2}(n, x)  = \frac{1}{2^{n/2} \Gamma(n/2)} x^{(n-2)/2}e^{-x/2}.
\end{equation}
%where $\Gamma(\cdot)$ is the Gamma function (see section \ref{GammaFunction}).

\subsubsection{CDF: General formulas}
\label{sec:ChiSquareDistribution_cdf}
\nomenclature{$F_{\chi^2}(n, x)$}{CDF of the central chi-square distribution}

%\vspace{0.3cm}
%
%\subsubsection{Integral representation}
The cdf of a central chi-square variable with $n$ degrees of freedom is given by

\begin{equation}
	\text{Pr}\left[\chi^2 \le x\right] = F_{\chi^2}\left(n, x\right) =  \int_{0}^{x} f_{\chi^2}(n, t) dt
\end{equation}


\subsubsection{CDF: Continued fraction}
For real $n > 0$, the CDF can be calculated using  continued fraction \citep{peizerNormalPart1_1968}.

If $(n-1) \le x$ let $1- F_{\chi^2}\left(n, x\right))$ be a right tail chi square probability. Then

\begin{equation}
	1- F_{\chi^2}\left(n, x\right) = f_{\chi^2}(n, x)  \frac{1}{\left(1+u_1/(v_1 + u_2 / (v_2 + u_3 / (v_3 + \ldots ))) \right)}
\end{equation}

where $M = \tfrac{1}{2}x$, $b = \tfrac{1}{2}n$, $u_{2j-1} = j-b$, $v_{2j-1} = M$, $u_{2j}=j$, $v_{2j}=1$, $j=1,2,\ldots$ 


If $(n-1) > x$ let $ F_{\chi^2}\left(n, x\right)$ be a left tail chi square probability. Then

\begin{equation}
	F_{\chi^2}\left(n, x\right) = f_{\chi^2}(n, x)  \frac{m}{b}  \frac{1}{\left(1+u_1/(v_1 + u_2 / (v_2 + u_3 / (v_3 + \ldots ))) \right)}
\end{equation}

where $M = \tfrac{1}{2}x$, $b = \tfrac{1}{2}n$, $u_1 = -M$,  $u_{2j} = jM$, $u_{2j+1}=-(b+j)M$, $v_{j}=b+j$, $j=1,2,\ldots$ 






\subsection{Quantiles}
\label{ChiSquareDistributionQuantiles}
\nomenclature{$\chi^2_{\nu,\alpha}$}{$\alpha$ quantile of the central $\chi^2$-distribution with $\nu$ degrees of freedom}


\begin{mpFunctionsExtract}
	\mpFunctionThree
	{CDistInv? mpNumList? quantiles and related information for the the central $\chi^2$-distribution}
	{Prob? mpNum? A real number between 0 and 1.}
	{n? mpNum? A real number greater 0, representing the degrees of freedom}
	{Output? String? A string describing the output choices}
\end{mpFunctionsExtract}

See section \ref{Functions returning Quantiles} for the options for  {\itshape\sffamily Prob} and {\itshape\sffamily Output}). 

\vspace{0.3cm}

%The following functions are provided for compatibility with established spreadsheet functions
%
%\vspace{0.3cm}
%\begin{mpFunctionsExtract}
%\mpWorksheetFunctionTwo
%{CHIINV? mpReal? the two-tailed inverse of the central $\chi^2$-distribution}
%{x? mpReal? A real number}
%{deg\_freedom? mpReal? An integer  greater 0, indicating the degrees of freedom}
%\end{mpFunctionsExtract}
%
%\vspace{0.6cm}
%\begin{mpFunctionsExtract}
%\mpWorksheetFunctionTwo
%{CHISQ.INV? mpReal? the left-tailed inverse of the central $\chi^2$-distribution}
%{x? mpReal? A real number}
%{deg\_freedom? mpReal? An integer  greater 0, indicating the degrees of freedom}
%\end{mpFunctionsExtract}
%
%\vspace{0.6cm}
%\begin{mpFunctionsExtract}
%\mpWorksheetFunctionTwo
%{CHISQ.INV.RT? mpReal? the two-tailed inverse of the central $\chi^2$-distribution}
%{x? mpReal? A real number}
%{deg\_freedom? mpReal? An integer  greater 0, indicating the degrees of freedom}
%\end{mpFunctionsExtract}

%\vspace{0.6cm}
%\begin{mpFunctionsExtract}
%\mpWorksheetFunctionTwo
%{CHISQINV? mpReal? the two-tailed inverse of the central $\chi^2$-distribution}
%{x? mpReal? A real number}
%{deg\_freedom? mpReal? An integer  greater 0, indicating the degrees of freedom}
%\end{mpFunctionsExtract}






\subsection{Properties}
\label{ChiSquareDistributionProperties}

\begin{mpFunctionsExtract}
	\mpFunctionTwo
	{CDistInfo? mpNumList? moments and related information for the central $\chi^2$-distribution}
	{n? mpNum? A real number greater 0, representing the degrees of freedom}
	{Output? String? A string describing the output choices}
\end{mpFunctionsExtract}

\vspace{0.3cm}

See section \ref{Functions returning moments and related information} for the options for {\itshape\sffamily Output}. Algorithms and formulas are given in section \ref{ChiSquareDistributionProperties}.






\subsection{Random Numbers}
\label{ChiSquareDistributionRandom}

\begin{mpFunctionsExtract}
	\mpFunctionFour
	{CDistRan? mpNumList? random numbers following a central $\chi^2$-distribution}
	{Size? mpNum? A positive integer up to $10^7$}
	{n? mpNum? A real number greater 0, representing the degrees of freedom}
	{Generator? String? A string describing the random generator}
	{Output? String? A string describing the output choices}
\end{mpFunctionsExtract}


\vspace{0.3cm}
See section \ref{Functions returning Random numbers} for the options for  {\itshape\sffamily Size},  {\itshape\sffamily Generator} and {\itshape\sffamily Output}. Algorithms and formulas are given in section \ref{ChiSquareDistributionRandom}.


\vspace{0.3cm}
As we saw above the sum of n independent standard normal random variables gave a
chi-square distribution with n degrees of freedom. This may be used as a technique to
produce pseudorandom numbers from a chi-square distribution. This required a generator
for standard normal random numbers and may be quite slow. However, if we make use of
the Box-Muller transformation in order to obtain the standard normal random numbers
we may simplify the calculations.
Adding n such squared random numbers implies that

\begin{center}
	
	$y_{2k} = -2 \ln(\xi_1 \cdot \xi_2 \cdot \ldots \cdot \xi_k)$
	
	$y_{2k+1} = -2\ln(\xi_1 \cdot \xi_2 \cdot \ldots \cdot \xi_k) - 2\ln(\xi_{k+1}) [\cos(2\pi\xi_{k+2})]^2$
	
	
\end{center}
for $k$ a positive integer will be distributed as chi-square variable with even or odd number
of degrees of freedom. In this manner a lot of unnecessary operations are avoided.
Since the chi-square distribution is a special case of the Gamma distribution we may
also use a generator for this distribution.



\subsection{Wishart Matrix}

See \cite{gleser1976}




\section{Exponential Distribution}
\label{ExponentialDistribution}


These functions return PDF, CDF, and ICDF of the exponential distribution with
location $a$, rate $\alpha > 0$, and the support interval $(a,+\infty)$ :




\subsection{Density and CDF}

\begin{mpFunctionsExtract}
	\mpFunctionThree
	{ExponentialDist? mpNumList? returns pdf, CDF and related information for the central Exponential distribution}
	{x? mpNum? The value of the distribution.}
	{lambda? mpNum? The parameter of the distribution.}
	{Output? String? A string describing the output choices}
\end{mpFunctionsExtract}


\vspace{0.3cm}
See section \ref{Functions returning pdf, CDF, and related information} for the options for {\itshape\sffamily Output}. Algorithms and formulas are given in sections \ref{BetaDistributionDensity} and \ref{BetaDistributionCDF}.



%\vspace{0.3cm}
%
%The following functions are provided for compatibility with established spreadsheet functions
%
%\vspace{0.6cm}
%\begin{mpFunctionsExtract}
%\mpWorksheetFunctionThree
%{EXPONDIST? mpReal? pdf, CDF, and related information of the central Binomial-distribution}
%{x? mpNum? The value of the distribution.}
%{lambda? mpNum? The parameter of the distribution.}
%{Cumulative ? Boolean? A logical value that determines the form of the function. If cumulative is TRUE, T.DIST returns the cumulative distribution function; if FALSE, it returns the probability density function}
%\end{mpFunctionsExtract}
%
%
%\vspace{0.6cm}
%\begin{mpFunctionsExtract}
%\mpWorksheetFunctionThree
%{EXPON.DIST? mpReal? the CDF and pdf of the central Binomial-distribution}
%{x? mpNum? The value of the distribution.}
%{lambda? mpNum? The parameter of the distribution.}
%{Cumulative ? Boolean? A logical value that determines the form of the function. If cumulative is TRUE, T.DIST returns the cumulative distribution function; if FALSE, it returns the probability density function}
%\end{mpFunctionsExtract}



\subsubsection{Density}
\label{ExponentialDistributionDensity}

\begin{equation} 
	f(x)=\alpha \exp(-\alpha (x-a))
\end{equation}

\subsubsection{CDF}
%\subsection{CDF}
%\label{EXPONDIST} \index{Spreadsheet Functions!EXPONDIST}
%\label{EXPON.DIST} \index{Spreadsheet Functions!EXPON.DIST}
%\begin{tabular}{p{481pt}}
%\toprule
%\textsf{Function \textbf{ExponentialDist}($\boldsymbol{a}\ As\ mpNum$, $\boldsymbol{b}\ As\ mpNum$) As mpNum}\index{Multiprecision Functions!ExponentialDist} \\
%\textsf{Function \textbf{EXPONDIST}($\boldsymbol{a}\ As\ mpNum$, $\boldsymbol{b}\ As\ mpNum$) As mpNum}\index{Multiprecision Functions!EXPONDIST} \\
%\textsf{Function \textbf{EXPON.DIST}($\boldsymbol{a}\ As\ mpNum$, $\boldsymbol{b}\ As\ mpNum$) As mpNum}\index{Multiprecision Functions!EXPON.DIST} \\
%\bottomrule
%\end{tabular}
%
%\vspace{0.3cm}
\begin{equation} 
	F(x)=1- \exp(-\alpha (x-a)) = \text{expm1}(-\alpha (x-a))
\end{equation}



\subsection{Quantiles}


\begin{mpFunctionsExtract}
	\mpFunctionThree
	{ExponentialDistInv? mpNumList? returns quantiles and related information for the the central Exponential distribution}
	{Prob? mpNum? A real number between 0 and 1.}
	{lambda? mpNum? The number of Bernoulli trials.}
	{Output? String? A string describing the output choices}
\end{mpFunctionsExtract}

\vspace{0.3cm}
See section \ref{Functions returning Quantiles} for the options for  {\itshape\sffamily Prob} and {\itshape\sffamily Output}). 

%
%\begin{tabular}{p{481pt}}
%\toprule
%\textsf{Function \textbf{ExponentialDistInv}($\boldsymbol{a}\ As\ mpNum$, $\boldsymbol{b}\ As\ mpNum$) As mpNum}\index{Multiprecision Functions!ExponentialDistInv} \\
%\bottomrule
%\end{tabular}

\vspace{0.3cm}
\begin{equation} 
	F^{-1}(y)=a- \text{ln1p}(-y)/\alpha
\end{equation}



\subsection{Properties}
\label{ExponentialDistributionProperties}


\begin{mpFunctionsExtract}
	\mpFunctionTwo
	{ExponentialDistInfo? mpNumList? returns moments and related information for the central $t$-distribution}
	{lambda? mpNum? A real number greater 0, representing the parameter of the distribution}
	{Output? String? A string describing the output choices}
\end{mpFunctionsExtract}

\vspace{0.3cm}

See section \ref{Functions returning moments and related information} for the options for {\itshape\sffamily Output}. Algorithms and formulas are given in section \ref{tDistributionProperties}.




\subsubsection{Moments and cumulants}
The mean or expected value of an exponentially distributed random variable X with rate parameter $\lambda$ is given by
\begin{equation} 
	E[X]=\frac{1}{\lambda}
\end{equation}

The variance of X is given by
\begin{equation} 
	E[X]=\frac{1}{\lambda^2}
\end{equation}
so the standard deviation is equal to the mean.

The moments of $X$, for $n = 1, 2, ...,$ are given by
\begin{equation} 
	E[X^n]=\frac{n!}{\lambda^n}
\end{equation}



%
%
%\subsection{Random Numbers}
%\begin{tabular}{p{481pt}}
%\toprule
%\textsf{Function \textbf{ExponentialDistRandom}($\boldsymbol{a}\ As\ mpNum$, $\boldsymbol{b}\ As\ mpNum$) As mpNum}\index{Multiprecision Functions!ExponentialDistRandom} \\
%\bottomrule
%\end{tabular}
%
%\vspace{0.3cm}
%\lipsum[2]
%

\subsection{Random Numbers}

\begin{mpFunctionsExtract}
	\mpFunctionFour
	{ExponentialDistRandom? mpNumList? returns random numbers following a central Beta-distribution}
	{Size? mpNum? A positive integer up to $10^7$}
	{lambda? mpNum? A real number greater 0, representing the numerator  degrees of freedom}
	{Generator? String? A string describing the random generator}
	{Output? String? A string describing the output choices}
\end{mpFunctionsExtract}

\vspace{0.3cm}

See section \ref{Functions returning Random numbers} for the options for  {\itshape\sffamily Size},  {\itshape\sffamily Generator} and {\itshape\sffamily Output}. Algorithms and formulas are given in section \ref{FDistributionRandom}.


\subsubsection{Random Numbers: algorithms and formulas}
Random numbers can be generated using the inversion formula.




\section{Fisher's F-Distribution}
\label{FDistribution}

\subsection{Definition}
\label{FDistributionDefinition}

If $X_1$ an $X_2$ is are independent random variables  following  $\chi^2$-distribution with $m$ and $n$ degrees of freedom respectively, 
then the distribution of the ratio $F=\frac{X_1/m}{X_2/n}$ is said to follow a F-distribution with  $m$ and $n$  degrees of freedom.


\subsection{Density and CDF}

\begin{mpFunctionsExtract}
	\mpFunctionFour
	{FDist? mpNumList? returns pdf, CDF and related information for the central $F$-distribution}
	{x? mpNum? A real number}
	{m? mpNum? A real number greater 0, representing the numerator  degrees of freedom}
	{n? mpNum? A real number greater 0, representing the denominator degrees of freedom}
	{Output? String? A string describing the output choices}
\end{mpFunctionsExtract}


\vspace{0.3cm}
See section \ref{Functions returning pdf, CDF, and related information} for the options for {\itshape\sffamily Output}. Algorithms and formulas are given in sections \ref{FDistributionDensity} and \ref{FDistributionCDF}.


%\vspace{0.3cm}
%
%The following functions are provided for compatibility with established spreadsheet functions
%
%\vspace{0.3cm}
%\begin{mpFunctionsExtract}
%\mpWorksheetFunctionThree
%{FDIST? mpReal? the CDF and of the central $F$-distribution}
%{x? mpReal? A real number. The numeric value at which to evaluate the distribution}
%{m? mpNum? A real number greater 0, representing the numerator  degrees of freedom}
%{n? mpNum? A real number greater 0, representing the denominator degrees of freedom}
%\end{mpFunctionsExtract}
%
%\vspace{0.6cm}
%\begin{mpFunctionsExtract}
%\mpWorksheetFunctionFour
%{F.DIST? mpReal? the CDF and of the central $F$-distribution}
%{x? mpReal? A real number. The numeric value at which to evaluate the distribution}
%{m? mpNum? A real number greater 0, representing the numerator  degrees of freedom}
%{n? mpNum? A real number greater 0, representing the denominator degrees of freedom}
%{Cumulative ? Boolean? A logical value that determines the form of the function. If cumulative is TRUE, F.DIST returns the cumulative distribution function; if FALSE, it returns the probability density function}
%\end{mpFunctionsExtract}

%\vspace{0.6cm}
%\begin{mpFunctionsExtract}
%\mpWorksheetFunctionThree
%{F.DIST.RT? mpReal? the complement of the CDF and of the central $F$-distribution}
%{x? mpReal? A real number}
%{m? mpNum? A real number greater 0, representing the numerator  degrees of freedom}
%{n? mpNum? A real number greater 0, representing the denominator degrees of freedom}
%\end{mpFunctionsExtract}






\subsubsection{Density}
\label{FDistributionDensity}
\nomenclature{$f_F(m,n,x)$}{pdf of the central $F$-distribution}
The density of a variable following a central F-distribution with  $m$ and $n$  degrees of freedom is given by
\begin{equation}
	f_F(m,n,x) = \frac{m^{m/2} n^{n/2}}{B(m/2,n/2)} x^{(m-2)/2} (n+mx)^{-(m+n)/2}
\end{equation}

\subsubsection{CDF: General formulas}
\label{FDistributionCDF}
\nomenclature{$F_F(m,n,x)$}{CDF of the central $F$-distribution}
The cdf of a variable following a central  F-distribution with $m$ and $n$ degrees of freedom is given by

\begin{equation}
	\text{Pr}\left[X \le x\right] = F_F\left(m,n,x\right) =  \int_{0}^{x} f(m,n,t) dt
\end{equation}




\subsection{Quantiles}
\label{FDistributionQuantiles}

\nomenclature{$F_{\nu_1,\nu_2,\alpha}$}{$\alpha$ quantile of the central $F$-distribution with $\nu_1$ and $\nu_2$ degrees of freedom}

\begin{mpFunctionsExtract}
	\mpFunctionThree
	{FDistInv? mpNumList? returns quantiles and related information for the the central $t$-distribution}
	{Prob? mpNum? A real number between 0 and 1.}
	{m? mpNum? A real number greater 0, representing the numerator  degrees of freedom}
	{n? mpNum? A real number greater 0, representing the denominator degrees of freedom}
	{Output? String? A string describing the output choices}
\end{mpFunctionsExtract}

See section \ref{Functions returning Quantiles} for the options for  {\itshape\sffamily Prob} and {\itshape\sffamily Output}). 

%\vspace{0.3cm}
%
%The following functions are provided for compatibility with established spreadsheet functions
%
%\vspace{0.3cm}
%\begin{mpFunctionsExtract}
%\mpWorksheetFunctionThree
%{FINV? mpReal? the two-tailed inverse of the central $t$-distribution}
%{x? mpReal? A real number}
%{m? mpNum? A real number greater 0, representing the numerator  degrees of freedom}
%{n? mpNum? A real number greater 0, representing the denominator degrees of freedom}
%\end{mpFunctionsExtract}
%
%\vspace{0.6cm}
%\begin{mpFunctionsExtract}
%\mpWorksheetFunctionThree
%{F.INV? mpReal? the left-tailed inverse of the central $t$-distribution}
%{x? mpReal? A real number}
%{m? mpNum? A real number greater 0, representing the numerator  degrees of freedom}
%{n? mpNum? A real number greater 0, representing the denominator degrees of freedom}
%\end{mpFunctionsExtract}

%\vspace{0.6cm}
%\begin{mpFunctionsExtract}
%\mpWorksheetFunctionThree
%{F.INV.RT? mpReal? the right-tailed inverse of the central $t$-distribution}
%{x? mpReal? A real number}
%{m? mpNum? A real number greater 0, representing the numerator  degrees of freedom}
%{n? mpNum? A real number greater 0, representing the denominator degrees of freedom}
%\end{mpFunctionsExtract}






\subsection{Properties}
\label{FDistributionProperties}

\begin{mpFunctionsExtract}
	\mpFunctionThree
	{FDistInfo? mpNumList? returns moments and related information for the central $t$-distribution}
	{m? mpNum? A real number greater 0, representing the numerator  degrees of freedom}
	{n? mpNum? A real number greater 0, representing the denominator degrees of freedom}
	{Output? String? A string describing the output choices}
\end{mpFunctionsExtract}

\vspace{0.3cm}

See section \ref{Functions returning moments and related information} for the options for {\itshape\sffamily Output}. Algorithms and formulas are given in section \ref{FDistributionProperties}.



\subsection{Random Numbers}
\label{FDistributionRandom}


\begin{mpFunctionsExtract}
	\mpFunctionFive
	{FDistRan? mpNumList? returns random numbers following a central $F$-distribution}
	{Size? mpNum? A positive integer up to $10^7$}
	{m? mpNum? A real number greater 0, representing the numerator  degrees of freedom}
	{n? mpNum? A real number greater 0, representing the denominator degrees of freedom}
	{Generator? String? A string describing the random generator}
	{Output? String? A string describing the output choices}
\end{mpFunctionsExtract}

\vspace{0.3cm}

See section \ref{Functions returning Random numbers} for the options for  {\itshape\sffamily Size},  {\itshape\sffamily Generator} and {\itshape\sffamily Output}. Algorithms and formulas are given in section \ref{FDistributionRandom}.


\subsubsection{Random Numbers: algorithms and formulas}

Following the definition the quantity $F = \frac{y_m/m}{y_n/n}$ where $y_n$ and $y_m$ are two variables distributed according to the chi-square distribution with
$n$ and $m$ degrees of freedom respectively follows the F-distribution. We may thus use this relation inserting random numbers from chi-square distributions (see section ...).





\section{Gamma (and Erlang) Distribution}
\label{GammaDistribution}

These functions return PDF, CDF, and ICDF of the gamma distribution with shape
$a > 0$, scale $b > 0$, and the support interval $(0,+\infty)$.

A gamma distribution with shape $a \in \mathbb{N}$ is called Erlang distribution.

\subsection{Density and CDF}

\begin{mpFunctionsExtract}
	\mpFunctionFour
	{GammaDist? mpNumList? returns pdf, CDF and related information for the central Gamma-distribution}
	{x? mpNum? A real number}
	{a? mpNum? A real number greater 0, a parameter to the distribution}
	{b? mpNum? A real number greater 0, a parameter to the distribution}
	{Output? String? A string describing the output choices}
\end{mpFunctionsExtract}


\vspace{0.3cm}
See section \ref{Functions returning pdf, CDF, and related information} for the options for {\itshape\sffamily Output}. Algorithms and formulas are given in sections \ref{BetaDistributionDensity} and \ref{BetaDistributionCDF}.


%\vspace{0.3cm}
%
%The following functions are provided for compatibility with established spreadsheet functions
%
%\vspace{0.3cm}
%\begin{mpFunctionsExtract}
%\mpWorksheetFunctionFour
%{GAMMADIST? mpReal? the CDF and of the central Gamma-distribution}
%{x? mpReal? A real number. The numeric value at which to evaluate the distribution}
%{a? mpNum? A real number greater 0, a parameter to the distribution}
%{b? mpNum? A real number greater 0, a parameter to the distribution}
%{Cumulative ? Boolean? A logical value that determines the form of the function. If cumulative is TRUE, GAMMA.DIST returns the cumulative distribution function; if FALSE, it returns the probability density function.}
%\end{mpFunctionsExtract}
%
%\vspace{0.6cm}
%\begin{mpFunctionsExtract}
%\mpWorksheetFunctionFour
%{GAMMA.DIST? mpReal? the CDF and of the central Gamma-distribution}
%{x? mpReal? A real number. The numeric value at which to evaluate the distribution}
%{a? mpNum? A real number greater 0, a parameter to the distribution}
%{b? mpNum? A real number greater 0, a parameter to the distribution}
%{Cumulative ? Boolean? A logical value that determines the form of the function. If cumulative is TRUE, GAMMA.DIST returns the cumulative distribution function; if FALSE, it returns the probability density function.}
%\end{mpFunctionsExtract}


\subsubsection{Density}
\label{GammaDistributionDensity}

\begin{equation} 
	f(x; a, b)= \frac{x^{a-1}e^{-x/b}}{\Gamma(a) b^a}
\end{equation}

\subsubsection{CDF: General formulas}
%\label{GAMMADIST} \index{Spreadsheet Functions!GAMMADIST}
%\label{GAMMA.DIST} \index{Spreadsheet Functions!GAMMA.DIST}
%\begin{tabular}{p{481pt}}
%\toprule
%\textsf{Function \textbf{GammaDist}($\boldsymbol{a}\ As\ mpNum$, $\boldsymbol{b}\ As\ mpNum$) As mpNum}\index{Multiprecision Functions!GammaDist} \\
%\textsf{Function \textbf{GAMMADIST}($\boldsymbol{a}\ As\ mpNum$, $\boldsymbol{b}\ As\ mpNum$) As mpNum}\index{Multiprecision Functions!GAMMADIST} \\
%\textsf{Function \textbf{GAMMA.DIST}($\boldsymbol{a}\ As\ mpNum$, $\boldsymbol{b}\ As\ mpNum$) As mpNum}\index{Multiprecision Functions!GAMMA.DIST} \\
%\bottomrule
%\end{tabular}

\vspace{0.3cm}
\begin{equation} 
	F(x; a, b)= P(a,x/b) = igammap(a,x/b)
\end{equation}

\subsection{Quantiles}

\begin{mpFunctionsExtract}
	\mpFunctionThree
	{GammaDistInv? mpNumList? returns quantiles and related information for the the central Gamma-distribution}
	{Prob? mpNum? A real number between 0 and 1.}
	{m? mpNum? A real number greater 0, a parameter to the distribution}
	{n? mpNum? A real number greater 0, a parameter to the distribution}
	{Output? String? A string describing the output choices}
\end{mpFunctionsExtract}

See section \ref{Functions returning Quantiles} for the options for  {\itshape\sffamily Prob} and {\itshape\sffamily Output}). 

%\vspace{0.3cm}
%
%The following functions are provided for compatibility with established spreadsheet functions
%
%\vspace{0.3cm}
%\begin{mpFunctionsExtract}
%\mpWorksheetFunctionThree
%{GAMMAINV? mpReal? the two-tailed inverse of the central Gamma-distribution}
%{Prob? mpReal? A real number}
%{a? mpNum? A real number greater 0, a parameter to the distribution}
%{b? mpNum? A real number greater 0, a parameter to the distribution}
%\end{mpFunctionsExtract}
%
%\vspace{0.6cm}
%\begin{mpFunctionsExtract}
%\mpWorksheetFunctionThree
%{GAMMA.INV? mpReal? the left-tailed inverse of the central Gamma-distribution}
%{Prob? mpReal? A real number}
%{a? mpNum? A real number greater 0, a parameter to the distribution}
%{b? mpNum? A real number greater 0, a parameter to the distribution}
%\end{mpFunctionsExtract}




%
%
%\subsection{Quantiles}
%\label{GAMMAINV} \index{Spreadsheet Functions!GAMMAINV}
%\label{GAMMA.INV} \index{Spreadsheet Functions!GAMMA.INV}
%\begin{tabular}{p{481pt}}
%\toprule
%\textsf{Function \textbf{GammaDistInv}($\boldsymbol{a}\ As\ mpNum$, $\boldsymbol{b}\ As\ mpNum$) As mpNum}\index{Multiprecision Functions!GammaDistInv} \\
%\textsf{Function \textbf{GAMMAINV}($\boldsymbol{a}\ As\ mpNum$, $\boldsymbol{b}\ As\ mpNum$) As mpNum}\index{Multiprecision Functions!GAMMAINV} \\
%\textsf{Function \textbf{GAMMA.INV}($\boldsymbol{a}\ As\ mpNum$, $\boldsymbol{b}\ As\ mpNum$) As mpNum}\index{Multiprecision Functions!GAMMA.INV} \\
%\bottomrule
%\end{tabular}

\vspace{0.3cm}
\begin{equation} 
	F^{-1}(y)= b \cdot igammapInv(a,y)
\end{equation}



\subsection{Properties}
\label{GammaDistributionProperties}


\begin{mpFunctionsExtract}
	\mpFunctionTwo
	{GammaDistInfo? mpNumList? returns moments and related information for the central Gamma-distribution}
	{a? mpNum? A real number greater 0, representing the degrees of freedom}
	{b? mpNum? A real number greater 0, representing the degrees of freedom}
	{Output? String? A string describing the output choices}
\end{mpFunctionsExtract}

\vspace{0.3cm}

See section \ref{Functions returning moments and related information} for the options for {\itshape\sffamily Output}. Algorithms and formulas are given in section \ref{tDistributionProperties}.



\subsubsection{Moments}
The algebraic moments are given by (Wolfram)
\begin{equation} 
	\mu'_r = \frac{b^r \Gamma(a+r)}{\Gamma(a)}
\end{equation}



%
%\subsection{Random Numbers}
%\begin{tabular}{p{481pt}}
%\toprule
%\textsf{Function \textbf{GammaDistRandom}($\boldsymbol{a}\ As\ mpNum$, $\boldsymbol{b}\ As\ mpNum$) As mpNum}\index{Multiprecision Functions!GammaDistRandom} \\
%\bottomrule
%\end{tabular}
%
%\vspace{0.3cm}
%


\subsection{Random Numbers}

\begin{mpFunctionsExtract}
	\mpFunctionFive
	{GammaDistRandom? mpNumList? returns random numbers following a central Beta-distribution}
	{Size? mpNum? A positive integer up to $10^7$}
	{a? mpNum? A real number greater 0, a parameter to the distribution}
	{b? mpNum? A real number greater 0, a parameter to the distribution}
	{Generator? String? A string describing the random generator}
	{Output? String? A string describing the output choices}
\end{mpFunctionsExtract}

\vspace{0.3cm}

See section \ref{Functions returning Random numbers} for the options for  {\itshape\sffamily Size},  {\itshape\sffamily Generator} and {\itshape\sffamily Output}. Algorithms and formulas are given below.



\subsubsection{Random Numbers: algorithms and formulas}
In the case of an Erlangian distribution ($b$ a positive integer) we obtain a random number
by adding $b$ independent random numbers from an exponential distribution i.e.

$x = - \ln(\xi_1 \cdot \xi_2 \cdot \ldots \cdot \xi_b)/a$

where all the $\xi_i$ are uniform random numbers in the interval from zero to one. Note that
care must be taken if $b$ is large in which case the product of uniform random numbers may
become zero due to machine precision. In such cases simply divide the product in pieces
and add the logarithms afterwards.

\subsubsection{General case}
In a more general case we use the so called Johnk's algorithm

\begin{enumerate}
	\item Denote the integer part of $b$ with $i$ and the fractional part with $f$ and put $r = 0$. Let $\xi$ denote uniform random numbers in the interval from zero to one.
	\item If $i > 0$ then put $r = - \ln(\xi_1 \cdot \xi_2 \cdot \ldots \cdot \xi_i)$.
	\item If $f = 0$ then go to 7.
	\item Calculate $w_1 = \xi_{i+1}^{1/f}$ and $w_1 = \xi_{i+2}^{1/(1-f)}$ .
	\item If $w_1 + w_2 > 1$ then go back to iv.
	\item Put $r = r - \ln(\xi_{i+3}) \cdot \frac{w_1}{w_1+w_2}$.
	\item Quit with $r = r/a$.
\end{enumerate}






\section{Hypergeometric Distribution}
\label{HypergeometricDistribution}


See \cite{Upton_1982}, \cite{Harkness_1964}

See \cite{ling_accuracy_1984}

See \cite{Knüsel_1987}

See also \cite{Conlon_1993}

See also \cite{Casagrande_1978}

\subsection{Definition}
\label{HypergeometricDistributionDefinition}

These functions return PMF and CDF of the (discrete) hypergeometric distribution;
the PMF gives the probability that among $n$ randomly chosen samples from a container
with $n_1$ type1 objects and $n_2$ type2 objects there are exactly $k$ type1 objects.



\subsection{Density and CDF}

\begin{mpFunctionsExtract}
	\mpFunctionFive
	{HypergeometricDist? mpNumList? returns pdf, CDF and related information for the central hypergeometric distribution}
	{x? mpNum? The number of successes in the sample.}
	{n? mpNum? The size of the sample.}
	{M? mpNum? The number of successes in the population}
	{N? mpNum? The population size}
	{Output? String? A string describing the output choices}
\end{mpFunctionsExtract}


\vspace{0.3cm}
See section \ref{Functions returning pdf, CDF, and related information} for the options for {\itshape\sffamily Output}. Algorithms and formulas are given in sections \ref{BetaDistributionDensity} and \ref{BetaDistributionCDF}.


%\vspace{0.3cm}
%
%The following functions are provided for compatibility with established spreadsheet functions
%
%\vspace{0.6cm}
%\begin{mpFunctionsExtract}
%\mpWorksheetFunctionFive
%{HYPGEOMDIST? mpReal? pdf, CDF, and related information of the central hypergeometric distribution}
%{x? mpNum? The number of successes in the sample.}
%{n? mpNum? The size of the sample.}
%{M? mpNum? The number of successes in the population}
%{N? mpNum? The population size}
%{Cumulative ? Boolean? A logical value that determines the form of the function. If cumulative is TRUE, T.DIST returns the cumulative distribution function; if FALSE, it returns the probability density function}
%\end{mpFunctionsExtract}
%
%
%\vspace{0.6cm}
%\begin{mpFunctionsExtract}
%\mpWorksheetFunctionFive
%{HYPGEOM.DIST? mpReal? the CDF and pdf of the central hypergeometric distribution}
%{x? mpNum? The number of successes in the sample.}
%{n? mpNum? The size of the sample.}
%{M? mpNum? The number of successes in the population}
%{N? mpNum? The population size}
%{Cumulative ? Boolean? A logical value that determines the form of the function. If cumulative is TRUE, T.DIST returns the cumulative distribution function; if FALSE, it returns the probability density function}
%\end{mpFunctionsExtract}




\subsubsection{Density}
\label{HypergeometricDistributionDensity}

\begin{equation} 
	f(k) = \frac{\binom{n_1}{k} \binom{n_2}{n-k}}{\binom{n_1+n_2}{n}}, \quad (n,n_1,n_2 \geq 0; n \leq n_1+n_2).
\end{equation}

$f(k)$ is computed with the R trick [39], which replaces the binomial coefficients by
binomial PMFs with $p = n/(n1 + n2)$.


\subsubsection{CDF}
There is no explicit formula for the CDF, it is calculated as $\sum f(i)$, using the lower tail if $k < nn_1/(n_1 + n_2)$ and the upper tail otherwise with one value of the PMF and the recurrence formulas:

\begin{equation} 
	f(k+1)= \frac{(n_1 - k)(n-k)}{(k+1)(n_2 - n+k+1} f(k)
\end{equation}

\begin{equation} 
	f(k-1)= \frac{k(n_2 - n + k)}{(n_1 - k+1)(n-k+1} f(k)
\end{equation}

\subsection{Quantiles}


\begin{mpFunctionsExtract}
	\mpFunctionFive
	{HypergeometricDistInv? mpNumList? returns quantiles and related information for the the central hypergeometric distribution}
	{Prob? mpNum? A real number between 0 and 1.}
	{n? mpNum? The size of the sample.}
	{M? mpNum? The number of successes in the population}
	{N? mpNum? The population size}
	{Output? String? A string describing the output choices}
\end{mpFunctionsExtract}

\vspace{0.3cm}
See section \ref{Functions returning Quantiles} for the options for  {\itshape\sffamily Prob} and {\itshape\sffamily Output}). 

\subsection{Sample Size}

See \cite{guenther_sample_1974}

\subsection{Properties}
\label{HypergeometricDistributionProperties}


\begin{mpFunctionsExtract}
	\mpFunctionFour
	{HypergeometricDistInfo? mpNumList? returns moments and related information for the central hypergeometric distribution}
	{n? mpNum? The size of the sample.}
	{M? mpNum? The number of successes in the population}
	{N? mpNum? The population size}
	{Output? String? A string describing the output choices}
\end{mpFunctionsExtract}

\vspace{0.3cm}

See section \ref{Functions returning moments and related information} for the options for {\itshape\sffamily Output}. Algorithms and formulas are given in section \ref{tDistributionProperties}.

\subsubsection{Moments}

\begin{equation} 
	\mu_1 = nP
\end{equation}

\begin{equation} 
	\mu_2 = nPQ \frac{N-n}{N-1}
\end{equation}

\begin{equation} 
	\mu_3 = nPQ (Q-P) \frac{(N-n)(N-2n)}{(N-1)(N-2)}
\end{equation}

\begin{equation} 
	\kappa_4 = \frac{6nP^2Q^2(N-n)}{N-1} \frac{n(N-n)(5N-6)-N(N-1)}{(N-2)(N-3)}
\end{equation}





\subsection{Random Numbers}

\begin{mpFunctionsExtract}
	\mpFunctionSix
	{HypergeometricDistRandom? mpNumList? returns random numbers following a central hypergeometric distribution}
	{Size? mpNum? A positive integer up to $10^7$}
	{n? mpNum? The size of the sample.}
	{M? mpNum? The number of successes in the population}
	{N? mpNum? The population size}
	{Generator? String? A string describing the random generator}
	{Output? String? A string describing the output choices}
\end{mpFunctionsExtract}

\vspace{0.3cm}

See section \ref{Functions returning Random numbers} for the options for  {\itshape\sffamily Size},  {\itshape\sffamily Generator} and {\itshape\sffamily Output}. Algorithms and formulas are given below.




\section{Lognormal Distribution}
\label{LognormalDistribution}


\subsection{Definition}
\label{LognormalDistributionDefinition}


These functions return PDF, CDF, and ICDF of the lognormal distribution with location
$a$, scale $b > 0$, and the support interval $(0,+\infty)$ :


A log-normal (or lognormal) distribution is a continuous probability distribution of a random variable whose logarithm is normally distributed. Thus, if the random variable  is log-normally distributed, then  has a normal distribution. Likewise, if  has a normal distribution, then  has a log-normal distribution. A random variable which is log-normally distributed takes only positive real values.

In a log-normal distribution $X$, the parameters denoted $\mu$ and $\sigma$ are, respectively, the mean and standard deviation of the variable's natural logarithm (by definition, the variable's logarithm is normally distributed), which means

\begin{equation}
	X = e^{\mu+\sigma Z}
\end{equation}

with $Z$ a standard normal variable.

This relationship is true regardless of the base of the logarithmic or exponential function. If $\log_a(Y)$ is normally distributed, then so is $\log_b(Y)$, for any two positive numbers $a$, $b \neq 1$. Likewise, if $e^X$ is log-normally distributed, then so is $a^X$, where  is $a$ positive number $\neq 1$.

On a logarithmic scale, $\mu$ and $\sigma$ can be called the location parameter and the scale parameter, respectively.

In contrast, the mean, standard deviation, and variance of the non-logarithmized sample values are respectively denoted $m$, $s.d$., and $v$ in this article. The two sets of parameters can be related as

\begin{equation}
	\mu = \ln \left(\frac{m^2}{\sqrt{v+m^2}}\right), \quad \sigma = \sqrt{\ln\left(1+\frac{v}{m^2} \right)}
\end{equation}


\subsection{Density and CDF}

\begin{mpFunctionsExtract}
	\mpFunctionFour
	{LogNormalDist? mpNumList? returns pdf, CDF and related information for the Lognormal-distribution}
	{x? mpNum? A real number}
	{mean? mpNum? A real number greater 0, representing the mean of the distribution}
	{stdev? mpNum? A real number greater 0, representing the standard deviation of the distribution}
	{Output? String? A string describing the output choices}
\end{mpFunctionsExtract}


\vspace{0.3cm}
See section \ref{Functions returning pdf, CDF, and related information} for the options for {\itshape\sffamily Output}. Algorithms and formulas are given in sections \ref{BetaDistributionDensity} and \ref{BetaDistributionCDF}.


%\vspace{0.3cm}
%
%The following functions are provided for compatibility with established spreadsheet functions
%
%\vspace{0.3cm}
%\begin{mpFunctionsExtract}
%\mpWorksheetFunctionThree
%{LOGNORMDIST? mpReal? the CDF and of the Lognormal-distribution}
%{x? mpReal? A real number. The numeric value at which to evaluate the distribution}
%{mean? mpNum? A real number greater 0, representing the mean of the distribution}
%{stdev? mpNum? A real number greater 0, representing the standard deviation of the distribution}
%\end{mpFunctionsExtract}
%
%\vspace{0.6cm}
%\begin{mpFunctionsExtract}
%\mpWorksheetFunctionFour
%{LOGNORM.DIST? mpReal? the CDF and of the Lognormal-distribution}
%{x? mpReal? A real number. The numeric value at which to evaluate the distribution}
%{mean? mpNum? A real number greater 0, representing the mean of the distribution}
%{stdev? mpNum? A real number greater 0, representing the standard deviation of the distribution}
%{Cumulative ? Boolean? A logical value that determines the form of the function. If cumulative is TRUE, T.DIST returns the cumulative distribution function; if FALSE, it returns the probability density function}
%\end{mpFunctionsExtract}




\subsubsection{Density}
\label{LognormalDistributionDensity}

\begin{equation} 
	f(x)= \frac{1}{b x \sqrt{2\pi}} \exp \left(- \frac{(\ln(x) - a)^2}{2b^2}\right)
\end{equation}


\subsubsection{CDF}
\label{LognormalDistributionCDF}
\begin{equation} 
	F(x)= \frac{1}{2}  \left(1+\text{erf} \left( \frac{\ln(x) - a}{b\sqrt{2}}\right)\right)
\end{equation}



\subsection{Quantiles}
\label{LognormalDistributionQuantiles}


\begin{mpFunctionsExtract}
	\mpFunctionFour
	{LognormalDistInv? mpNumList? returns quantiles and related information for the the Lognormal-distribution}
	{Prob? mpNum? A real number between 0 and 1.}
	{mean? mpNum? A real number greater 0, representing the mean of the distribution}
	{stdev? mpNum? A real number greater 0, representing the standard deviation of the distribution}
	{Output? String? A string describing the output choices}
\end{mpFunctionsExtract}

See section \ref{Functions returning Quantiles} for the options for  {\itshape\sffamily Prob} and {\itshape\sffamily Output}). 

%\vspace{0.3cm}
%
%The following functions are provided for compatibility with established spreadsheet functions
%
%\vspace{0.3cm}
%\begin{mpFunctionsExtract}
%\mpWorksheetFunctionThree
%{LOGINV? mpReal? the two-tailed inverse of the Lognormal-distribution}
%{Prob? mpReal? A real number}
%{mean? mpNum? A real number greater 0, representing the mean of the distribution}
%{stdev? mpNum? A real number greater 0, representing the standard deviation of the distribution}
%\end{mpFunctionsExtract}
%
%\vspace{0.6cm}
%\begin{mpFunctionsExtract}
%\mpWorksheetFunctionThree
%{LOGNORM.INV? mpReal? the left-tailed inverse of the Lognormal-distribution}
%{Prob? mpReal? A real number}
%{mean? mpNum? A real number greater 0, representing the mean of the distribution}
%{stdev? mpNum? A real number greater 0, representing the standard deviation of the distribution}
%\end{mpFunctionsExtract}


\subsubsection{Quantiles: algorithms and formulas}

\begin{equation} 
	F^{-1}(y)= \exp(a+b \cdot \text{normstdinv}(y))
\end{equation}



\subsection{Properties}
\label{LognormalDistributionProperties}


\begin{mpFunctionsExtract}
	\mpFunctionThree
	{LognormalDistInfo? mpNumList? returns moments and related information for the central Lognormal-distribution}
	{mean? mpNum? A real number greater 0, representing the mean of the distribution}
	{stdev? mpNum? A real number greater 0, representing the standard deviation of the distribution}
	{Output? String? A string describing the output choices}
\end{mpFunctionsExtract}

\vspace{0.3cm}

See section \ref{Functions returning moments and related information} for the options for {\itshape\sffamily Output}. Algorithms and formulas are given in section \ref{tDistributionProperties}.




\subsubsection{Moments: algorithms and formulas}
Algebraic moments of the log-normal distribution are given by
\begin{equation} 
	\mu'_k= e^{k\mu+k^2\sigma^2 /2}
\end{equation}


\subsection{Random Numbers}
\label{LognormalDistributionRandom}


\begin{mpFunctionsExtract}
	\mpFunctionFive
	{LognormalRandom? mpNumList? returns random numbers following a central Beta-distribution}
	{Size? mpNum? A positive integer up to $10^7$}
	{mean? mpNum? A real number greater 0, representing the mean of the distribution}
	{stdev? mpNum? A real number greater 0, representing the standard deviation of the distribution}
	{Generator? String? A string describing the random generator}
	{Output? String? A string describing the output choices}
\end{mpFunctionsExtract}

\vspace{0.3cm}

See section \ref{Functions returning Random numbers} for the options for  {\itshape\sffamily Size},  {\itshape\sffamily Generator} and {\itshape\sffamily Output}. Algorithms and formulas are given in section \ref{FDistributionRandom}.


\subsubsection{Random Numbers: algorithms and formulas}
The most straightforward way of achieving random numbers from a log-normal distribution is to generate a random number $u$ from a normal distribution with mean $\mu$ and standard deviation $\sigma$ and construct $r = e^u$.




\section{Negative Binomial Distribution}
\label{NegativBinomialDistribution}

These functions return PMF and CDF of the (discrete) negative binomial distribution with target for number of successful trials $r > 0$ and success probability $0 \leq  p \leq 1$.

If $r = n$ is a positive integer the name Pascal distribution is used, and for $r = 1$ it is called geometric distribution.

See \cite{ong_non-central_1979} for information on the noncentral negative binomial distribution

\subsection{Density and CDF}

\begin{mpFunctionsExtract}
	\mpFunctionFour
	{NegativeBinomialDist? mpNumList? returns pdf, CDF and related information for the central negative binomial distribution}
	{x? mpNum? The number of failures in trials.}
	{r? mpNum? The threshold number of successes.}
	{p? mpNum? The probability of a success}
	{Output? String? A string describing the output choices}
\end{mpFunctionsExtract}


\vspace{0.3cm}
See section \ref{Functions returning pdf, CDF, and related information} for the options for {\itshape\sffamily Output}. Algorithms and formulas are given in sections \ref{BinomialDistributionDensity} and \ref{BinomialDistributionCDF}.


%\vspace{0.3cm}
%
%The following functions are provided for compatibility with established spreadsheet functions
%
%\vspace{0.6cm}
%\begin{mpFunctionsExtract}
%\mpWorksheetFunctionFour
%{NEGBINOMDIST? mpReal? pdf, CDF, and related information of the central negative binomial distribution}
%{x? mpNum? The number of failures in trials.}
%{r? mpNum? The threshold number of successes.}
%{p? mpNum? The probability of a success}
%{Cumulative ? Boolean? A logical value that determines the form of the function. If cumulative is TRUE, T.DIST returns the cumulative distribution function; if FALSE, it returns the probability density function}
%\end{mpFunctionsExtract}
%
%
%\vspace{0.6cm}
%\begin{mpFunctionsExtract}
%\mpWorksheetFunctionFour
%{NEGBINOM.DIST? mpReal? the CDF and pdf of the central negative binomial distribution}
%{x? mpNum? The number of failures in trials.}
%{r? mpNum? The threshold number of successes.}
%{p? mpNum? The probability of a success}
%{Cumulative ? Boolean? A logical value that determines the form of the function. If cumulative is TRUE, T.DIST returns the cumulative distribution function; if FALSE, it returns the probability density function}
%\end{mpFunctionsExtract}




\subsubsection{Density}
\nomenclature{$f_{\text{NegBin}}(n, k; p)$}{pmf of the negative binomial distribution}
\label{NegativBinomialDistributionDensity}

\begin{equation} 
	f_{\text{NegBin}}(r, k; p) = \frac{\Gamma(k+r)}{k! \Gamma(r)} p^r (1-p)^k = \frac{p}{r+k} f_{\text{Beta}}(r,k+1,p)
\end{equation}


\subsubsection{CDF}
\nomenclature{$F_{\text{NegBin}}(n, k; p)$}{CDF of the negative binomial distribution}
\vspace{0.3cm}
\begin{equation} 
	F_{\text{NegBin}}(r, k; p)= I_{1-p}(r,k+1) = ibeta(r,k+1,1-p)
\end{equation}



\subsection{Quantiles}

\begin{mpFunctionsExtract}
	\mpFunctionFour
	{NegativeBinomialDistInv? mpNumList? returns quantiles and related information for the the central binomial-distribution}
	{Prob? mpNum? A real number between 0 and 1.}
	{r? mpNum? The threshold number of successes.}
	{p? mpNum? The probability of a success}
	{Output? String? A string describing the output choices}
\end{mpFunctionsExtract}

\vspace{0.3cm}
See section \ref{Functions returning Quantiles} for the options for  {\itshape\sffamily Prob} and {\itshape\sffamily Output}). 



\subsection{Properties}
\label{NegativBinomialDistributionProperties}


\begin{mpFunctionsExtract}
	\mpFunctionThree
	{NegativeBinomialDistInfo? mpNumList? returns moments and related information for the central Binomial-distribution}
	{r? mpNum? The threshold number of successes.}
	{p? mpNum? The probability of a success}
	{Output? String? A string describing the output choices}
\end{mpFunctionsExtract}

\vspace{0.3cm}

See section \ref{Functions returning moments and related information} for the options for {\itshape\sffamily Output}. Algorithms and formulas are given in section \ref{tDistributionProperties}.



\subsubsection{Moments: algorithms and formulas}

\begin{equation} 
	\mu_1 = np
\end{equation}

\begin{equation} 
	\mu_2 = np(1-p) = npq
\end{equation}

\begin{equation} 
	\mu_3 = npq(q+p)
\end{equation}

\begin{equation} 
	\mu_4 = npq(3npq+6pq+1)
\end{equation}



\subsubsection{Recurrence relations}
The following recurrence relations hold:
\begin{equation} 
	f_{\text{NegBin}}(r, k+1; p) = \frac{(r+k)(1-p)}{k+1} f_{\text{NegBin}}(r, k; p)
\end{equation}
\begin{equation} 
	f_{\text{NegBin}}(r, k-1; p) = \frac{k}{(r+k-1)(1-p)} f_{\text{NegBin}}(r, k; p)
\end{equation}



\subsection{Random Numbers}

\begin{mpFunctionsExtract}
	\mpFunctionFive
	{NegativeBinomialDistRandom? mpNumList? returns random numbers following a central Binomial-distribution}
	{Size? mpNum? A positive integer up to $10^7$}
	{r? mpNum? The threshold number of successes.}
	{p? mpNum? The probability of a success}
	{Generator? String? A string describing the random generator}
	{Output? String? A string describing the output choices}
\end{mpFunctionsExtract}

\vspace{0.3cm}

See section \ref{Functions returning Random numbers} for the options for  {\itshape\sffamily Size},  {\itshape\sffamily Generator} and {\itshape\sffamily Output}. Algorithms and formulas are given below.

\subsubsection{Random Numbers: algorithms and formulas}
Random numbers from a negative binomial distribution can be obtained using the algorithms outline for the beta distribution.




\section{Normal Distribution}
\label{sec:NormalDistribution}


\subsection{Definition}
\label{sec:NormalDistributionDefinition}
A random variable is said to follow a normal distribution with  mean $\mu$ and variance $\sigma^2$, if its pdf is given by \ref{eq:Normal_pdf}. It is said to follow a  standardized normal distribution if its pdf is given by \ref{eq:StandardNormal_pdf}.


\subsection{Density and CDF}

\begin{mpFunctionsExtract}
	\mpFunctionFour
	{NDist? mpNumList? returns pdf, CDF and related information for the normal-distribution}
	{x? mpNum? A real number}
	{mean? mpNum? A real number greater 0, representing the mean of the distribution}
	{stdev? mpNum? A real number greater 0, representing the standard deviation of the distribution}
	{Output? String? A string describing the output choices}
\end{mpFunctionsExtract}


\vspace{0.3cm}
See section \ref{Functions returning pdf, CDF, and related information} for the options for {\itshape\sffamily Output}. Algorithms and formulas are given in sections \ref{BetaDistributionDensity} and \ref{BetaDistributionCDF}.


%\vspace{0.6cm}
%
%The following functions are provided for compatibility with established spreadsheet functions
%
%\vspace{0.3cm}
%\begin{mpFunctionsExtract}
%\mpWorksheetFunctionThree
%{NORMDIST? mpReal? the CDF and of the Lognormal-distribution}
%{x? mpReal? A real number. The numeric value at which to evaluate the distribution}
%{mean? mpNum? A real number greater 0, representing the mean of the distribution}
%{stdev? mpNum? A real number greater 0, representing the standard deviation of the distribution}
%\end{mpFunctionsExtract}
%
%\vspace{0.6cm}
%\begin{mpFunctionsExtract}
%\mpWorksheetFunctionFour
%{NORM.DIST? mpReal? the CDF and of the Lognormal-distribution}
%{x? mpReal? A real number. The numeric value at which to evaluate the distribution}
%{mean? mpNum? A real number greater 0, representing the mean of the distribution}
%{stdev? mpNum? A real number greater 0, representing the standard deviation of the distribution}
%{Cumulative ? Boolean? A logical value that determines the form of the function. If cumulative is TRUE, T.DIST returns the cumulative distribution function; if FALSE, it returns the probability density function}
%\end{mpFunctionsExtract}


%\vspace{0.6cm}
%\begin{mpFunctionsExtract}
%\mpWorksheetFunctionOne
%{NORMSDIST? mpReal? the CDF and of the  standard normal distribution}
%{x? mpReal? A real number. The numeric value at which to evaluate the distribution}
%\end{mpFunctionsExtract}
%
%\vspace{0.6cm}
%\begin{mpFunctionsExtract}
%\mpWorksheetFunctionTwo
%{NORM.S.DIST? mpReal? the CDF and of the  standard normal distribution}
%{x? mpReal? A real number. The numeric value at which to evaluate the distribution}
%{Cumulative ? Boolean? A logical value that determines the form of the function. If cumulative is TRUE, NORM.S.DIST returns the cumulative distribution function; if FALSE, it returns the probability density function}
%\end{mpFunctionsExtract}


%\vspace{0.6cm}
%\begin{mpFunctionsExtract}
%\mpWorksheetFunctionOne
%{GAUSS? mpReal? the CDF of the  standard normal distribution}
%{x? mpReal? A real number. The numeric value at which to evaluate the distribution}
%\end{mpFunctionsExtract}
%
%
%\vspace{0.6cm}
%\begin{mpFunctionsExtract}
%\mpWorksheetFunctionOne
%{PHI? mpReal? the pdf of the  standard normal distribution}
%{x? mpReal? A real number. The numeric value at which to evaluate the distribution}
%\end{mpFunctionsExtract}




\subsubsection{Density}
\label{sec:NormalDistribution_pdf}
\nomenclature{$\phi(x)$}{pdf of the standardized normal distribution}
\nomenclature{$F_N(x; \mu, \sigma^2)$}{pdf of the normal distribution with mean $\mu$ and variance $\sigma^2$}%

\vspace{0.3cm}
This functions returns the pdf of the normal distribution with mean $\mu$ and variance $\sigma^2$, which is given by

\begin{equation} \label{eq:Normal_pdf}
	f_N(x; \mu, \sigma^2) = \frac{1}{\sigma \sqrt{2\pi}} e^{- \frac{1}{2} \left(\frac{x-\mu}{\sigma}\right)^2}
\end{equation}

The pdf of the standardized normal distribution with mean $0$ and variance $1$ is given by

\begin{equation} \label{eq:StandardNormal_pdf}
	\phi(u) = \frac{1}{\sqrt{2\pi}} e^{- \frac{1}{2} u^2}, 
\end{equation}

These two functions are related by

\begin{equation} 
	f_N(x; \mu, \sigma^2) =  \frac{1}{\sigma} \phi \left(\frac{x-\mu}{\sigma} \right), \text{and} \quad  \phi(u) = \sigma f_N(\mu + \sigma u)
\end{equation}


\subsubsection{CDF}
\label{sec:NormalDistribution_CDF}
\nomenclature{$\Phi(x)$}{CDF of the standardized normal distribution}%
\nomenclature{$F_N(x; \mu, \sigma^2)$}{CDF of the normal distribution with mean $\mu$ and variance $\sigma^2$}
This functions returns the cdf of the normal distribution with mean $\mu$ and variance $\sigma^2$, which is given by
\begin{equation}
	F_N(x; \mu, \sigma^2) = \int_{-\infty}^x f_N(v) dv
\end{equation}

The cdf of the standardized normal distribution with mean $0$ and variance $1$ is given by
\begin{equation}
	\Phi(u) = \int_{-\infty}^u \phi(w) dw
\end{equation}

These two functions are related by

\begin{equation} 
	F_N(x; \mu, \sigma^2) =  \Phi \left(\frac{x-\mu}{\sigma} \right), \text{and} \quad  \Phi(u) = F_N(\mu + \sigma u)
\end{equation}




\subsection{Quantiles}
\label{sec:NormalDistribution_Quantiles}
\nomenclature{$\Phi^{-1}(\alpha)$}{Inverse CDF of the standardized normal distribution}
\nomenclature{$z_{\alpha}$}{$\alpha$ quantile of the standardized normal distribution}
\nomenclature{$F_N^{-1}(\alpha; \mu, \sigma^2)$}{Inverse CDF of the normal distribution with mean $\mu$ and variance $\sigma^2$}

These functions return the quantile of the normal distribution with  mean $\mu$ and variance $\sigma^2$, $F_N^{-1}(\alpha; \mu, \sigma^2)$, or the standardized normal distribution with mean $0$ and variance $1$, $\Phi^{-1}(\alpha)$.

\vspace{0.3cm}
\begin{mpFunctionsExtract}
	\mpFunctionFour
	{NDistInv? mpNumList? returns quantiles and related information for the the Lognormal-distribution}
	{Prob? mpNum? A real number between 0 and 1.}
	{mean? mpNum? A real number greater 0, representing the mean of the distribution}
	{stdev? mpNum? A real number greater 0, representing the standard deviation of the distribution}
	{Output? String? A string describing the output choices}
\end{mpFunctionsExtract}

See section \ref{Functions returning Quantiles} for the options for  {\itshape\sffamily Prob} and {\itshape\sffamily Output}). 


%\vspace{0.6cm}
%
%The following functions are provided for compatibility with established spreadsheet functions
%
%\vspace{0.3cm}
%\begin{mpFunctionsExtract}
%\mpWorksheetFunctionThree
%{NORMINV? mpReal? the two-tailed inverse of the normal distribution}
%{Prob? mpReal? A real number}
%{mean? mpNum? A real number greater 0, representing the mean of the distribution}
%{stdev? mpNum? A real number greater 0, representing the standard deviation of the distribution}
%\end{mpFunctionsExtract}
%
%\vspace{0.6cm}
%\begin{mpFunctionsExtract}
%\mpWorksheetFunctionThree
%{NORM.INV? mpReal? the left-tailed inverse of the normal distribution}
%{Prob? mpReal? A real number}
%{mean? mpNum? A real number greater 0, representing the mean of the distribution}
%{stdev? mpNum? A real number greater 0, representing the standard deviation of the distribution}
%\end{mpFunctionsExtract}


%\vspace{0.6cm}
%\begin{mpFunctionsExtract}
%\mpWorksheetFunctionOne
%{NORMSINV? mpReal? the two-tailed inverse of the standardized normal distribution}
%{Prob? mpReal? A real number}
%\end{mpFunctionsExtract}
%
%\vspace{0.6cm}
%\begin{mpFunctionsExtract}
%\mpWorksheetFunctionOne
%{NORM.S.INV? mpReal? the left-tailed inverse of the standardized normal distribution}
%{Prob? mpReal? A real number}
%\end{mpFunctionsExtract}


\subsubsection{Quantiles: algorithms and formulas}

\begin{equation} 
	F^{-1}(y)= \exp(a+b \cdot \text{normstdinv}(y))
\end{equation}




\subsection{Properties}
\label{NormalDistributionProperties}


\begin{mpFunctionsExtract}
	\mpFunctionThree
	{NormalDistInfo? mpNumList? returns moments and related information for the central Lognormal-distribution}
	{mean? mpNum? A real number greater 0, representing the mean of the distribution}
	{stdev? mpNum? A real number greater 0, representing the standard deviation of the distribution}
	{Output? String? A string describing the output choices}
\end{mpFunctionsExtract}

\vspace{0.3cm}

See section \ref{Functions returning moments and related information} for the options for {\itshape\sffamily Output}. Algorithms and formulas are given in section \ref{tDistributionProperties}.




\subsubsection{Moments: algorithms and formulas}
$\kappa_1 = \mu$

$\kappa_2 = \sigma^2$

$\kappa_r = 0$ for $r \geq 3$.

\subsubsection{Differential Equation}
\label{Differential Equation}
Let $Z^{(m)}$ denote the $m^{th}$ derivative of $Z(x)$. Then \citep{abramowitz_handbook_1970}
\begin{equation}
	Z^{(1)} = -x Z(x)
\end{equation}
\begin{equation}
	Z^{(m+2)} +xZ^{(m+1)} + (m+1)Z^{(m)}  = 0
\end{equation}




\subsection{Random Numbers}
\label{NormalDistributionRandom}


\begin{mpFunctionsExtract}
	\mpFunctionFive
	{NormalRandom? mpNumList? returns random numbers following a central Beta-distribution}
	{Size? mpNum? A positive integer up to $10^7$}
	{mean? mpNum? A real number greater 0, representing the mean of the distribution}
	{stdev? mpNum? A real number greater 0, representing the standard deviation of the distribution}
	{Generator? String? A string describing the random generator}
	{Output? String? A string describing the output choices}
\end{mpFunctionsExtract}

\vspace{0.3cm}

See section \ref{Functions returning Random numbers} for the options for  {\itshape\sffamily Size},  {\itshape\sffamily Generator} and {\itshape\sffamily Output}. Algorithms and formulas are given in section \ref{FDistributionRandom}.


\subsubsection{Random Numbers: algorithms and formulas}
Let $Z_1 \sim Re(0;1), Z_2 \sim Re(0,1)$ be independent random variables. Then

\vspace{0.3cm}
$X_1 = \sqrt{-2 \ln Z_1} \cos(2 \pi Z_2)$ and $X_2 = \sqrt{-2 \ln Z_1} \sin(2 \pi Z_2)$ are $\sim No(0;1)$.


\vspace{0.3cm}
It is also possible to directly use $\Phi^{-1}(\alpha)$.





\section{Poisson Distribution}
\label{PoissonDistribution}

\subsection{Definition}
The Poisson distribution is a discrete probability distribution that expresses the probability of a given number of events occurring in a fixed interval of time and/or space if these events occur with a known average rate and independently of the time since the last event.
The following functions return PMF and CDF of the Poisson distribution with mean $\mu \geq 0$.



\subsection{Density and CDF}

\begin{mpFunctionsExtract}
	\mpFunctionThree
	{PoissonDist? mpNumList? returns pdf, CDF and related information for the Poisson distribution}
	{x? mpNum? A real number}
	{lambda? mpNum? A real number greater 0, representing the degrees of freedom}
	{Output? String? A string describing the output choices}
\end{mpFunctionsExtract}


\vspace{0.3cm}
See section \ref{Functions returning pdf, CDF, and related information} for the options for {\itshape\sffamily Output}. Algorithms and formulas are given in sections \ref{ChiSquareDistributionDensity} and \ref{sec:ChiSquareDistribution_cdf}.



%\vspace{0.3cm}
%The following functions are provided for compatibility with established spreadsheet functions
%
%\vspace{0.3cm}
%\begin{mpFunctionsExtract}
%\mpWorksheetFunctionThree
%{POISSON? mpReal? the CDF and of the Poisson distribution}
%{x? mpReal? A real number. The numeric value at which to evaluate the distribution}
%{deg\_freedom? mpReal? An integer  greater 0, indicating the degrees of freedom}
%{Tails? Integer? Specifies the number of distribution tails to return. If tails = 1, TDIST returns the one-tailed distribution. If tails = 2, TDIST returns the two-tailed distribution.}
%\end{mpFunctionsExtract}
%
%\vspace{0.6cm}
%\begin{mpFunctionsExtract}
%\mpWorksheetFunctionThree
%{POISSON.DIST? mpReal? the CDF and of the Poisson distribution}
%{x? mpReal? A real number. The numeric value at which to evaluate the distribution}
%{deg\_freedom? mpReal? An integer  greater 0, indicating the degrees of freedom}
%{Cumulative ? Boolean? A logical value that determines the form of the function. If cumulative is TRUE, POISSON.DIST returns the cumulative distribution function; if FALSE, it returns the probability density function}
%\end{mpFunctionsExtract}



\subsubsection{Density}
\label{PoissonDistributionDensity}

\begin{equation} 
	f(k)= \frac{\mu^k}{k!} e^{-\mu} = sfcIgprefix(1+k,\mu)
\end{equation}


\subsubsection{CDF}
\begin{equation} 
	F(k)=  e^{-\mu} \sum_{i=0}^k \frac{\mu^i}{i!} = igammaq(1+k,\mu)
\end{equation}




\subsection{Quantiles}

\begin{mpFunctionsExtract}
	\mpFunctionThree
	{PoissonDistInv? mpNumList? quantiles and related information for the the Poisson distribution}
	{Prob? mpNum? A real number between 0 and 1.}
	{lambda? mpNum? A real number greater 0, representing the degrees of freedom}
	{Output? String? A string describing the output choices}
\end{mpFunctionsExtract}

See section \ref{Functions returning Quantiles} for the options for  {\itshape\sffamily Prob} and {\itshape\sffamily Output}). Algorithms and formulas are given in section \ref{PoissonDistributionQuantilesAlgorithm}.

\subsubsection{Quantiles: algorithms and formulas}
\label{PoissonDistributionQuantilesAlgorithm}
The algorithms follow the one for the chisquare distribution.

\subsection{Properties}
\label{PoissonDistributionProperties}


\begin{mpFunctionsExtract}
	\mpFunctionTwo
	{PoissonDistInfo? mpNumList? moments and related information for the Poisson distribution}
	{lambda? mpNum? A real number greater 0, representing the degrees of freedom}
	{Output? String? A string describing the output choices}
\end{mpFunctionsExtract}

\vspace{0.3cm}

See section \ref{Functions returning moments and related information} for the options for {\itshape\sffamily Output}. Algorithms and formulas are given in section \ref{ChiSquareDistributionProperties}.

\subsubsection{Moments and Cumulants}
The momemts and cumulants are given by
\begin{equation}
	\kappa_{r} = \lambda
\end{equation}
\begin{equation}
	\mu_{1} = \mu_{2} =\mu_{3} = \lambda
\end{equation}
\begin{equation}
	\mu_{4} = 3 \lambda^2 + \lambda
\end{equation}


\subsection{Random Numbers}

\begin{mpFunctionsExtract}
	\mpFunctionFour
	{PoissonDistRan? mpNumList? random numbers following a Poisson distribution}
	{Size? mpNum? A positive integer up to $10^7$}
	{lambda? mpNum? A real number greater 0, representing the degrees of freedom}
	{Generator? String? A string describing the random generator}
	{Output? String? A string describing the output choices}
\end{mpFunctionsExtract}


\vspace{0.3cm}
See section \ref{Functions returning Random numbers} for the options for  {\itshape\sffamily Size},  {\itshape\sffamily Generator} and {\itshape\sffamily Output}. Algorithms and formulas are given in section \ref{ChiSquareDistributionRandom}.







\section{Student's t-Distribution}
\subsection{Definition}
\label{tDistributionDefinition}

If $X$ is a random variable following a normal distribution with mean zero and variance unity and $\chi^2$ is a random variable following an independent $\chi^2$-distribution with $n$ degrees of freedom, 
then the distribution of the ratio $\frac{X}{\sqrt{\chi^2 / n}}$ is called Student's t-distribution with $n$ degrees of freedom


\subsection{Density and CDF}

\begin{mpFunctionsExtract}
	\mpFunctionThree
	{TDist? mpNumList? returns pdf, CDF and related information for the central $t$-distribution}
	{x? mpNum? A real number}
	{n? mpNum? A real number greater 0, representing the degrees of freedom}
	{Output? String? A string describing the output choices}
\end{mpFunctionsExtract}


\vspace{0.3cm}
See section \ref{Functions returning pdf, CDF, and related information} for the options for {\itshape\sffamily Output}. Algorithms and formulas are given in sections \ref{tDistributionDensity} and \ref{tDistributionCDF}.


%\vspace{0.3cm}
%The following functions are provided for compatibility with established spreadsheet functions
%
%\vspace{0.3cm}
%\begin{mpFunctionsExtract}
%\mpWorksheetFunctionThree
%{TDIST? mpReal? the CDF and of the central $t$-distribution}
%{x? mpReal? A real number. The numeric value at which to evaluate the distribution}
%{deg\_freedom? mpReal? An integer  greater 0, indicating the degrees of freedom}
%{Tails? Integer? Specifies the number of distribution tails to return. If tails = 1, TDIST returns the one-tailed distribution. If tails = 2, TDIST returns the two-tailed distribution.}
%\end{mpFunctionsExtract}
%
%\vspace{0.6cm}
%\begin{mpFunctionsExtract}
%\mpWorksheetFunctionThree
%{T.DIST? mpReal? the CDF and of the central $t$-distribution}
%{x? mpReal? A real number. The numeric value at which to evaluate the distribution}
%{deg\_freedom? mpReal? An integer  greater 0, indicating the degrees of freedom}
%{Cumulative ? Boolean? A logical value that determines the form of the function. If cumulative is TRUE, T.DIST returns the cumulative distribution function; if FALSE, it returns the probability density function}
%\end{mpFunctionsExtract}

%\vspace{0.6cm}
%\begin{mpFunctionsExtract}
%\mpWorksheetFunctionTwo
%{T.DIST.RT? mpReal? the complement of the CDF and of the central $t$-distribution}
%{x? mpReal? A real number}
%{deg\_freedom? mpReal? An integer  greater 0, indicating the degrees of freedom}
%\end{mpFunctionsExtract}
%
%\vspace{0.6cm}
%\begin{mpFunctionsExtract}
%\mpWorksheetFunctionTwo
%{T.DIST.2T? mpReal? the two-sided CDF of the central $t$-distribution}
%{x? mpReal? A real number}
%{deg\_freedom? mpReal? An integer  greater 0, indicating the degrees of freedom}
%\end{mpFunctionsExtract}



\subsubsection{Density}
\label{tDistributionDensity}
\nomenclature{$f_t(n, x)$}{pdf of the central $t$-distribution}
The density of a variable following a central Student's t-distribution with $n$ degrees of freedom is given by
\begin{equation}
	f_t(n,x) = \frac{\Gamma((n+1)/2)}{\sqrt{n\pi}\Gamma(n/2)} \left(\frac{n}{n+t^2}\right)^{(n+1)/2}
\end{equation}
where $\Gamma(\cdot)$ denotes the Gamma function (see section \ref{GammaFunction}.)



\subsubsection{CDF: General formulas}
\label{tDistributionCDF}
\nomenclature{$F_t(n, x)$}{CDF of the central $t$-distribution}
The cdf of a variable following a central  t-distribution with $n$ degrees of freedom is defined as

\begin{equation}
	\text{Pr}\left[X \le x\right] = F_t\left(n,x\right) =  \int_{0}^{x} f_t(n,t) dt
\end{equation}

The cdf of the central t-distribution is calculated for any positive degrees of freedom $n$ using the relationships
\begin{equation}
	2F_t\left(n,x\right) = F_F(1,n;x^2), \quad x \leq 0
\end{equation}
\begin{equation}
	F_t\left(n,x\right)-F_t\left(n,-x\right)  = F_F(1,n;x^2), \quad x \geq 0
\end{equation}
\begin{equation}
	F_t\left(n,x\right) = 1-F_t\left(n,-x\right) 
\end{equation}
where $F_F(1,n,x^2)$ denotes the cdf of the central $F$-distribution with 1 and $n$ of freedom (see section \ref{FDistributionCDF}).






\subsection{Quantiles}
\label{tDistributionQuantile}
\nomenclature{$t_{\nu,\alpha}$}{$\alpha$ quantile of the central $t$-distribution with $\nu$ degrees of freedom}

\begin{mpFunctionsExtract}
	\mpFunctionThree
	{TDistInv? mpNumList? returns quantiles and related information for the the central $t$-distribution}
	{Prob? mpNum? A real number between 0 and 1.}
	{n? mpNum? A real number greater 0, representing the degrees of freedom}
	{Output? String? A string describing the output choices}
\end{mpFunctionsExtract}

See section \ref{Functions returning Quantiles} for the options for  {\itshape\sffamily Prob} and {\itshape\sffamily Output}). 

%\vspace{0.3cm}
%
%The following functions are provided for compatibility with established spreadsheet functions
%
%\vspace{0.3cm}
%\begin{mpFunctionsExtract}
%\mpWorksheetFunctionTwo
%{TINV? mpReal? the two-tailed inverse of the central $t$-distribution}
%{x? mpReal? A real number}
%{deg\_freedom? mpReal? An integer  greater 0, indicating the degrees of freedom}
%\end{mpFunctionsExtract}
%
%\vspace{0.6cm}
%\begin{mpFunctionsExtract}
%\mpWorksheetFunctionTwo
%{T.INV? mpReal? the left-tailed inverse of the central $t$-distribution}
%{x? mpReal? A real number}
%{deg\_freedom? mpReal? An integer  greater 0, indicating the degrees of freedom}
%\end{mpFunctionsExtract}

%\vspace{0.6cm}
%\begin{mpFunctionsExtract}
%\mpWorksheetFunctionTwo
%{T.INV.2T? mpReal? the two-tailed inverse of the central $t$-distribution}
%{x? mpReal? A real number}
%{deg\_freedom? mpReal? An integer  greater 0, indicating the degrees of freedom}
%\end{mpFunctionsExtract}






\subsection{Properties}
\label{tDistributionProperties}

\begin{mpFunctionsExtract}
	\mpFunctionTwo
	{TDistInfo? mpNumList? returns moments and related information for the central $t$-distribution}
	{n? mpNum? A real number greater 0, representing the degrees of freedom}
	{Output? String? A string describing the output choices}
\end{mpFunctionsExtract}

\vspace{0.3cm}

See section \ref{Functions returning moments and related information} for the options for {\itshape\sffamily Output}. Algorithms and formulas are given in section \ref{tDistributionProperties}.



\subsubsection{Moments: algorithms and formulas}

The algebraic moments (defined for $n>r$) are given by
\begin{equation}
	\mu'_r = \left({\tfrac{1}{2}n}\right)^{r/2} \frac{\Gamma\left(\tfrac{1}{2}(n-r)\right)}{\Gamma\left(\tfrac{1}{2}n\right)}.
\end{equation}



\subsection{Random Numbers}
\label{tDistributionRandom}

\begin{mpFunctionsExtract}
	\mpFunctionFour
	{TDistRan? mpNumList? returns random numbers following a central $t$-distribution}
	{Size? mpNum? A positive integer up to $10^7$}
	{n? mpNum? A real number greater 0, representing the degrees of freedom}
	{Generator? String? A string describing the random generator}
	{Output? String? A string describing the output choices}
\end{mpFunctionsExtract}

\vspace{0.3cm}

See section \ref{Functions returning Random numbers} for the options for  {\itshape\sffamily Size},  {\itshape\sffamily Generator} and {\itshape\sffamily Output}. Algorithms and formulas are given in section \ref{tDistributionRandom}.


\subsubsection{Random Numbers: algorithms and formulas}

Following the definition we may define a random number $t$ from a $t$-distribution, using
random numbers from a normal and a chi-square distribution, as $t=\frac{z}{\sqrt{y_n/n}}$, 
where $z$ is a standard normal and $y_n$ a chi-squared variable with $n$ degrees of freedom. To obtain random numbers from these distributions see the appropriate sections.



\subsection{Behrens-Fisher Problem}

See \cite{Golhar_1972}





\section{Weibull Distribution}
\label{WeibullDistribution}

These functions return PDF, CDF, and ICDF of the Weibull distribution with shape parameter $a$ and scale
$b > 0$ and the support interval $(0,+\infty)$ :



\subsection{Density and CDF}

\begin{mpFunctionsExtract}
	\mpFunctionFour
	{WeibullDist? mpNumList? returns pdf, CDF and related information for the Weibull distribution}
	{x? mpNum? A real number}
	{a? mpNum? A real number greater 0, representing the numerator  degrees of freedom}
	{b? mpNum? A real number greater 0, representing the denominator degrees of freedom}
	{Output? String? A string describing the output choices}
\end{mpFunctionsExtract}


\vspace{0.3cm}
See section \ref{Functions returning pdf, CDF, and related information} for the options for {\itshape\sffamily Output}. Algorithms and formulas are given in sections \ref{BetaDistributionDensity} and \ref{BetaDistributionCDF}.


%\vspace{0.3cm}
%
%The following functions are provided for compatibility with established spreadsheet functions
%
%\vspace{0.3cm}
%\begin{mpFunctionsExtract}
%\mpWorksheetFunctionThree
%{WEIBULL? mpReal? the CDF and of the Weibull distribution}
%{x? mpReal? A real number. The numeric value at which to evaluate the distribution}
%{a? mpNum? A real number greater 0, representing the numerator  degrees of freedom}
%{b? mpNum? A real number greater 0, representing the denominator degrees of freedom}
%\end{mpFunctionsExtract}
%
%\vspace{0.6cm}
%\begin{mpFunctionsExtract}
%\mpWorksheetFunctionFour
%{WEIBULL.DIST? mpReal? the CDF and of the Weibull distribution}
%{x? mpReal? A real number. The numeric value at which to evaluate the distribution}
%{a? mpNum? A real number greater 0, representing the numerator  degrees of freedom}
%{b? mpNum? A real number greater 0, representing the denominator degrees of freedom}
%{Cumulative ? Boolean? A logical value that determines the form of the function. If cumulative is TRUE, T.DIST returns the cumulative distribution function; if FALSE, it returns the probability density function}
%\end{mpFunctionsExtract}



\subsubsection{Density}
\label{WeibullDistributionDensity}

\begin{equation} 
	f(x)= \frac{x}{b^2} \exp \left(- \frac{x^2}{2b^2}\right) \exp(-(x/b)^a)
\end{equation}


\subsubsection{CDF}

\begin{equation} 
	F(x)= 1 - \exp \left(- (x/b)^a\right)
	= -\text{expm1} \left(- (x/b)^a\right)
\end{equation}



\subsection{Quantiles}

\begin{mpFunctionsExtract}
	\mpFunctionFour
	{WeibullDistInv? mpNumList? returns quantiles and related information for the the central Beta-distribution}
	{Prob? mpNum? A real number between 0 and 1.}
	{a? mpNum? A real number greater 0, representing the numerator  degrees of freedom}
	{b? mpNum? A real number greater 0, representing the denominator degrees of freedom}
	{Output? String? A string describing the output choices}
\end{mpFunctionsExtract}

See section \ref{Functions returning Quantiles} for the options for  {\itshape\sffamily Prob} and {\itshape\sffamily Output}). 


\vspace{0.3cm}
\begin{equation} 
	F^{-1}(y)= b (- \text{ln1p}(-y))^{1/a}
\end{equation}



\subsection{Properties}


\begin{mpFunctionsExtract}
	\mpFunctionThree
	{WeibullDistInfo? mpNumList? returns moments and related information for the central Beta-distribution}
	{a? mpNum? A real number greater 0, representing the degrees of freedom}
	{b? mpNum? A real number greater 0, representing the degrees of freedom}
	{Output? String? A string describing the output choices}
\end{mpFunctionsExtract}

\vspace{0.3cm}

See section \ref{Functions returning moments and related information} for the options for {\itshape\sffamily Output}. Algorithms and formulas are given in section \ref{tDistributionProperties}.



\subsubsection{Moments: algorithms and formulas}

\begin{equation} 
	\mu_r^{'} = \sum_{j=0}^r \binom{r}{j} \Gamma\left(\frac{r-j}{c}+1 \right) b^{r-j}
\end{equation}

\begin{equation} 
	\mu_1 = b \Gamma\left(\frac{1}{c}+1 \right) 
\end{equation}

\begin{equation} 
	\mu_2 = b^2 \left[ \Gamma\left(\frac{1}{c}+1\right) \Gamma^2\left(\frac{1}{c}+1\right)  \right]
\end{equation}
See \cite{Rinne_book_2008} for further details.

\subsection{Random Numbers}

\begin{mpFunctionsExtract}
	\mpFunctionFive
	{WeibullDistRandom? mpNumList? returns random numbers following a central Beta-distribution}
	{Size? mpNum? A positive integer up to $10^7$}
	{a? mpNum? A real number greater 0, representing the numerator  degrees of freedom}
	{b? mpNum? A real number greater 0, representing the denominator degrees of freedom}
	{Generator? String? A string describing the random generator}
	{Output? String? A string describing the output choices}
\end{mpFunctionsExtract}

\vspace{0.3cm}

See section \ref{Functions returning Random numbers} for the options for  {\itshape\sffamily Size},  {\itshape\sffamily Generator} and {\itshape\sffamily Output}. Algorithms and formulas are given in section \ref{FDistributionRandom}.





\section{Bernoulli Distribution}

The Bernoulli distribution is a discrete distribution of the outcome of a single trial with only two results, 0 (failure) or 1 (success), with a probability of success $p$. The Bernoulli distribution is the simplest building block on which other discrete distributions of sequences of independent Bernoulli trials can be based. The Bernoulli is the binomial distribution ((k = 1, p)) with only one trial.



\subsection{Density and CDF}

\begin{mpFunctionsExtract}
	\mpFunctionThree
	{BernoulliDistBoost? mpNumList? returns pdf, CDF and related information for the central $t$-distribution}
	{k? mpNum? A real number, 0 or 1}
	{p? mpNum? A real number greater 0, representing the degrees of freedom}
	{Output? String? A string describing the output choices}
\end{mpFunctionsExtract}


\vspace{0.3cm}
See section \ref{Functions returning pdf, CDF, and related information} for the options for {\itshape\sffamily Output}. Algorithms and formulas are given in sections \ref{BernoulliDistributionDensity} and \ref{BernoulliDistributionCDF}.



\subsubsection{Density}
\label{BernoulliDistributionDensity}


\begin{equation}
	f(x)=\begin{cases}
		q=1-p & \text{for }k=0\\
		p & \text{for }k=1.
	\end{cases}
\end{equation}

\subsubsection{CDF}
\label{BernoulliDistributionCDF}
\begin{equation}
	F(x)=\begin{cases}
		0 & \text{for }k=0\\
		q & \text{for }k=0\\
		1 & \text{for }k=1.
	\end{cases}
\end{equation}




\subsection{Quantiles}

\begin{mpFunctionsExtract}
	\mpFunctionThree
	{BernoulliDistInvBoost? mpNumList? returns quantiles and related information for the the central $t$-distribution}
	{Prob? mpNum? A real number between 0 and 1.}
	{p? mpNum? A real number greater 0, representing the degrees of freedom}
	{Output? String? A string describing the output choices}
\end{mpFunctionsExtract}

See section \ref{Functions returning Quantiles} for the options for  {\itshape\sffamily Prob} and {\itshape\sffamily Output}). 
%Algorithms and formulas are given in section \ref{tDistributionQuantileAlgorithm}.

\subsubsection{Quantiles: Algorithm}
Using the relation: cdf = $1 - p$ for $k = 0$, else $1$.


\subsection{Properties}
\label{BernoulliDistributionProperties}


\begin{mpFunctionsExtract}
	\mpFunctionTwo
	{BernoulliDistInfoBoost? mpNumList? returns moments and related information for the central $t$-distribution}
	{p? mpNum? A real number greater 0, representing the degrees of freedom}
	{Output? String? A string describing the output choices}
\end{mpFunctionsExtract}

\vspace{0.3cm}

See section \ref{Functions returning moments and related information} for the options for {\itshape\sffamily Output}. Algorithms and formulas are given in section \ref{BernoulliDistributionProperties}.



\subsubsection{Moments: algorithms and formulas}

\begin{equation} 
	\mu_r^{'} = \sum_{i=0}^{r-1} \binom{r}{i}  (-1)^i p^{i+1} + (-p)^r
\end{equation}

\begin{equation} 
	\mu_1 = p
\end{equation}

\begin{equation} 
	\mu_2 = pq
\end{equation}

\begin{equation} 
	\mu_3 = pq(1-2p)
\end{equation}

\begin{equation} 
	\mu_4 = pq (1-3pq)
\end{equation}



\subsection{Random Numbers}
\begin{mpFunctionsExtract}
	\mpFunctionFour
	{BernoulliDistRandomBoost? mpNumList? returns random numbers following a central Binomial-distribution}
	{Size? mpNum? A positive integer up to $10^7$}
	{p? mpNum? The probability of a success on each trial.}
	{Generator? String? A string describing the random generator}
	{Output? String? A string describing the output choices}
\end{mpFunctionsExtract}

\vspace{0.3cm}

See section \ref{Functions returning Random numbers} for the options for  {\itshape\sffamily Size},  {\itshape\sffamily Generator} and {\itshape\sffamily Output}. Algorithms and formulas are given below.




\section{Cauchy Distribution}
\label{CauchyDistribution}



\subsection{Density and CDF}

\begin{mpFunctionsExtract}
	\mpFunctionFour
	{CauchyDistBoost? mpNumList? returns pdf, CDF and related information for the Cauchy distribution}
	{x? mpNum? A real number}
	{a? mpNum? A real number greater 0, representing the numerator  degrees of freedom}
	{b? mpNum? A real number greater 0, representing the denominator degrees of freedom}
	{Output? String? A string describing the output choices}
\end{mpFunctionsExtract}


\vspace{0.3cm}
See section \ref{Functions returning pdf, CDF, and related information} for the options for {\itshape\sffamily Output}. 
%Algorithms and formulas are given in sections \ref{BetaDistributionDensity} and \ref{BetaDistributionCDF}.


\subsubsection{Density}
\label{CauchyDistributionDensity}

\begin{equation} 
	f(x)=\frac{1}{\pi(1+((x-a)/b)^2)}
\end{equation}


\subsubsection{CDF}
\begin{equation} 
	F(x)=\frac{1}{2} + \frac{1}{\pi} \arctan \left(\frac{x-a}{b} \right)
\end{equation}



\subsection{Quantiles}
\begin{mpFunctionsExtract}
	\mpFunctionFour
	{CauchyDistInvBoost? mpNumList? returns quantiles and related information for the Cauchy distribution}
	{Prob? mpNum? A real number between 0 and 1.}
	{a? mpNum? A real number greater 0, representing the numerator  degrees of freedom}
	{b? mpNum? A real number greater 0, representing the denominator degrees of freedom}
	{Output? String? A string describing the output choices}
\end{mpFunctionsExtract}

See section \ref{Functions returning Quantiles} for the options for  {\itshape\sffamily Prob} and {\itshape\sffamily Output}). Algorithms and formulas are given below.

\begin{equation}
	F^{-1}(y)=\begin{cases}
		a-b/\tan(\pi y), & y<0.5,\\
		a, &  y=0.5,\\
		a-b/\tan(\pi (1-y)) & y>0.5.
	\end{cases}
\end{equation}




\subsection{Properties}
\label{CauchyDistributionProperties}


\begin{mpFunctionsExtract}
	\mpFunctionThree
	{CauchyDistInfoBoost? mpNumList? returns moments and related information for the Cauchy distribution}
	{a? mpNum? A real number greater 0, representing the degrees of freedom}
	{b? mpNum? A real number greater 0, representing the degrees of freedom}
	{Output? String? A string describing the output choices}
\end{mpFunctionsExtract}

\vspace{0.3cm}

See section \ref{Functions returning moments and related information} for the options for {\itshape\sffamily Output}. 

All the usual non-member accessor functions that are generic to all distributions are supported: Cumulative Distribution Function,
Probability Density Function, Quantile, Hazard Function, Cumulative Hazard Function, mean, median, mode, variance, standard
deviation, skewness, kurtosis, kurtosis\_excess, range and support.
Note however that the Cauchy distribution does not have a mean, standard deviation, etc. See mathematically undefined function  to
control whether these should fail to compile with a BOOST\_STATIC\_ASSERTION\_FAILURE, which is the default.
Alternately, the functions mean, standard deviation, variance, skewness, kurtosis and kurtosis\_excess will all return a domain\_error
if called.



\subsection{Random Numbers}
\begin{mpFunctionsExtract}
	\mpFunctionFive
	{CauchyDistRandomBoost? mpNumList? returns random numbers following a Cauchy distribution}
	{Size? mpNum? A positive integer up to $10^7$}
	{a? mpNum? A real number greater 0, representing the numerator  degrees of freedom}
	{b? mpNum? A real number greater 0, representing the denominator degrees of freedom}
	{Generator? String? A string describing the random generator}
	{Output? String? A string describing the output choices}
\end{mpFunctionsExtract}

\vspace{0.3cm}

See section \ref{Functions returning Random numbers} for the options for  {\itshape\sffamily Size},  {\itshape\sffamily Generator} and {\itshape\sffamily Output}. Algorithms and formulas are given in below.





\section{Extreme Value (or Gumbel) Distribution}

These functions return PDF, CDF, and ICDF of the Extreme Value Type I distribution
with location $a$, scale $b > 0$, and the support interval $(-\infty,+\infty)$ :



\subsection{Density and CDF}

\begin{mpFunctionsExtract}
	\mpFunctionFour
	{ExtremevalueDistBoost? mpNumList? returns pdf, CDF and related information for the Extreme Value distribution}
	{x? mpNum? A real number}
	{a? mpNum? A real number greater 0, representing the numerator  degrees of freedom}
	{b? mpNum? A real number greater 0, representing the denominator degrees of freedom}
	{Output? String? A string describing the output choices}
\end{mpFunctionsExtract}


\vspace{0.3cm}
See section \ref{Functions returning pdf, CDF, and related information} for the options for {\itshape\sffamily Output}. Algorithms and formulas are given in sections \ref{ExtremevalueDistributionDensity} and \ref{ExtremevalueDistributionCDF}.


\subsubsection{Density}
\label{ExtremevalueDistributionDensity}

\begin{equation} 
	f(x)=\frac{e^{-(x-a)/b}}{b} e^{e^{-(x-a)/b}}
\end{equation}


\subsubsection{CDF}
\label{ExtremevalueDistributionCDF}
\begin{equation} 
	F(x)= e^{e^{-(x-a)/b}}
\end{equation}



\subsection{Quantiles}
\begin{mpFunctionsExtract}
	\mpFunctionFour
	{ExtremevalueDistInvBoost? mpNumList? returns quantiles and related information for the the Extreme Value distribution}
	{Prob? mpNum? A real number between 0 and 1.}
	{a? mpNum? A real number greater 0, representing the numerator  degrees of freedom}
	{b? mpNum? A real number greater 0, representing the denominator degrees of freedom}
	{Output? String? A string describing the output choices}
\end{mpFunctionsExtract}

See section \ref{Functions returning Quantiles} for the options for  {\itshape\sffamily Prob} and {\itshape\sffamily Output}). Algorithms and formulas are given below.

\begin{equation} 
	F^{-1}(y)= a-\ln(-\ln(y))
\end{equation}


\subsection{Properties}
\label{ExtremevalueDistributionProperties}


\begin{mpFunctionsExtract}
	\mpFunctionThree
	{ExtremevalueDistInfoBoost? mpNumList? returns moments and related information for the Extreme Value distribution}
	{a? mpNum? A real number greater 0, representing the degrees of freedom}
	{b? mpNum? A real number greater 0, representing the degrees of freedom}
	{Output? String? A string describing the output choices}
\end{mpFunctionsExtract}

\vspace{0.3cm}

See section \ref{Functions returning moments and related information} for the options for {\itshape\sffamily Output}. 




\subsection{Random Numbers}

\begin{mpFunctionsExtract}
	\mpFunctionFive
	{ExtremevalueDistRandomBoost? mpNumList? returns random numbers following a Extreme Value distribution}
	{Size? mpNum? A positive integer up to $10^7$}
	{a? mpNum? A real number greater 0, representing the numerator  degrees of freedom}
	{b? mpNum? A real number greater 0, representing the denominator degrees of freedom}
	{Generator? String? A string describing the random generator}
	{Output? String? A string describing the output choices}
\end{mpFunctionsExtract}

\vspace{0.3cm}

See section \ref{Functions returning Random numbers} for the options for  {\itshape\sffamily Size},  {\itshape\sffamily Generator} and {\itshape\sffamily Output}. Algorithms and formulas are given in below.





\section{Geometric Distribution}

Geometric distribution: it is used when there are exactly two mutually exclusive outcomes of a Bernoulli trial: these outcomes are labelled "success" and "failure". For Bernoulli trials each with success fraction $p$, the geometric distribution gives the probability of observing $k$ trials (failures, events,
occurrences, or arrivals) before the first success.




\subsection{Density and CDF}

\begin{mpFunctionsExtract}
	\mpFunctionThree
	{GeometricDistBoost? mpNumList? returns pdf, CDF and related information for the Geometric distribution}
	{k? mpNum? A real number}
	{p? mpNum? A real number greater 0, representing the numerator  degrees of freedom}
	{Output? String? A string describing the output choices}
\end{mpFunctionsExtract}


\vspace{0.3cm}
See section \ref{Functions returning pdf, CDF, and related information} for the options for {\itshape\sffamily Output}. Algorithms and formulas are given in sections \ref{GeometricDistributionDensity} and \ref{GeometricDistributionCDF}.


\subsubsection{Density}
\label{GeometricDistributionDensity}

\begin{equation} 
	f(k;p)= p (1-p)^k 
\end{equation}


\subsubsection{CDF}
\label{GeometricDistributionCDF}
\begin{equation} 
	F(k;p)= 1- (1-p)^{k+1} 
\end{equation}

\subsection{Quantiles}

\begin{mpFunctionsExtract}
	\mpFunctionThree
	{GeometricDistInvBoost? mpNumList? returns quantiles and related information for the Geometric distribution}
	{Prob? mpNum? A real number between 0 and 1.}
	{p? mpNum? A real number greater 0, representing the numerator  degrees of freedom}
	{Output? String? A string describing the output choices}
\end{mpFunctionsExtract}

See section \ref{Functions returning Quantiles} for the options for  {\itshape\sffamily Prob} and {\itshape\sffamily Output}). Algorithms and formulas are given below.

\begin{equation} 
	F^{-1}(x;p)= \frac{\text{log1p}(-x)}{\text{log1p}(-p)} -1
\end{equation}



\subsection{Properties}
\label{GeometricDistributionProperties}


\begin{mpFunctionsExtract}
	\mpFunctionTwo
	{GeometricDistInfoBoost? mpNumList? returns moments and related information for the Geometric distribution}
	{p? mpNum? A real number greater 0, representing the degrees of freedom}
	{Output? String? A string describing the output choices}
\end{mpFunctionsExtract}

\vspace{0.3cm}

See section \ref{Functions returning moments and related information} for the options for {\itshape\sffamily Output}. 




\subsection{Random Numbers}
\begin{mpFunctionsExtract}
	\mpFunctionFour
	{GeometricDistRandomBoost? mpNumList? returns random numbers following a Geometric distribution}
	{Size? mpNum? A positive integer up to $10^7$}
	{p? mpNum? A real number greater 0, representing the denominator degrees of freedom}
	{Generator? String? A string describing the random generator}
	{Output? String? A string describing the output choices}
\end{mpFunctionsExtract}

\vspace{0.3cm}

See section \ref{Functions returning Random numbers} for the options for  {\itshape\sffamily Size},  {\itshape\sffamily Generator} and {\itshape\sffamily Output}. Algorithms and formulas are given in below.





\section{Inverse Chi Squared Distribution}


\subsection{Definition}
\label{InverseChiSquaredDistributionDefinition}

The inverse-chi-squared distribution (or inverted-chi-square distribution[1] ) is the probability distribution of a random variable whose multiplicative inverse (reciprocal) has a chi-squared distribution. It is also often defined as the distribution of a random variable whose reciprocal divided by its degrees of freedom is a chi-squared distribution. That is, if $X$ has the chi-squared distribution with $\nu$  degrees of freedom, then according to the first definition, $1/X$ has the inverse-chi-squared distribution with  $\nu$  degrees of freedom; while according to the second definition, $\nu/X$ has the inverse-chi-squared distribution with  $\nu$  degrees of freedom. 

The inverse-chi-squared distribution is a special case of a inverse-gamma distribution with $\nu$ (degrees of freedom), shape ($\alpha$) and scale ($\beta$), where $\alpha=\nu/2$ and $\beta=1/2$.



\subsection{Density and CDF}

\begin{mpFunctionsExtract}
	\mpFunctionThree
	{InverseChiSquaredDistBoost? mpNumList? returns pdf, CDF and related information for the inverse-chi-squared -distribution}
	{x? mpNum? A real number}
	{n? mpNum? A real number greater 0, representing the degrees of freedom}
	{Output? String? A string describing the output choices}
\end{mpFunctionsExtract}


\vspace{0.3cm}
See section \ref{Functions returning pdf, CDF, and related information} for the options for {\itshape\sffamily Output}. 
%Algorithms and formulas are given in sections \ref{InverseChiSquaredDistributionDensity} and \ref{sec:ChiSquareDistribution_cdf}.


\subsubsection{Density}
\label{InverseChiSquaredDistributionDensity}

The first definition yields a probability density function given by
\begin{equation} 
	f(x;\nu)= \frac{2^{-\nu/2}}{\Gamma(\nu /2)} x^{-\nu/2-1} e^{-1/(2x)}
\end{equation}
while the second definition yields the density function
\begin{equation} 
	f(x;\nu)= \frac{(\nu/2)^{\nu/2}}{\Gamma(\nu /2)} x^{-\nu/2-1} e^{-\nu/(2x)}
\end{equation}
In both cases, $x>0$  and  $\nu$  is the degrees of freedom parameter. Further, $\Gamma$ is the gamma function. Both definitions are special cases of the scaled-inverse-chi-squared distribution. For the first definition the variance of the distribution is $\sigma=1/\nu$, while for the second definition $\sigma=1$ .



\subsubsection{CDF}
\begin{equation} 
	F(x;\nu)= \frac{1}{\Gamma(\nu /2)} \Gamma\left(\frac{\nu}{2},\frac{1}{2x}\right) 
\end{equation}



\subsection{Quantiles}


\begin{mpFunctionsExtract}
	\mpFunctionThree
	{InverseChiSquaredDistInvBoost? mpNumList? quantiles and related information for the inverse-chi-squared distribution}
	{Prob? mpNum? A real number between 0 and 1.}
	{n? mpNum? A real number greater 0, representing the degrees of freedom}
	{Output? String? A string describing the output choices}
\end{mpFunctionsExtract}

See section \ref{Functions returning Quantiles} for the options for  {\itshape\sffamily Prob} and {\itshape\sffamily Output}). 
%Algorithms and formulas are given in section \ref{ChiSquareDistributionQuantilesAlgorithm}.

\begin{equation} 
	F^{-1}(prob;\nu)= \beta / / gamma-q-inv(\alpha, p)
\end{equation}


\subsection{Properties}
\label{InverseChiSquaredDistributionProperties}


\begin{mpFunctionsExtract}
	\mpFunctionTwo
	{InverseChiSquaredDistInfoBoost? mpNumList? moments and related information for the inverse-chi-squared distribution}
	{n? mpNum? A real number greater 0, representing the degrees of freedom}
	{Output? String? A string describing the output choices}
\end{mpFunctionsExtract}

\vspace{0.3cm}

See section \ref{Functions returning moments and related information} for the options for {\itshape\sffamily Output}. Algorithms and formulas are given below.

\subsubsection{Moments and Cumulants}
\begin{equation} 
	\mu_1 = \frac{\nu}{\nu-2} \text{for } \nu>2.
\end{equation}



\subsection{Random Numbers}

\begin{mpFunctionsExtract}
	\mpFunctionFour
	{InverseChiSquaredDistRanBoost? mpNumList? random numbers following a inverse-chi-squared distribution}
	{Size? mpNum? A positive integer up to $10^7$}
	{n? mpNum? A real number greater 0, representing the degrees of freedom}
	{Generator? String? A string describing the random generator}
	{Output? String? A string describing the output choices}
\end{mpFunctionsExtract}


\vspace{0.3cm}
See section \ref{Functions returning Random numbers} for the options for  {\itshape\sffamily Size},  {\itshape\sffamily Generator} and {\itshape\sffamily Output}. Algorithms and formulas are given below.





\section{Inverse Gamma Distribution}


\subsection{Definition}
\label{InverseGammaDistributionDefinition}

In probability theory and statistics, the inverse gamma distribution is a two-parameter family of continuous probability distributions on the positive real line, which is the distribution of the reciprocal of a variable distributed according to the gamma distribution. Perhaps the chief use of the inverse gamma distribution is in Bayesian statistics, where the distribution arises as the marginal posterior distribution for the unknown variance of a normal distribution if an uninformative prior is used; and as an analytically tractable conjugate prior if an informative prior is required.

However, it is common among Bayesians to consider an alternative parametrization of the normal distribution in terms of the precision, defined as the reciprocal of the variance, which allows the gamma distribution to be used directly as a conjugate prior. Other Bayesians prefer to parametrize the inverse gamma distribution differently, as a scaled inverse chi-squared distribution



\subsection{Density and CDF}

\begin{mpFunctionsExtract}
	\mpFunctionFour
	{InverseGammaDistBoost? mpNumList? returns pdf, CDF and related information for the inverse gamma distribution}
	{x? mpNum? A real number}
	{a? mpNum? A real number greater 0, representing the numerator  degrees of freedom}
	{b? mpNum? A real number greater 0, representing the denominator degrees of freedom}
	{Output? String? A string describing the output choices}
\end{mpFunctionsExtract}


\vspace{0.3cm}
See section \ref{Functions returning pdf, CDF, and related information} for the options for {\itshape\sffamily Output}. Algorithms and formulas are given in sections \ref{InverseGammaDistributionDensity} and \ref{InverseGammaDistributionCDF}.


\subsubsection{Density}
\label{InverseGammaDistributionDensity}

The inverse gamma distribution's probability density function is defined over the support $x>0$
\begin{equation} 
	f(x;\alpha,\beta)= \frac{\beta^{\alpha}}{\Gamma(\alpha)} x^{-\alpha-1} \exp \left(-\frac{\beta}{x} \right)
\end{equation}


with shape parameter $\alpha$ and scale parameter $\beta$.



\subsubsection{CDF}
\label{InverseGammaDistributionCDF}
The cumulative distribution function is the regularized gamma function
\begin{equation} 
	F(x;\alpha,\beta)= \frac{\Gamma(\alpha),\beta/x}{\Gamma(\alpha)} = Q \left(\alpha, -\frac{\beta}{x} \right)
\end{equation}


where the numerator is the upper incomplete gamma function and the denominator is the gamma function. Many math packages allow you to compute $Q$, the regularized gamma function, directly.




\subsection{Quantiles}

\begin{mpFunctionsExtract}
	\mpFunctionFour
	{InverseGammaDistInvBoost? mpNumList? returns quantiles and related information for the the inverse gamma distribution}
	{Prob? mpNum? A real number between 0 and 1.}
	{m? mpNum? A real number greater 0, representing the numerator  degrees of freedom}
	{n? mpNum? A real number greater 0, representing the denominator degrees of freedom}
	{Output? String? A string describing the output choices}
\end{mpFunctionsExtract}

See section \ref{Functions returning Quantiles} for the options for  {\itshape\sffamily Prob} and {\itshape\sffamily Output}). Algorithms and formulas are given below.

\begin{equation} 
	F^{-1}(prob;\nu)= \beta / / gamma-q-inv(\alpha, p)
\end{equation}


\subsection{Properties}
\label{InverseGammaDistributionProperties}


\begin{mpFunctionsExtract}
	\mpFunctionThree
	{InverseGammaDistInfoBoost? mpNumList? returns moments and related information for the inverse gamma distribution}
	{a? mpNum? A real number greater 0, representing the degrees of freedom}
	{b? mpNum? A real number greater 0, representing the degrees of freedom}
	{Output? String? A string describing the output choices}
\end{mpFunctionsExtract}

\vspace{0.3cm}

See section \ref{Functions returning moments and related information} for the options for {\itshape\sffamily Output}. Algorithms and formulas are given below.

\subsubsection{Moments and Cumulants}
\begin{equation} 
	\mu_1 = \frac{\nu}{\nu-2} \text{for } \nu>2.
\end{equation}




\subsection{Random Numbers}

\begin{mpFunctionsExtract}
	\mpFunctionFive
	{InverseGammaDistRanBoost? mpNumList? returns random numbers following a inverse gamma distribution}
	{Size? mpNum? A positive integer up to $10^7$}
	{a? mpNum? A real number greater 0, representing the numerator  degrees of freedom}
	{b? mpNum? A real number greater 0, representing the denominator degrees of freedom}
	{Generator? String? A string describing the random generator}
	{Output? String? A string describing the output choices}
\end{mpFunctionsExtract}

\vspace{0.3cm}

See section \ref{Functions returning Random numbers} for the options for  {\itshape\sffamily Size},  {\itshape\sffamily Generator} and {\itshape\sffamily Output}. Algorithms and formulas are given below.





\section{Inverse Gaussian (or Wald) Distribution}


\subsection{Definition}
\label{InverseGaussianDistributionDefinition}

In probability theory, the inverse Gaussian distribution (also known as the Wald distribution) is a two-parameter family of continuous probability distributions  with mean $\mu$ and shape parameter $\lambda$ and support on $(0,\infty)$.
As $\lambda$ tends to infinity, the inverse Gaussian distribution becomes more like a normal (Gaussian) distribution. The inverse Gaussian distribution has several properties analogous to a Gaussian distribution. The name can be misleading: it is an "inverse" only in that its cumulant generating function (logarithm of the characteristic function) is the inverse of the cumulant generating function of a Gaussian random variable.

While the Gaussian describes a Brownian Motion's level at a fixed time, the inverse Gaussian describes the distribution of the time a Brownian Motion with positive drift takes to reach a fixed positive level.

See also \href{http://en.wikipedia.org/wiki/Inverse_Gaussian_distribution}{http://en.wikipedia.org/wiki/Inverse\_Gaussian\_distribution}. 


\subsection{Density and CDF}

\begin{mpFunctionsExtract}
	\mpFunctionFour
	{InverseGaussianDistBoost? mpNumList? returns pdf, CDF and related information for the inverse Gaussian distribution}
	{x? mpNum? A real number}
	{mu? mpNum? A real number greater 0, representing the numerator  degrees of freedom}
	{lambda? mpNum? A real number greater 0, representing the denominator degrees of freedom}
	{Output? String? A string describing the output choices}
\end{mpFunctionsExtract}


\vspace{0.3cm}
See section \ref{Functions returning pdf, CDF, and related information} for the options for {\itshape\sffamily Output}. Algorithms and formulas are given in sections \ref{InverseGaussianDistributionDensity} and \ref{InverseGaussianDistributionCDF}.



\subsubsection{Density}
\label{InverseGaussianDistributionDensity}

\begin{equation} 
	f(x;\mu,\lambda)= \sqrt{\frac{\lambda}{2\pi x^3}} \exp \left( \frac{-\lambda(x-\mu)^2}{2\mu^2 x} \right)
\end{equation}


\subsubsection{CDF}
\label{InverseGaussianDistributionCDF}

\begin{equation} 
	F(x;\mu,\lambda)= \Phi\left(\sqrt{\frac{\lambda}{x}} \left(\frac{x}{\mu}-1\right)\right) +\exp \left( \frac{2\lambda}{\mu} \right) \Phi\left(-\sqrt{\frac{\lambda}{x}} \left(\frac{x}{\mu}+1\right)\right)
\end{equation}



\subsection{Quantiles}

\begin{mpFunctionsExtract}
	\mpFunctionFour
	{InverseGaussianDistInvBoost? mpNumList? returns quantiles and related information for the the inverse Gaussian distribution}
	{Prob? mpNum? A real number between 0 and 1.}
	{mu? mpNum? A real number greater 0, representing the numerator  degrees of freedom}
	{lambda? mpNum? A real number greater 0, representing the denominator degrees of freedom}
	{Output? String? A string describing the output choices}
\end{mpFunctionsExtract}

See section \ref{Functions returning Quantiles} for the options for  {\itshape\sffamily Prob} and {\itshape\sffamily Output}). Algorithms and formulas are given below.

\begin{equation} 
	F^{-1}(prob;\nu)= \beta / / gamma-q-inv(\alpha, p)
\end{equation}



\subsection{Properties}
\label{InverseGaussianDistributionProperties}

\begin{mpFunctionsExtract}
	\mpFunctionThree
	{InverseGaussianDistInfoBoost? mpNumList? returns moments and related information for the inverse Gaussian distribution}
	{mu? mpNum? A real number greater 0, representing the degrees of freedom}
	{lambda? mpNum? A real number greater 0, representing the degrees of freedom}
	{Output? String? A string describing the output choices}
\end{mpFunctionsExtract}

\vspace{0.3cm}

See section \ref{Functions returning moments and related information} for the options for {\itshape\sffamily Output}. Algorithms and formulas are given below.

\subsubsection{Moments and Cumulants}
\begin{equation} 
	\mu_1 = \frac{\nu}{\nu-2} \text{for } \nu>2.
\end{equation}





\subsection{Random Numbers}

\begin{mpFunctionsExtract}
	\mpFunctionFive
	{InverseGaussianDistRanBoost? mpNumList? returns random numbers following a inverse Gaussian distribution}
	{Size? mpNum? A positive integer up to $10^7$}
	{mu? mpNum? A real number greater 0, representing the numerator  degrees of freedom}
	{lambda? mpNum? A real number greater 0, representing the denominator degrees of freedom}
	{Generator? String? A string describing the random generator}
	{Output? String? A string describing the output choices}
\end{mpFunctionsExtract}

\vspace{0.3cm}

See section \ref{Functions returning Random numbers} for the options for  {\itshape\sffamily Size},  {\itshape\sffamily Generator} and {\itshape\sffamily Output}. Algorithms and formulas are given below.





\section{Laplace Distribution}
These functions return PDF, CDF, and ICDF of the Laplace distribution with location
$a$, scale $b > 0$, and the support interval $(-\infty,+\infty)$ :


\subsection{Density and CDF}
\begin{mpFunctionsExtract}
	\mpFunctionFour
	{LaplaceDistBoost? mpNumList? returns pdf, CDF and related information for the Laplace distribution}
	{x? mpNum? A real number}
	{a? mpNum? A real number greater 0, representing the numerator  degrees of freedom}
	{b? mpNum? A real number greater 0, representing the denominator degrees of freedom}
	{Output? String? A string describing the output choices}
\end{mpFunctionsExtract}


\vspace{0.3cm}
See section \ref{Functions returning pdf, CDF, and related information} for the options for {\itshape\sffamily Output}. Algorithms and formulas are given in sections \ref{LaplaceDistributionDensity} and \ref{LaplaceDistributionCDF}.


\subsubsection{Density}
\label{LaplaceDistributionDensity}

\begin{equation} 
	f(x)= \exp(- \vert x-a \vert /b)/(2b)
\end{equation}


\subsubsection{CDF}
\label{LaplaceDistributionCDF}

\begin{equation}
	F(x)=\begin{cases}
		\frac{1}{2} - \frac{1}{2} \text{expm1}\left(- \frac{x-a}{b}\right) & x \geq a\\
		\frac{1}{2} \text{exp}\left(- \frac{x-a}{b}\right) & x<a.
	\end{cases}
\end{equation}



\subsection{Quantiles}
\begin{mpFunctionsExtract}
	\mpFunctionFour
	{LaplaceDistInvBoost? mpNumList? returns quantiles and related information for the the Laplace distribution}
	{Prob? mpNum? A real number between 0 and 1.}
	{a? mpNum? A real number greater 0, representing the numerator  degrees of freedom}
	{b? mpNum? A real number greater 0, representing the denominator degrees of freedom}
	{Output? String? A string describing the output choices}
\end{mpFunctionsExtract}

See section \ref{Functions returning Quantiles} for the options for  {\itshape\sffamily Prob} and {\itshape\sffamily Output}). Algorithms and formulas are given below.

\vspace{0.3cm}
\begin{equation}
	F^{-1}(y)=\begin{cases}
		a+b \: \text{ln}(2y), & y<0.5,\\
		a-b \: \text{ln}(2(1-y))  & y>0.5.
	\end{cases}
\end{equation}



\subsection{Properties}
\label{LaplaceDistributionProperties}

\begin{mpFunctionsExtract}
	\mpFunctionThree
	{LaplaceDistInfoBoost? mpNumList? returns moments and related information for the Laplace distribution}
	{a? mpNum? A real number greater 0, representing the degrees of freedom}
	{b? mpNum? A real number greater 0, representing the degrees of freedom}
	{Output? String? A string describing the output choices}
\end{mpFunctionsExtract}

\vspace{0.3cm}

See section \ref{Functions returning moments and related information} for the options for {\itshape\sffamily Output}. 



\subsection{Random Numbers}

\begin{mpFunctionsExtract}
	\mpFunctionFive
	{LaplaceDistRanBoost? mpNumList? returns random numbers following a Laplace distribution}
	{Size? mpNum? A positive integer up to $10^7$}
	{a? mpNum? A real number greater 0, representing the numerator  degrees of freedom}
	{b? mpNum? A real number greater 0, representing the denominator degrees of freedom}
	{Generator? String? A string describing the random generator}
	{Output? String? A string describing the output choices}
\end{mpFunctionsExtract}

\vspace{0.3cm}

See section \ref{Functions returning Random numbers} for the options for  {\itshape\sffamily Size},  {\itshape\sffamily Generator} and {\itshape\sffamily Output}. Algorithms and formulas are given in below.





\section{Logistic Distribution}

\subsection{Definition}
These functions return PDF, CDF, and ICDF of the logistic distribution with location
$a$, scale $b > 0$, and the support interval $(-\infty,+\infty)$ :

\subsection{Density and CDF}
\begin{mpFunctionsExtract}
	\mpFunctionFour
	{LogisticDistBoost? mpNumList? returns pdf, CDF and related information for the Logistic distribution}
	{x? mpNum? A real number}
	{a? mpNum? A real number greater 0, representing the numerator  degrees of freedom}
	{b? mpNum? A real number greater 0, representing the denominator degrees of freedom}
	{Output? String? A string describing the output choices}
\end{mpFunctionsExtract}

\vspace{0.3cm}
See section \ref{Functions returning pdf, CDF, and related information} for the options for {\itshape\sffamily Output}. Algorithms and formulas are given in sections \ref{LogisticDistributionDensity} and \ref{LogisticDistributionCDF}.


\subsubsection{Density}
\label{LogisticDistributionDensity}

\begin{equation} 
	f(x)= \frac{1}{b} \frac{\exp \left(-\frac{x-a}{b}\right)}{\left(1+\exp \left(-\frac{x-a}{b}\right)\right)^2}
\end{equation}


\subsubsection{CDF}
\label{LogisticDistributionCDF}

\begin{equation} 
	F(x)= \frac{1}{1+\exp \left(-\frac{x-a}{b}\right)}
\end{equation}


\subsection{Quantiles}
\begin{mpFunctionsExtract}
	\mpFunctionFour
	{LogisticDistInvBoost? mpNumList? returns quantiles and related information for the the Logistic distribution}
	{Prob? mpNum? A real number between 0 and 1.}
	{a? mpNum? A real number greater 0, representing the numerator  degrees of freedom}
	{b? mpNum? A real number greater 0, representing the denominator degrees of freedom}
	{Output? String? A string describing the output choices}
\end{mpFunctionsExtract}

See section \ref{Functions returning Quantiles} for the options for  {\itshape\sffamily Prob} and {\itshape\sffamily Output}). Algorithms and formulas are given below.

\begin{equation} 
	F^{-1}(y)= a-b \: \text{ln}\left((1-y)/y\right)
\end{equation}



\subsection{Properties}
\label{LogisticDistributionProperties}

\begin{mpFunctionsExtract}
	\mpFunctionThree
	{LogisticDistInfoBoost? mpNumList? returns moments and related information for the Logistic distribution}
	{a? mpNum? A real number greater 0, representing the degrees of freedom}
	{b? mpNum? A real number greater 0, representing the degrees of freedom}
	{Output? String? A string describing the output choices}
\end{mpFunctionsExtract}

\vspace{0.3cm}

See section \ref{Functions returning moments and related information} for the options for {\itshape\sffamily Output}. 



\subsection{Random Numbers}

\begin{mpFunctionsExtract}
	\mpFunctionFive
	{LogisticDistRanBoost? mpNumList? returns random numbers following a Logistic distribution}
	{Size? mpNum? A positive integer up to $10^7$}
	{a? mpNum? A real number greater 0, representing the numerator  degrees of freedom}
	{b? mpNum? A real number greater 0, representing the denominator degrees of freedom}
	{Generator? String? A string describing the random generator}
	{Output? String? A string describing the output choices}
\end{mpFunctionsExtract}

\vspace{0.3cm}

See section \ref{Functions returning Random numbers} for the options for  {\itshape\sffamily Size},  {\itshape\sffamily Generator} and {\itshape\sffamily Output}. Algorithms and formulas are given in below.






\section{Pareto Distribution}

\subsection{Definition}
These functions return PDF, CDF, and ICDF of the Pareto distribution with minimum
(real) value $k > 0$, shape $a > 0$, and the support interval $(k,+\infty)$ :
This is a reference: \cite{wiki_Pareto}


\subsection{Density and CDF}
\begin{mpFunctionsExtract}
	\mpFunctionFour
	{ParetoDistBoost? mpNumList? returns pdf, CDF and related information for the Pareto distribution}
	{x? mpNum? A real number}
	{a? mpNum? A real number greater 0, representing the numerator  degrees of freedom}
	{b? mpNum? A real number greater 0, representing the denominator degrees of freedom}
	{Output? String? A string describing the output choices}
\end{mpFunctionsExtract}

\vspace{0.3cm}
See section \ref{Functions returning pdf, CDF, and related information} for the options for {\itshape\sffamily Output}. Algorithms and formulas are given in sections \ref{ParetoDistributionDensity} and \ref{ParetoDistributionCDF}.

\subsubsection{Density}
\label{ParetoDistributionDensity}

\begin{equation} 
	f(x)= \frac{a}{x} \left(\frac{k}{x}\right)^a
\end{equation}


\subsubsection{CDF}
\label{ParetoDistributionCDF}

\vspace{0.3cm}
\begin{equation} 
	F(x)= 1 - \left(\frac{k}{x}\right)^a = - \text{powm1}(k/x,a)
\end{equation}



\subsection{Quantiles}
\begin{mpFunctionsExtract}
	\mpFunctionFour
	{ParetoDistInvBoost? mpNumList? returns quantiles and related information for the the Pareto distribution}
	{Prob? mpNum? A real number between 0 and 1.}
	{a? mpNum? A real number greater 0, representing the numerator  degrees of freedom}
	{b? mpNum? A real number greater 0, representing the denominator degrees of freedom}
	{Output? String? A string describing the output choices}
\end{mpFunctionsExtract}

See section \ref{Functions returning Quantiles} for the options for  {\itshape\sffamily Prob} and {\itshape\sffamily Output}). Algorithms and formulas are given below.

\begin{equation} 
	F^{-1}(y)= k(1-y)^{-1/a}
\end{equation}



\subsection{Properties}
\label{ParetoDistributionProperties}

\begin{mpFunctionsExtract}
	\mpFunctionThree
	{ParetoDistInfoBoost? mpNumList? returns moments and related information for the Pareto distribution}
	{a? mpNum? A real number greater 0, representing the degrees of freedom}
	{b? mpNum? A real number greater 0, representing the degrees of freedom}
	{Output? String? A string describing the output choices}
\end{mpFunctionsExtract}

\vspace{0.3cm}

See section \ref{Functions returning moments and related information} for the options for {\itshape\sffamily Output}. 



\subsection{Random Numbers}

\begin{mpFunctionsExtract}
	\mpFunctionFive
	{ParetoDistRanBoost? mpNumList? returns random numbers following a Pareto distribution}
	{Size? mpNum? A positive integer up to $10^7$}
	{a? mpNum? A real number greater 0, representing the numerator  degrees of freedom}
	{b? mpNum? A real number greater 0, representing the denominator degrees of freedom}
	{Generator? String? A string describing the random generator}
	{Output? String? A string describing the output choices}
\end{mpFunctionsExtract}

\vspace{0.3cm}

See section \ref{Functions returning Random numbers} for the options for  {\itshape\sffamily Size},  {\itshape\sffamily Generator} and {\itshape\sffamily Output}. Algorithms and formulas are given in below.





\section{Raleigh Distribution}
\label{RaleighDistribution}

\subsection{Definition}
These functions return PDF, CDF, and ICDF of the Rayleigh distribution with scale
$b > 0$ and the support interval $(0,+\infty)$ :


\subsection{Density and CDF}

\begin{mpFunctionsExtract}
	\mpFunctionThree
	{RaleighDistBoost? mpNumList? returns pdf, CDF and related information for the Raleigh distribution}
	{x? mpNum? A real number}
	{n? mpNum? A real number greater 0, representing the degrees of freedom}
	{Output? String? A string describing the output choices}
\end{mpFunctionsExtract}


\vspace{0.3cm}
See section \ref{Functions returning pdf, CDF, and related information} for the options for {\itshape\sffamily Output}. Algorithms and formulas are given in sections \ref{RaleighDistributionDensity} and \ref{RaleighDistributionCDF}.


\subsection{Density}
\label{RaleighDistributionDensity}

\begin{equation} 
	f(x)= \frac{x}{b^2} \exp \left(- \frac{x^2}{2b^2}\right)
\end{equation}


\subsection{CDF}
\label{RaleighDistributionCDF}

\begin{equation} 
	F(x)= 1 - \exp \left(- \frac{x^2}{2b^2}\right)
	= -\text{expm1} \left(- \frac{x^2}{2b^2}\right)
\end{equation}



\subsection{Quantiles}

\begin{mpFunctionsExtract}
	\mpFunctionThree
	{RaleighDistInvBoost? mpNumList? quantiles and related information for the Raleigh distribution}
	{Prob? mpNum? A real number between 0 and 1.}
	{n? mpNum? A real number greater 0, representing the degrees of freedom}
	{Output? String? A string describing the output choices}
\end{mpFunctionsExtract}

See section \ref{Functions returning Quantiles} for the options for  {\itshape\sffamily Prob} and {\itshape\sffamily Output}). Algorithms and formulas are below.

\begin{equation} 
	F^{-1}(y)= b \sqrt{-2 \cdot \text{ln1p}(-y)}
\end{equation}



\subsection{Properties}
\label{RaleighDistributionProperties}


\begin{mpFunctionsExtract}
	\mpFunctionTwo
	{RaleighDistInfoBoost? mpNumList? moments and related information for the Raleigh distribution}
	{n? mpNum? A real number greater 0, representing the degrees of freedom}
	{Output? String? A string describing the output choices}
\end{mpFunctionsExtract}

\vspace{0.3cm}

See section \ref{Functions returning moments and related information} for the options for {\itshape\sffamily Output}. Algorithms and formulas are given below.

\subsubsection{Moments and Cumulants}
\begin{equation} 
	\mu_1 = s \sqrt{\pi/2}
\end{equation}



\subsection{Random Numbers}

\begin{mpFunctionsExtract}
	\mpFunctionFour
	{RaleighDistRanBoost? mpNumList? random numbers following a Raleigh distribution}
	{Size? mpNum? A positive integer up to $10^7$}
	{n? mpNum? A real number greater 0, representing the degrees of freedom}
	{Generator? String? A string describing the random generator}
	{Output? String? A string describing the output choices}
\end{mpFunctionsExtract}


\vspace{0.3cm}
See section \ref{Functions returning Random numbers} for the options for  {\itshape\sffamily Size},  {\itshape\sffamily Generator} and {\itshape\sffamily Output}. Algorithms and formulas are given below.





\section{Triangular Distribution}

\subsection{Definition}
These functions return PDF, CDF, and ICDF of the triangular distribution on the
support interval $[a, b]$ with finite $a < b$ and mode $c, a \leq c \leq b$.

\subsection{Density and CDF}

\begin{mpFunctionsExtract}
	\mpFunctionFive
	{TriangularDistBoost? mpNumList? returns pdf, CDF and related information for the triangular distribution}
	{x? mpNum? A real number.}
	{a? mpNum? The left border parameter.}
	{b? mpNum? The right border parameter.}
	{c? mpNum? The mode parameter.}
	{Output? String? A string describing the output choices}
\end{mpFunctionsExtract}


\vspace{0.3cm}
See section \ref{Functions returning pdf, CDF, and related information} for the options for {\itshape\sffamily Output}. Algorithms and formulas are given in sections \ref{TriangularDistributionDensity} and \ref{TriangularDistributionCDF}.



\subsubsection{Density}
\label{TriangularDistributionDensity}

\begin{equation}
	f(x)=\begin{cases}
		0  & x<a\\
		\frac{2(x-a)}{(b-a)(c-a)} & a \leq x < c\\
		\frac{2}{b-a} & x = c\\
		\frac{2(b-x)}{(b-a)(b-c)} & c < x \leq b\\
		0  & x>b
	\end{cases}
\end{equation}


\subsubsection{CDF}
\label{TriangularDistributionCDF}

\begin{equation}
	F(x)=\begin{cases}
		0  & x<a\\
		\frac{(x-a)^2}{(b-a)(c-a)} & a \leq x < c\\
		\frac{c-a}{b-a} & x = c\\
		1-\frac{(b-x)^2}{(b-a)(b-c)} & c < x \leq b\\
		1  & x>b
	\end{cases}
\end{equation}



\subsection{Quantiles}

\begin{mpFunctionsExtract}
	\mpFunctionFive
	{TriangularDistInvBoost? mpNumList? returns quantiles and related information for the the triangular distribution}
	{Prob? mpNum? A real number between 0 and 1.}
	{a? mpNum? The left border parameter.}
	{b? mpNum? The right border parameter.}
	{c? mpNum? The mode parameter.}
	{Output? String? A string describing the output choices}
\end{mpFunctionsExtract}

\vspace{0.3cm}
See section \ref{Functions returning Quantiles} for the options for  {\itshape\sffamily Prob} and {\itshape\sffamily Output}). 

\vspace{0.3cm}
\begin{equation}
	F^{-1}(y)=\begin{cases}
		a+\sqrt{(b-a)(c-a)y} & y<t\\
		c & y=t \\
		b-\sqrt{(b-a)(b-c)(1-y)} & y>t
	\end{cases}
\end{equation}
where $t=(c-a)/(b-a)$.


\subsection{Properties}
\label{TriangularDistributionProperties}


\begin{mpFunctionsExtract}
	\mpFunctionFour
	{TriangularDistInfoBoost? mpNumList? returns moments and related information for the triangular distribution}
	{a? mpNum? The left border parameter.}
	{b? mpNum? The right border parameter.}
	{c? mpNum? The mode parameter.}
	{Output? String? A string describing the output choices}
\end{mpFunctionsExtract}

\vspace{0.3cm}

See section \ref{Functions returning moments and related information} for the options for {\itshape\sffamily Output}. Algorithms and formulas are given below.

\subsubsection{Moments}

\begin{equation} 
	\mu_1 = \frac{a+b+c}{3}
\end{equation}

\begin{equation} 
	\mu_2 = \frac{a^2+b^2+c^2-ab-ac-bc}{18}
\end{equation}

\begin{equation} 
	\gamma_1 = \frac{\sqrt{2}(a+b-2c)(2a-b-c)(a-2b+c)}{5(a^2+b^2+c^2-ab-ac-bc)^{3/2}}
\end{equation}

\begin{equation} 
	\gamma_2 = -\frac{3}{5}
\end{equation}




\subsection{Random Numbers}

\begin{mpFunctionsExtract}
	\mpFunctionSix
	{TriangularDistRanBoost? mpNumList? returns random numbers following a triangular distribution}
	{Size? mpNum? A positive integer up to $10^7$}
	{a? mpNum? The left border parameter.}
	{b? mpNum? The right border parameter.}
	{c? mpNum? The mode parameter.}
	{Generator? String? A string describing the random generator}
	{Output? String? A string describing the output choices}
\end{mpFunctionsExtract}

\vspace{0.3cm}

See section \ref{Functions returning Random numbers} for the options for  {\itshape\sffamily Size},  {\itshape\sffamily Generator} and {\itshape\sffamily Output}. Algorithms and formulas are given below.





\section{Uniform Distribution}

\subsection{Definition}
These functions return PDF, CDF, and ICDF of the uniform distribution on the support
interval $[a, b]$ with finite $a < b$:



\subsection{Density and CDF}
\begin{mpFunctionsExtract}
	\mpFunctionFour
	{UniformDistBoost? mpNumList? returns pdf, CDF and related information for the uniform distribution}
	{x? mpNum? A real number}
	{a? mpNum? The left border parameter.}
	{b? mpNum? The right border parameter.}
	{Output? String? A string describing the output choices}
\end{mpFunctionsExtract}

\vspace{0.3cm}
See section \ref{Functions returning pdf, CDF, and related information} for the options for {\itshape\sffamily Output}. Algorithms and formulas are given in sections \ref{UniformDistributionDensity} and \ref{UniformDistributionCDF}.

\subsubsection{Density}
\label{UniformDistributionDensity}

\begin{equation} 
	f(x)=\frac{1}{b-a}
\end{equation}


\subsubsection{CDF}
\label{UniformDistributionCDF}

\begin{equation} 
	F(x)=\frac{x-a}{b-a}
\end{equation}



\subsection{Quantiles}

\begin{mpFunctionsExtract}
	\mpFunctionFour
	{UniformDistInvBoost? mpNumList? returns quantiles and related information for the the uniform distribution}
	{Prob? mpNum? A real number between 0 and 1.}
	{a? mpNum? The left border parameter.}
	{b? mpNum? The right border parameter.}
	{Output? String? A string describing the output choices}
\end{mpFunctionsExtract}


\vspace{0.3cm}
See section \ref{Functions returning Quantiles} for the options for  {\itshape\sffamily Prob} and {\itshape\sffamily Output}). Algorithms and formulas are given below.

\begin{equation} 
	F^{-1}(y)= a+y(b-a)
\end{equation}



\subsection{Properties}
\label{UniformDistributionProperties}

\begin{mpFunctionsExtract}
	\mpFunctionThree
	{UniformDistInfoBoost? mpNumList? returns moments and related information for the uniform distribution}
	{a? mpNum? A real number greater 0, representing the degrees of freedom}
	{b? mpNum? A real number greater 0, representing the degrees of freedom}
	{Output? String? A string describing the output choices}
\end{mpFunctionsExtract}

\vspace{0.3cm}

See section \ref{Functions returning moments and related information} for the options for {\itshape\sffamily Output}. 


\subsubsection{Moments}

\begin{equation} 
	\mu_1 = \frac{a+b}{2}
\end{equation}

\begin{equation} 
	\mu_2 = \frac{(b-a)^2}{12}
\end{equation}

\begin{equation} 
	\gamma_1 = 0
\end{equation}

\begin{equation} 
	\gamma_2 = -\frac{6}{5}
\end{equation}



\subsection{Random Numbers}

\begin{mpFunctionsExtract}
	\mpFunctionFive
	{UniformDistRanBoost? mpNumList? returns random numbers following a uniform distribution}
	{Size? mpNum? A positive integer up to $10^7$}
	{a? mpNum? A real number greater 0, representing the degrees of freedom}
	{b? mpNum? A real number greater 0, representing the degrees of freedom}
	{Generator? String? A string describing the random generator}
	{Output? String? A string describing the output choices}
\end{mpFunctionsExtract}

\vspace{0.3cm}

See section \ref{Functions returning Random numbers} for the options for  {\itshape\sffamily Size},  {\itshape\sffamily Generator} and {\itshape\sffamily Output}. Algorithms and formulas are given in below.











\chapter{Noncentral Distribution Functions (based on Boost)}
\label{BoostDistributionFunctions} 


\section{Noncentral Beta-Distribution}
\label{NoncentralBetaDistribution}


\subsection{Definition}
\label{NoncentralBetaDistributionDefinition}

If $X_1$ an $X_2$ are independent random variables, $X_1$  following a non-central $\chi^2$-distribution with noncentrality parameter $\lambda$ and $2a$ degrees of freedom, and  $X_2$  following a $\chi^2$-distribution with $2b$ degrees of freedom, then the distribution of the ratio $B=\frac{X_1}{X_1+X_2}$  is said to follow a non-central Beta-distribution with  $a$ and $b$  degrees of freedom.

Note: The univariate version of the noncentral distribution of Wilks Lambda: GLM 
%(see section \ref{WilksLambdaDistributionDistributionDefinition_GLM}) 
is equivalent to $W=1-B$

See \cite{Tiwari_1997}



\subsection{Density and CDF}

\begin{mpFunctionsExtract}
	\mpFunctionFive
	{NoncentralBetaDistBoost? mpNumList? returns pdf, CDF and related information for the central Beta-distribution}
	{x? mpNum? A real number}
	{m? mpNum? A real number greater 0, representing the numerator  degrees of freedom}
	{n? mpNum? A real number greater 0, representing the denominator degrees of freedom}
	{lambda? mpNum? A real number greater 0, representing the noncentrality parameter}
	{Output? String? A string describing the output choices}
\end{mpFunctionsExtract}



\subsubsection{Density}
\label{NoncentralBetaDistributionDensity}
\nomenclature{$f_{\text{Beta'}}(x;a,b,\lambda)$}{pdf of the (singly) noncentral Beta-distribution}
The density function of the noncentral beta-Distribution is given by \citep{Wang1993}: this needs to be checked, see Paolella 2007:
\begin{equation}
	f_{\text{Beta'}}(x;n_1,n_2,\lambda) = e^{-\lambda/2} f_{B}(x;n_1,n_2) {}_1F_1 \left(\tfrac{1}{2}(m+n), \tfrac{1}{2}n, \tfrac{n x \lambda}{2(m+n x)}\right)
\end{equation}
%where $f_{B}(x;m,n)$ is the pdf of the (central) beta-distribution (see section \ref{BetaDistributionDensity} and ${}_1F_1(\cdot)$ is the confluent hypergeometric function (see section \ref{Hypergeometric1F1}).


%\subsection{CDF: General }
\label{NoncentralBetaDistributionCDF}
\nomenclature{$F_{\text{Beta'}}(x;a,b,\lambda)$}{CDF of the (singly) noncentral Beta-distribution}
%\begin{tabular}{p{481pt}}
%\toprule
%\textsf{Function \textbf{NoncentralBetaDist}($\boldsymbol{a}\ As\ mpNum$, $\boldsymbol{b}\ As\ mpNum$) As mpNum}\index{Multiprecision Functions!NoncentralBetaDist} \\
%\bottomrule
%\end{tabular}
%
%\vspace{0.3cm}

\subsubsection{CDF: Infinite Series}
The cdf can be calculated using the following infinite series \cite{Benton_2003}:
\begin{equation}
	\text{Pr}[F \leq x] = F_{B'}(x;a,b,\lambda) = e^{-\lambda/2} \sum_{j=0}^{\infty}{\frac{(\lambda/2)^j}{j!}I(a+j,b,x)}
\end{equation}



\subsection{Quantiles}

\begin{mpFunctionsExtract}
	\mpFunctionFive
	{NoncentralBetaDistInvBoost? mpNumList? returns quantiles and related information for the the noncentral Beta-distribution}
	{Prob? mpNum? A real number between 0 and 1.}
	{m? mpNum? A real number greater 0, representing the numerator  degrees of freedom}
	{n? mpNum? A real number greater 0, representing the denominator degrees of freedom}
	{lambda? mpNum? A real number greater 0, representing the noncentrality parameter}
	{Output? String? A string describing the output choices}
\end{mpFunctionsExtract}

See section \ref{Functions returning Quantiles} for the options for  {\itshape\sffamily Prob} and {\itshape\sffamily Output}). Algorithms and formulas are given below.




\subsection{Properties}
\label{NoncentralBetaDistributionProperties}


\begin{mpFunctionsExtract}
	\mpFunctionFour
	{NoncentralBetaDistInfoBoost? mpNumList? returns moments and related information for the noncentral Beta-distribution}
	{m? mpNum? A real number greater 0, representing the numerator  degrees of freedom}
	{n? mpNum? A real number greater 0, representing the denominator degrees of freedom}
	{lambda? mpNum? A real number greater 0, representing the noncentrality parameter}
	{Output? String? A string describing the output choices}
\end{mpFunctionsExtract}

\vspace{0.3cm}

See section \ref{Functions returning moments and related information} for the options for {\itshape\sffamily Output}. Algorithms and formulas are given below.

\subsubsection{Moments of the non-central Beta-distribution}
Currently, Boost does not calculate the moments of the noncentral beta distribution. 

%See \ref{Special case: moments of the non-central Beta-distribution} for a discussion.


\subsection{Random Numbers}

\begin{mpFunctionsExtract}
	\mpFunctionSix
	{NoncentralBetaDistRanBoost? mpNumList? returns random numbers following a noncentral Beta-distribution}
	{Size? mpNum? A positive integer up to $10^7$}
	{m? mpNum? A real number greater 0, representing the numerator  degrees of freedom}
	{n? mpNum? A real number greater 0, representing the denominator degrees of freedom}
	{lambda? mpNum? A real number greater 0, representing the noncentrality parameter}
	{Generator? String? A string describing the random generator}
	{Output? String? A string describing the output choices}
\end{mpFunctionsExtract}

\vspace{0.3cm}
See section \ref{Functions returning Random numbers} for the options for  {\itshape\sffamily Size},  {\itshape\sffamily Generator} and {\itshape\sffamily Output}. Algorithms and formulas are given below.

\vspace{0.3cm}
Random numbers from a non-central Beta-distribution with integer or half-integer p− and q−values is easily obtained using the definition above i.e. by using a random number from a non-central chi-square distribution and another from a (central) chi-square distribution.



\section{Noncentral Chi-Square Distribution}
\label{NoncentralChiSquareDistribution}


\subsection{Definition}
\label{NoncentralChiSquareDistributionDefinitionBoost}

Let $X_1, X_2, \ldots, X_n$ be independent and identically distributed random variables each following a normal distribution with mean $\mu_j$ and unit variance. 
Then $\chi^2 = \sum_{j=1}^n X_j$ is said to follow a noncentral $\chi^2$-distribution with $n$ degress of freedom and noncentrality parameter \mbox{$\lambda = \sum_{j=1}^n (\mu_j - \mu)$.} 


\subsection{Density and CDF}

\begin{mpFunctionsExtract}
	\mpFunctionFour
	{NoncentralCDistBoost? mpNumList? returns pdf, CDF and related information for the noncentral $\chi^2$-distribution}
	{x? mpNum? A real number}
	{n? mpNum? A real number greater 0, representing the degrees of freedom}
	{lambda? mpNum? A real number greater 0, representing the noncentrality parameter}
	{Output? String? A string describing the output choices}
\end{mpFunctionsExtract}


\vspace{0.3cm}
See section \ref{Functions returning pdf, CDF, and related information} for the options for {\itshape\sffamily Output}. 
%Algorithms and formulas are given in sections \ref{ChiSquareDistributionDensity} and \ref{sec:ChiSquareDistribution_cdf}.




\label{NoncentralChiSquareDistributionDensityBoost}
\nomenclature{$f_{\chi^2}(n, x; \lambda)$}{pdf of the noncentral chi-square distribution}



\subsubsection{CDF: General formulas}
\label{NoncentralChiSquareDistributionCDFBoost}
\nomenclature{$F_{\chi^2}(n, x; \lambda)$}{CDF of the noncentral chi-square distribution}

The cdf of a noncentral chi-square variable with $n$ degrees of freedom and $\lambda$ is given by
\begin{equation}
	\text{Pr}\left[\chi^2 \le x\right] = F_{\chi^2}\left(n, x; \lambda\right) =  \int_{0}^{x} f_{\chi^2}\left(n, t; \lambda\right) dt
\end{equation} 

\subsubsection{CDF: Infinite series in terms of the central cdf}
The cdf of a noncentral chi-square variable with $n$ degrees of freedom and $\lambda$ is given by
\begin{equation}
	F_{\chi^2}\left(n, x; \lambda\right) = e^{-\lambda/2} \sum_{j=0}^\infty {\frac{(\lambda /2)^j}{j!} F_{\chi^2}\left(n+2+j, x\right) }
\end{equation} 
% where $F_{\chi^2}(n, \cdot)$ is the cdf of the (central) $\chi^2$ distribution (see section \ref{sec:ChiSquareDistribution_cdf}).

\subsubsection{CDF: Infinite series in terms of the central pdf}
\cite{ding_1992} gives the following representation (this is used by Boost for small lambda):
\begin{equation}
	F_{\chi^2}\left(n, x; \lambda\right) = 2e^{-\lambda/2} \sum_{i=0}^\infty f_{\chi^2}\left(n+2+2i, x\right) \left(\sum_{k=0}^i{\frac{(\lambda /2)^k}{k!}}\right)
\end{equation} 
% where $f_{\chi^2}(n, \cdot)$ is the pdf of the (central) $\chi^2$ distribution (see section \ref{ChiSquareDistributionDensity}).





\subsection{Quantiles}
\label{NoncentralChiSquareDistributionQuantilesBoost}

\begin{mpFunctionsExtract}
	\mpFunctionFour
	{NoncentralCDistInvBoost? mpNumList? quantiles and related information for the noncentral $\chi^2$-distribution}
	{Prob? mpNum? A real number between 0 and 1.}
	{n? mpNum? A real number greater 0, representing the degrees of freedom}
	{lambda? mpNum? A real number greater 0, representing the noncentrality parameter}
	{Output? String? A string describing the output choices}
\end{mpFunctionsExtract}

\vspace{0.3cm}
See section \ref{Functions returning Quantiles} for the options for  {\itshape\sffamily Prob} and {\itshape\sffamily Output}). Algorithms and formulas are given below


\vspace{0.3cm}
The quantile is approximated as
\begin{equation}
	\chi^2_{n,\lambda,\alpha}  \thickapprox  (1+b) \chi^2_{n_1,\lambda,\alpha} , \quad \text{where } n_1= \frac{(n+\lambda)^2}{n+2\lambda} , \quad  b = \frac{\lambda}{n+\lambda}
\end{equation}



\subsection{Properties}
\label{NoncentralChiSquareDistributionPropertiesBoost}


\begin{mpFunctionsExtract}
	\mpFunctionThree
	{NoncentralCDistInfoBoost? mpNumList? moments and related information for the noncentral $\chi^2$-distribution}
	{n? mpNum? A real number greater 0, representing the degrees of freedom}
	{lambda? mpNum? A real number greater 0, representing the noncentrality parameter}
	{Output? String? A string describing the output choices}
\end{mpFunctionsExtract}

\vspace{0.3cm}

See section \ref{Functions returning moments and related information} for the options for {\itshape\sffamily Output}. Algorithms and formulas are given below.



\subsubsection{Moments and Cumulants}
The cumulants of noncentral $\chi^2$are given by
\begin{equation}
	\kappa_{r}(n, \lambda) = 2^{r-1} (r-1)! (n+r\lambda)
\end{equation}





\subsection{Random Numbers}
\label{NoncentralChiSquareDistributionRandomBoost}


\begin{mpFunctionsExtract}
	\mpFunctionFive
	{NoncentralCDistRanBoost? mpNumList? random numbers following a noncentral $\chi^2$-distribution}
	{Size? mpNum? A positive integer up to $10^7$}
	{n? mpNum? A real number greater 0, representing the degrees of freedom}
	{lambda? mpNum? A real number greater 0, representing the noncentrality parameter}
	{Generator? String? A string describing the random generator}
	{Output? String? A string describing the output choices}
\end{mpFunctionsExtract}


\vspace{0.3cm}
See section \ref{Functions returning Random numbers} for the options for  {\itshape\sffamily Size},  {\itshape\sffamily Generator} and {\itshape\sffamily Output}. Algorithms and formulas are given below


Random numbers from a non-central chi-square distribution is easily obtained using the
definition above by e.g.

\begin{enumerate}
	\item Put $\mu = \sqrt{\lambda/n}$
	\item Sum $n$ random numbers from a normal distribution with mean $\mu$ and variance unity.
	Note that this is not a unique choice. The only requirement is that $\lambda = \sum \mu_i^2$.
	\item Return the sum as a random number from a non-central chi-square distribution with
	$n$ degrees of freedom and non-central parameter $\lambda$.
\end{enumerate}

This ought to be sufficient for most applications but if needed more efficient techniques
may easily be developed e.g. using more general techniques.





\section{NonCentral F-Distribution}
\label{NonCentralFDistributionBoost}

\subsection{Definition}
\label{NonCentralFDistributionDefinitionBoost}

If $X_1$ an $X_2$ are independent random variables, $X_1$  following a non-central $\chi^2$-distribution with noncentrality parameter $\lambda$ and $m$ degrees of freedom, and  
$X_2$  following a $\chi^2$-distribution with $m$ degrees of freedom, then the distribution of the ratio $F=\frac{X_1/m}{X_2/n}$ is said to follow a non-central
F-distribution with  noncentrality parameter $\lambda$ and  $m$ and $n$  degrees of freedom.


\subsection{Density and CDF}

\begin{mpFunctionsExtract}
	\mpFunctionFive
	{NoncentralFDistBoost? mpNumList? returns pdf, CDF and related information for the noncentral $F$-distribution}
	{x? mpNum? A real number}
	{m? mpNum? A real number greater 0, representing the numerator  degrees of freedom}
	{n? mpNum? A real number greater 0, representing the denominator degrees of freedom}
	{lambda? mpNum? A real number greater 0, representing the noncentrality parameter}
	{Output? String? A string describing the output choices}
\end{mpFunctionsExtract}



\subsubsection{Density}
\label{NonCentralFDistributionDensityBoost}
\nomenclature{$f_{F'}(x;m,n)$}{pdf of the (singly) noncentral $F$-distribution}




%\subsection{CDF}
\label{NoncentralFDistributionCDFBoost}
\nomenclature{$f_{F'}(x;m,n)$}{CDF of the (singly) noncentral $F$-distribution}



\subsubsection{CDF (singly noncentral: Infinite Series}
The cdf of a variable following a (singly) noncentral F-distribution with $n$ and $m$ degrees of freedom and noncentrality parameter $\lambda_1$ and is given by
\begin{equation}
	\text{Pr}[F \leq x] = F_{F'}(x;m,n,\lambda) = e^{-\lambda} \sum_{j=0}^{\infty}{\frac{(\lambda/2)^j}{j!}F(m+2j,n,x)}
\end{equation}
% where $F_{F}(\cdot)$ denotes the cdf of the central F-distribution (see section \ref{FDistributionCDF}).





\subsection{Quantiles}


\begin{mpFunctionsExtract}
	\mpFunctionFive
	{NoncentralFDistInvBoost? mpNumList? returns quantiles and related information for the the noncentral $F$-distribution}
	{Prob? mpNum? A real number between 0 and 1.}
	{m? mpNum? A real number greater 0, representing the numerator  degrees of freedom}
	{n? mpNum? A real number greater 0, representing the denominator degrees of freedom}
	{lambda? mpNum? A real number greater 0, representing the noncentrality parameter}
	{Output? String? A string describing the output choices}
\end{mpFunctionsExtract}

See section \ref{Functions returning Quantiles} for the options for  {\itshape\sffamily Prob} and {\itshape\sffamily Output}). Algorithms and formulas are given below.




\subsection{Properties}


\begin{mpFunctionsExtract}
	\mpFunctionFour
	{NoncentralFDistInfoBoost? mpNumList? returns moments and related information for the noncentral $F$-distribution}
	{m? mpNum? A real number greater 0, representing the numerator  degrees of freedom}
	{n? mpNum? A real number greater 0, representing the denominator degrees of freedom}
	{lambda? mpNum? A real number greater 0, representing the noncentrality parameter}
	{Output? String? A string describing the output choices}
\end{mpFunctionsExtract}

\vspace{0.3cm}

See section \ref{Functions returning moments and related information} for the options for {\itshape\sffamily Output}. Algorithms and formulas are given below.


\subsubsection{Moments (singly noncentral}
The algebraic moments (defined for $f_2 > 2r$) are given by
\begin{equation}
	\mu'_r = \frac{\Gamma( \tfrac{1}{2}f_1+r)-\Gamma( \tfrac{1}{2}f_2-r)}{\Gamma( \tfrac{1}{2}f_2)} \sum_{j=0}^{r} { \binom{r}{j} \frac{\tfrac{1}{2}\lambda f_1)^j} {\Gamma( \tfrac{1}{2}f_1+j)} }, \quad \text{for } f_2 > 2r.
\end{equation}
The first four raw moments (defined for $n > 2k$) are given by 
\begin{equation*}
	\mu'_1 = \frac{n}{m} \frac{m+\lambda}{n-2}
\end{equation*}
\begin{equation*}
	\mu'_2 = \left(\frac{n}{m}\right)^2 \frac{\lambda^2+(2\lambda+m)(m+2)}{(n-2)(n-4)}
\end{equation*}
\begin{equation*}
	\mu'_3 = \left(\frac{n}{m}\right)^3 \frac{\lambda^3+3(m+4)\lambda^2+(3\lambda+m)(m+4)(m+2)}{(n-2)(n-4)(n-6)}
\end{equation*}
\begin{equation*}
	\mu'_4 = \left(\frac{n}{m}\right)^4 \frac{\lambda^3+4(m+6)\lambda^3+6(m+6)(m+4)\lambda^2+(4\lambda+m)(m+6)(m+4)(m+2)}{(n-2)(n-4)(n-6)(n-8)}
\end{equation*}




\subsubsection{Relationships to other distributions (singly noncentral)}
\begin{equation}
	F_{F'}(x;m,n,\lambda) = F_B\left(\tfrac{1}{2} n, \tfrac{1}{2} m,\frac{mx}{mx+n};\lambda\right)
\end{equation}

\subsection{Random Numbers}
\label{NonCentralFDistributionRandom}


\begin{mpFunctionsExtract}
	\mpFunctionSix
	{NoncentralFDistRanBoost? mpNumList? returns random numbers following a noncentral $F$-distribution}
	{Size? mpNum? A positive integer up to $10^7$}
	{m? mpNum? A real number greater 0, representing the numerator  degrees of freedom}
	{n? mpNum? A real number greater 0, representing the denominator degrees of freedom}
	{lambda? mpNum? A real number greater 0, representing the noncentrality parameter}
	{Generator? String? A string describing the random generator}
	{Output? String? A string describing the output choices}
\end{mpFunctionsExtract}

\vspace{0.3cm}
See section \ref{Functions returning Random numbers} for the options for  {\itshape\sffamily Size},  {\itshape\sffamily Generator} and {\itshape\sffamily Output}. Algorithms and formulas are given below.

Random numbers from a non-central F-distribution is easily obtained using the definition in terms of the ratio between two independent random numbers from central and non-central chi-square distributions. This ought to be sufficient for most applications but if needed more efficient techniques may easily be developed e.g. using more general techniques.





\section{Noncentral Student's t-Distribution}
\label{NoncentraltDistributionBoost}


\subsection{Definition}
\label{NoncentraltDistributionDefinitionBoost}


If $X$ is a random variable following a normal distribution with mean $\delta$ and variance unity and $\chi^2$ is a random variable following an independent $\chi°2$-distribution with $n$ degrees of freedom, 
then the distribution of the ratio $\frac{X}{\sqrt{\chi^2 / n}}$ is called noncentral t-distribution with $n$ degrees of freedom and noncentrality parameter $\delta$.



\subsection{Density and CDF}

\begin{mpFunctionsExtract}
	\mpFunctionFour
	{NoncentralTDistBoost? mpNumList? returns pdf, CDF and related information for the noncentral $t$-distribution}
	{x? mpNum? A real number}
	{n? mpNum? A real number greater 0, representing the degrees of freedom}
	{delta? mpNum? A real number greater 0, representing the noncentrality parameter}
	{Output? String? A string describing the output choices}
\end{mpFunctionsExtract}


\vspace{0.3cm}
See section \ref{Functions returning pdf, CDF, and related information} for the options for {\itshape\sffamily Output}. Algorithms and formulas are given in sections \ref{NoncentraltDistributionDensityBoost} and \ref{NoncentraltDistributionCDFBoost}.



%\subsection{Density}
\label{NoncentraltDistributionDensityBoost}
\nomenclature{$f_{t'}\left(n,x, \delta\right)$}{pdf of the (singly) noncentral t-distribution}



\subsubsection{Density (singly noncentral): Infinite series}
The pdf of a variable following a noncentral  t-distribution with $n$ degrees of freedom and noncentrality parameter $\delta$ is given by \citep{boost_math}

\begin{equation}
	f_{t'}\left(n,x, \delta\right) = \frac{nt}{n^2+2nt^2+t^4} + \frac{1}{2} \sum_{i=0}^{\infty} P_i I'_x\left(i+ \frac{1}{2} , \frac{n}{2}\right) + \frac{\delta}{\sqrt{2}} Q_i I'_x\left(i+1, \frac{n}{2}\right), \quad \text{and}
\end{equation} 
$I'_x(\cdot,\cdot)$ denotes the derivative of the (normalized) incomplete beta function (see section \ref{sec:DerivativeNormalisedIncompleteBetaFunction}), and $P_i$ and $Q_i$ are defined in equation \ref{eq:NonCentralTSeriesCoeff}.



%\subsubsection{CDF: General formulas}
\label{NoncentraltDistributionCDFBoost}
\nomenclature{$F_{t'}\left(n,x, \delta\right)$}{CDF of the (singly) noncentral t-distribution}



\subsubsection{CDF (singly noncentral): Infinite series}
The cdf of a variable following a noncentral  t-distribution with $n$ degrees of freedom and noncentrality parameter $\delta$ is given by \citep{Benton_2003, boost_math}

\begin{equation}
	F_{t'}\left(n,x, \delta\right) = \Phi(-\delta) + \frac{1}{2} \sum_{i=0}^{\infty} P_i I_x\left(i+ \frac{1}{2} , \frac{n}{2}\right) + \frac{\delta}{\sqrt{2}} Q_i I_x\left(i+1, \frac{n}{2}\right), \quad \text{and}
\end{equation} 
\begin{equation}
	1-F_{t'}\left(n,x, \delta\right) = \frac{1}{2} \sum_{i=0}^{\infty} P_i I_y\left( \frac{n}{2}, i+ \frac{1}{2} \right) + \frac{\delta}{\sqrt{2}} Q_i I_y\left(\frac{n}{2},i+1\right), \quad \text{where}
\end{equation}
\begin{equation}\label{eq:NonCentralTSeriesCoeff}
	\lambda = \tfrac{1}{2}\delta^2; \quad P_i =  \frac{e^{-\lambda} \lambda^i}{i!} ; \quad Q_i = \frac{e^{-\lambda} \lambda^i}{\Gamma(i+3/2)};  \quad x=\frac{t^2}{n+t^2};  \quad y=1-x,
\end{equation}
% $I_x(\cdot,\cdot)$ denotes the (normalized) incomplete beta function (see section \ref{sec:NormalisedIncompleteBetaFunction}), and $\Phi(\cdot)$ denotes the cdf of the normal distribution (see section \ref{sec:NormalDistribution_CDF}).




\subsection{Quantiles}
\nomenclature{$t_{n,\delta;\alpha}$}{$\alpha$ quantile of the noncentral $t$-distribution with $\nu$ degrees of freedom and noncentrality parameter $\delta$}
\label{NoncentralTQuantileBoost}


\begin{mpFunctionsExtract}
	\mpFunctionFour
	{NoncentralTDistInvBoost? mpNumList? quantiles and related information for the  noncentral $t$-distribution}
	{Prob? mpNum? A real number between 0 and 1.}
	{n? mpNum? A real number greater 0, representing the degrees of freedom}
	{delta? mpNum? A real number greater 0, representing the noncentrality parameter}
	{Output? String? A string describing the output choices}
\end{mpFunctionsExtract}

\vspace{0.3cm}
See section \ref{Functions returning Quantiles} for the options for  {\itshape\sffamily Prob} and {\itshape\sffamily Output}). Algorithms and formulas are given below:



\subsection{Properties}
\label{NoncentraltDistributionPropertiesBoost}

\begin{mpFunctionsExtract}
	\mpFunctionThree
	{NoncentralTDistInfoBoost? mpNumList? moments and related information for the noncentral $t$-distribution}
	{n? mpNum? A real number greater 0, representing the degrees of freedom}
	{delta? mpNum? A real number greater 0, representing the noncentrality parameter}
	{Output? String? A string describing the output choices}
\end{mpFunctionsExtract}

\vspace{0.3cm}

See section \ref{Functions returning moments and related information} for the options for {\itshape\sffamily Output}. Algorithms and formulas are given below.



\subsubsection{Moments (singly noncentral)}

The algebraic moments (defined for $n>r$) are given by
\begin{equation}
	\mu'_r = \left({\tfrac{1}{2}n}\right)^{r/2} \frac{\Gamma\left(\tfrac{1}{2}(n-r)\right)}{\Gamma\left(\tfrac{1}{2}n\right)}  \sum_{i=0}^{[r/2]_G} { \binom{r}{2i} \frac{(2i)!} {2^i i!}} \delta^{r-2i}.
\end{equation}


The first four raw moments are given by
\begin{equation}
	E(t)=\delta \sqrt{\tfrac{1}{2}n} \frac{\Gamma\left(\tfrac{1}{2}(n-1)\right)}{\Gamma\left(\tfrac{1}{2}n\right)} 
\end{equation}
\begin{equation}
	E(t^2)= (\delta^2+1) \frac{n}{n-2} 
\end{equation}
\begin{equation}
	E(t^3)=\delta(\delta^2+3) \sqrt{\tfrac{1}{8}n^3} \frac{\Gamma\left(\tfrac{1}{2}(n-3)\right)}{\Gamma\left(\tfrac{1}{2}n\right)} 
\end{equation}
\begin{equation}
	E(t^4)= (\delta^4+ 6\delta^2+3) \frac{n^2}{(n-2)(n-4)} 
\end{equation}




\subsection{Random Numbers}
\label{NoncentraltDistributionRandomBoost}

\begin{mpFunctionsExtract}
	\mpFunctionFive
	{NoncentralTDistRanBoost? mpNumList? random numbers following a noncentral $t$-distribution}
	{Size? mpNum? A positive integer up to $10^7$}
	{n? mpNum? A real number greater 0, representing the degrees of freedom}
	{delta? mpNum? A real number greater 0, representing the noncentrality parameter}
	{Generator? String? A string describing the random generator}
	{Output? String? A string describing the output choices}
\end{mpFunctionsExtract}


\vspace{0.3cm}
See section \ref{Functions returning Random numbers} for the options for  {\itshape\sffamily Size},  {\itshape\sffamily Generator} and {\itshape\sffamily Output}. Algorithms and formulas are given below:

\vspace{0.3cm}
Random numbers from a non-central t-distribution is easily obtained using the definition in terms of the ratio between two independent random numbers from a normal and a central chi-square distribution. This ought to be sufficient for most applications but if needed more efficient techniques may easily be developed e.g. using more general techniques.





\section{Skew Normal Distribution}

\subsection{Definition}
\label{SkewNormalDistributionDefinition}

In probability theory and statistics, the skew normal distribution is a continuous probability distribution that generalises the normal distribution to allow for non-zero skewness.

See also \href{http://en.wikipedia.org/wiki/Skew_normal_distribution}{http://en.wikipedia.org/wiki/Skew\_normal\_distribution}. 


\subsection{Density and CDF}

\begin{mpFunctionsExtract}
	\mpFunctionFive
	{SkewNormalDistBoost? mpNumList? returns pdf, CDF and related information for the skew normal distribution}
	{x? mpNum? A real number.}
	{a? mpNum? The location parameter.}
	{b? mpNum? The scale parameter}
	{c? mpNum? The shape parameter}
	{Output? String? A string describing the output choices}
\end{mpFunctionsExtract}


\vspace{0.3cm}
See section \ref{Functions returning pdf, CDF, and related information} for the options for {\itshape\sffamily Output}. Algorithms and formulas are given in sections \ref{SkewNormalDistributionDensity} and \ref{SkewNormalDistributionCDF}.




\subsubsection{Density}
\label{SkewNormalDistributionDensity}

The probability density function with location parameter $\xi$, scale parameter $\omega$, and shape parameter $\alpha$  is
\begin{equation} 
	f(\xi;\omega,\alpha)= \frac{2}{\omega} \phi \left(\frac{x-\xi}{\omega}\right)  \Phi \left(\alpha \left(\frac{x-\xi}{\omega}\right)\right)  
\end{equation}

The probability density function with location parameter $a$, scale parameter $b$, and shape parameter $c$  is
\begin{equation} 
	f(x;a,b,c)= \frac{2}{b} \phi \left(\frac{x-a}{b}\right)  \Phi \left(c \left(\frac{x-a}{b}\right)\right)  
\end{equation}


\subsubsection{CDF}
\label{SkewNormalDistributionCDF}

\begin{equation} 
	F(\xi;\omega,\alpha)=  \Phi \left(\frac{x-\xi}{\omega}\right) -  2T \left(\frac{x-\xi}{\omega}, \alpha \right)
\end{equation}

\begin{equation} 
	F(a,b,c)=  \Phi \left(\frac{x-a}{b}\right) -  2T \left(\frac{x-a}{b}, c \right)
\end{equation}


\subsection{Quantiles}
\label{SkewNormalDistributionQuantiles}



\begin{mpFunctionsExtract}
	\mpFunctionFive
	{SkewNormalDistInvBoost? mpNumList? returns quantiles and related information for the the skew normal distribution}
	{Prob? mpNum? A real number between 0 and 1.}
	{a? mpNum? The location parameter.}
	{b? mpNum? The scale parameter}
	{c? mpNum? The shape parameter}
	{Output? String? A string describing the output choices}
\end{mpFunctionsExtract}

\vspace{0.3cm}
See section \ref{Functions returning Quantiles} for the options for  {\itshape\sffamily Prob} and {\itshape\sffamily Output}). 
The quantile is determined using an iterative alogorithm.


\subsection{Properties}
\label{SkewNormalDistributionProperties}


\begin{mpFunctionsExtract}
	\mpFunctionFour
	{SkewNormalDistInfoBoost? mpNumList? returns moments and related information for the skew normal distribution}
	{a? mpNum? The location parameter.}
	{b? mpNum? The scale parameter}
	{c? mpNum? The shape parameter}
	{Output? String? A string describing the output choices}
\end{mpFunctionsExtract}

\vspace{0.3cm}

See section \ref{Functions returning moments and related information} for the options for {\itshape\sffamily Output}. Algorithms and formulas are given below.

\subsubsection{Moments}

\begin{equation} 
	\mu_1 = a+bd\sqrt{\frac{2}{\pi}}, \quad \text{where } d=\frac{c}{\sqrt{1+c^2}}
\end{equation}

\begin{equation} 
	\mu_2 = b^2 \left(1-\frac{2d^2}{\pi} \right)
\end{equation}

\begin{equation} 
	\gamma_1 = \frac{4-\pi}{2} \frac{\left(d\sqrt{2/\pi}\right)^3}{(1-2d^2/\pi)^{3/2}}
\end{equation}

\begin{equation} 
	\gamma_2 = 2(\pi-3) \frac{\left(d\sqrt{2/\pi}\right)^4}{(1-2d^2/\pi)^{2}}
\end{equation}


\subsection{Random Numbers}
\label{SkewNormalDistributionRandom}

\begin{mpFunctionsExtract}
	\mpFunctionSix
	{SkewNormalDistRanBoost? mpNumList? returns random numbers following a skew normal distribution}
	{Size? mpNum? A positive integer up to $10^7$}
	{a? mpNum? The location parameter.}
	{b? mpNum? The scale parameter}
	{c? mpNum? The shape parameter}
	{Generator? String? A string describing the random generator}
	{Output? String? A string describing the output choices}
\end{mpFunctionsExtract}

\vspace{0.3cm}

See section \ref{Functions returning Random numbers} for the options for  {\itshape\sffamily Size},  {\itshape\sffamily Generator} and {\itshape\sffamily Output}. Algorithms and formulas are given below.






\section{Owen's T-Function}
\subsection{Owen's T-Function}
\label{sec:OwenTFunction}
\nomenclature{$T_{\text{Owen}}(a,b)$}{Owen's T-Function}


\begin{mpFunctionsExtract}
	\mpFunctionTwo
	{TOwenBoost? mpNum? Owen's T-Function}
	{h? mpNum? A real number.}
	{a? mpNum? A real number.}
\end{mpFunctionsExtract}

\vspace{0.3cm}
Owen's T-Function is defined as \citep{owen_1956}:

\begin{equation}
	T(h,a) = \frac{1}{2\pi} \int_0^a \frac{\exp \left[-\tfrac{1}{2} h^2 (1+x^2)\right]}{1+x^2} dx
\end{equation}

The implementation uses the algorithm descibed in \cite{patefield_2000}.




\chapter{Ordinary Differential Equations}
\label{OrdinaryDifferentialEquations} 

The procedures in this chapter are based on Boost.Numeric.Odeint (see \cite{boost_odeint}), a library for solving initial value problems (IVP) of ordinary differential equations. Mathematically, these problems are formulated as follows: 

$x'(t) = f(x,t), x(0) = x0. $

$x$ and $f$ can be vectors and the solution is some function $x(t)$ fulfilling both equations above. In the following we will refer to $x'(t)$ also $dxdt$ which is also our notation for the derivative in the source code. 

Ordinary differential equations occur nearly everywhere in natural sciences. For example, the whole Newtonian mechanics are described by second order differential equations. Be sure, you will find them in every discipline. They also occur if partial differential equations (PDEs) are discretized. Then, a system of coupled ordinary differential occurs, sometimes also referred as lattices ODEs. 

Numerical approximations for the solution $x(t)$ are calculated iteratively. The easiest algorithm is the Euler scheme, where starting at $x(0)$ one finds $x(dt) = x(0) + dt f(x(0),0)$. Now one can use $x(dt)$ and obtain $x(2dt)$ in a similar way and so on. The Euler method is of order 1, that means the error at each step is $\approx dt2$. This is, of course, not very satisfactory, which is why the Euler method is rarely used for real life problems and serves just as illustrative example. 

The main focus of odeint is to provide numerical methods implemented in a way where the algorithm is completely independent on the data structure used to represent the state x. In doing so, odeint is applicable for a broad variety of situations and it can be used with many other libraries. Besides the usual case where the state is defined as a std::vector or a boost::array, we provide native support for the following libraries: 

General Literature includes:

General information about numerical integration of ordinary differential equations: 

\cite{NumericalRecipes_2007}

\cite{Hairer_2009}

\cite{Hairer_2010}



Symplectic integration of numerical integration: 

\cite{Hairer_2006}

\cite{Leimkuhler_2005}



Special symplectic methods: 

\cite{Yoshida_1990}


\cite{McLachlan_1995}




Special systems: 

Fermi-Pasta-Ulam nonlinear lattice oscillations %at http://www.scholarpedia.org/article/Fermi-Pasta-Ulam_nonlinear_lattice_oscillations

\cite{Pikovsky_2001}



\section{Defining the ODE System}
\label{IntroductionODE}

The routines solve the general n-dimensional first-order system,
\begin{equation}
\frac{dy_i(t)}{dt} = f_i(t, y_1(t), . . . y_n(t))
\end{equation}
for $i = 1, . . . , n$. The stepping functions rely on the vector of derivatives $f_i$ and the Jacobian
matrix, $J_{ij} =\partial f_i(t, y(t))/\partial y_j$. A system of equations is defined using the 
system datatype.

\section{Stepping Functions}
\label{SteppingFunctionsODE}

Solving ordinary differential equation numerically is usually done iteratively, that is a given state of an ordinary differential equation is iterated forward $x(t) -> x(t+dt) -> x(t+2dt)$. The steppers in odeint perform one single step. The most general stepper type is described by the Stepper concept. The stepper concepts of odeint are described in detail in section Concepts, here we briefly present the mathematical and numerical details of the steppers. The Stepper has two versions of the do\_step method, one with an in-place transform of the current state and one with an out-of-place transform: 

do\_step( sys , inout , t , dt ) 

do\_step( sys , in , t , out , dt ) 

The first parameter is always the system function - a function describing the ODE. In the first version the second parameter is the step which is here updated in-place and the third and the fourth parameters are the time and step size (the time step). After a call to do\_step the state inout is updated and now represents an approximate solution of the ODE at time t+dt. In the second version the second argument is the state of the ODE at time t, the third argument is t, the fourth argument is the approximate solution at time t+dt which is filled by do\_step and the fifth argument is the time step. Note that these functions do not change the time t. 

System functions 

Up to now, we have nothing said about the system function. This function depends on the stepper. For the explicit Runge-Kutta steppers this function can be a simple callable object hence a simple (global) C-function or a functor. The parameter syntax is $sys( x , dxdt , t )$ and it is assumed that it calculates $dx/dt = f(x,t)$. The function structure in most cases looks like: 


void sys( const state\_type \& /*x*/ , state\_type \& /*dxdt*/ , const double /*t*/ )
{
	// ...
}


Other types of system functions might represent Hamiltonian systems or systems which also compute the Jacobian needed in implicit steppers. For information which stepper uses which system function see the stepper table below. It might be possible that odeint will introduce new system types in near future. Since the system function is strongly related to the stepper type, such an introduction of a new stepper might result in a new type of system function. 

Explicit steppers 
A first specialization are the explicit steppers. Explicit means that the new state of the ode can be computed explicitly from the current state without solving implicit equations. Such steppers have in common that they evaluate the system at time t such that the result of f(x,t) can be passed to the stepper. In odeint, the explicit stepper have two additional methods 


Which steppers should be used in which situation 

odeint provides a quite large number of different steppers such that the user is left with the question of which stepper fits his needs. Our personal recommendations are: 

runge\_kutta\_dopri5 is maybe the best default stepper. It has step size control as well as dense-output functionality. Simple create a dense-output stepper by make\_dense\_output( 1.0e-6 , 1.0e-5 , runge\_kutta\_dopri5< state\_type >() ). 
runge\_kutta4 is a good stepper for constant step sizes. It is widely used and very well known. If you need to create artificial time series this stepper should be the first choice. 
'runge\_kutta\_fehlberg78' is similar to the 'runge\_kutta4' with the advantage that it has higher precision. It can also be used with step size control. 
adams\_bashforth\_moulton is very well suited for ODEs where the r.h.s. is expensive (in terms of computation time). It will calculate the system function only once during each step. 



\subsection{Explicit Euler}
\label{Explicit Euler}

In mathematics and computational science, the Euler method is a first-order numerical procedure for solving ordinary differential equations (ODEs) with a given initial value. It is the most basic explicit method for numerical integration of ordinary differential equations and is the simplest Runge–Kutta method. The Euler method is named after Leonhard Euler, who treated it in his book Institutionum calculi integralis (published 1768–70).[1]

The Euler method is a first-order method, which means that the local error (error per step) is proportional to the square of the step size, and the global error (error at a given time) is proportional to the step size. The Euler method often serves as the basis to construct more complicated methods.





\subsection{Modified Midpoint}
\label{Modified Midpoint}

In numerical analysis, a branch of applied mathematics, the midpoint method is a one-step method for numerically solving the differential equation,

$y'(t)=f(t,y(t), y(t_0)=y_0$

and is given by the formula

$y_{n+1}=y_n + h f(t_n+h/2, y_n+ (h/2) f(t_n, y_n),$

for $n=0,1,2,...$  Here, $h$ is the step size - a small positive number, $t_n=t_0+nh$ and $y_n$ is the computed approximate value of $y(t_n)$. The midpoint method is also known as the modified Euler method.[1]

The name of the method comes from the fact that in the formula above the function $f$ is evaluated at $t=t_n+h/2$ which is the midpoint between $t_n$ at which the value of $y(t)$ is known and $t_{n+1}$ at which the value of $y(t)$ needs to be found.

The local error at each step of the midpoint method is of order $O(h^3)$, giving a global error of order $O(h^2)$. Thus, while more computationally intensive than Euler's method, the midpoint method generally gives more accurate results.

The method is an example of a class of higher-order methods known as Runge-Kutta methods.




\subsection{Runge-Kutta 4}
\label{RungeKutta4}

In numerical analysis, the Runge–Kutta methods are an important family of implicit and explicit iterative methods, which are used in temporal discretization for the approximation of solutions of ordinary differential equations. These techniques were developed around 1900 by the German mathematicians C. Runge and M. W. Kutta.

See the article on numerical methods for ordinary differential equations for more background and other methods. See also List of Runge–Kutta methods.

One member of the family of Runge–Kutta methods is often referred to as "RK4", "classical Runge–Kutta method" or simply as "the Runge–Kutta method".

Let an initial value problem be specified as follows.


Here, y is an unknown function (scalar or vector) of time t which we would like to approximate; we are told that , the rate at which y changes, is a function of t and of y itself. At the initial time  the corresponding y-value is . The function f and the data ,  are given.




\subsection{Cash-Karp}
\label{CashKarp}

In numerical analysis, the Cash–Karp method is a method for solving ordinary differential equations (ODEs). It was proposed by Professor Jeff R. Cash [1] from Imperial College London and Alan H. Karp from IBM Scientific Center. The method is a member of the Runge–Kutta family of ODE solvers. More specifically, it uses six function evaluations to calculate fourth- and fifth-order accurate solutions. The difference between these solutions is then taken to be the error of the (fourth order) solution. This error estimate is very convenient for adaptive stepsize integration algorithms. Other similar integration methods are Fehlberg (RKF) and Dormand–Prince (RKDP).

J. R. Cash, A. H. Karp. "A variable order Runge-Kutta method for initial value problems with rapidly varying right-hand sides", ACM Transactions on Mathematical Software 16: 201-222, 1990. doi:10.1145/79505.79507.

Shampine, Lawrence F. (1986), "Some Practical Runge-Kutta Formulas", Mathematics of Computation (American Mathematical Society) 46 (173): 135–150, doi:10.2307/2008219, JSTOR 2008219 .



\subsection{Dormand-Prince 5}
\label{DormandPrince}

In numerical analysis, the Dormand–Prince method, or DOPRI method, is an explicit method for solving ordinary differential equations \citep{Dormand1980}. The method is a member of the Runge–Kutta family of ODE solvers. More specifically, it uses six function evaluations to calculate fourth- and fifth-order accurate solutions. The difference between these solutions is then taken to be the error of the (fourth-order) solution. This error estimate is very convenient for adaptive stepsize integration algorithms. Other similar integration methods are Fehlberg (RKF) and Cash–Karp (RKCK).

The Dormand–Prince method has seven stages, but it uses only six function evaluations per step because it has the FSAL (First Same As Last) property: the last stage is evaluated at the same point as the first stage of the next step. Dormand and Prince choose the coefficients of their method to minimize the error of the fifth-order solution. This is the main difference with the Fehlberg method, which was constructed so that the fourth-order solution has a small error. For this reason, the Dormand–Prince method is more suitable when the higher-order solution is used to continue the integration, a practice known as local extrapolation (Shampine 1986; Hairer, Nørsett \& Wanner 2008, pp. 178–179).

Dormand–Prince is currently the default method in MATLAB and GNU Octave's ode45 solver and is the default choice for the Simulink's model explorer solver. A Fortran free software implementation of the algorithm called DOPRI5 is also available.[1]




\subsection{Fehlberg 78}
In mathematics, the Runge–Kutta–Fehlberg method (or Fehlberg method) is an algorithm in numerical analysis for the numerical solution of ordinary differential equations. It was developed by the German mathematician Erwin Fehlberg and is based on the large class of Runge–Kutta methods.

The novelty of Fehlberg's method is that it is an embedded method from the Runge-Kutta family, meaning that identical function evaluations are used in conjunction with each other to create methods of varying order and similar error constants. The method presented in Fehlberg's 1969 paper has been dubbed the RKF45 method, and is a method of order O(h4) with an error estimator of order O(h5).[1] By performing one extra calculation, the error in the solution can be estimated and controlled by using the higher-order embedded method that allows for an adaptive stepsize to be determined automatically.

Erwin Fehlberg (1970). "Klassische Runge-Kutta-Formeln vierter und niedrigerer Ordnung mit Schrittweiten-Kontrolle und ihre Anwendung auf Wärmeleitungsprobleme," Computing (Arch. Elektron. Rechnen), vol. 6, pp. 61–71. doi:10.1007/BF02241732




\subsection{Adams-Bashforth}
\label{Adams-Bashforth}

Three families of linear multistep methods are commonly used: Adams–Bashforth methods, Adams–Moulton methods, and the backward differentiation formulas (BDFs).

Adams–Bashforth methods[edit]The Adams–Bashforth methods are explicit methods. The coefficients are  and , while the  are chosen such that the methods has order s (this determines the methods uniquely).

The Adams–Bashforth methods with s = 1, 2, 3, 4, 5 are (Hairer, Nørsett \& Wanner 1993, §III.1; Butcher 2003, p. 103):




\subsection{Adams-Moulton}
\label{Adams-Moulton}

The Adams–Moulton methods are similar to the Adams–Bashforth methods in that they also have  and . Again the b coefficients are chosen to obtain the highest order possible. However, the Adams–Moulton methods are implicit methods. By removing the restriction that , an s-step Adams–Moulton method can reach order , while an s-step Adams–Bashforth methods has only order s.

The Adams–Moulton methods with s = 0, 1, 2, 3, 4 are (Hairer, Nørsett \& Wanner 1993, §III.1; Quarteroni, Sacco \& Saleri 2000):




\subsection{Adams-Bashforth-Moulton}
\label{Adams-Bashforth-Moulton}

The methods of Euler, Heun, Taylor and Runge-Kutta are called single-step methods because they use only the information from one previous point to compute the successive point, that is, only the initial point    is used to compute    and in general    is needed to compute  .  After several points have been found it is feasible to use several prior points in the calculation.  The Adams-Bashforth-Moulton method uses   in the calculation of .  This method is not self-starting;  four initial points  , , ,  and  must be given in advance in order to generate the points .  

A desirable feature of a multistep method is that the local truncation error (L. T. E.) can be determined and a correction term can be included, which improves the accuracy of the answer at each step.  Also, it is possible to determine if the step size is small enough to obtain an accurate value for  , yet large enough so that unnecessary and time-consuming calculations are eliminated.  If the code for the subroutine is fine-tuned, then the combination of a  predictor and corrector requires only two function evaluations of  f(t,y)  per step. 

See also: http://mathfaculty.fullerton.edu/mathews//n2003/AdamsBashforthMod.html


\subsection{Controlled Runge-Kutta}
\label{Controlled Runge-Kutta}

\lipsum[2]


\subsection{Dense Output Runge-Kutta}
\label{Dense Output Runge-Kutta}

\lipsum[2]


\subsection{Bulirsch-Stoer}
\label{Bulirsch-Stoer}

\lipsum[4]


\subsection{Bulirsch-Stoer Dense Output}
\label{Bulirsch-Stoer Dense Output}

\lipsum[4]



\subsection{Implicit Euler}
\label{Implicit Euler}

\lipsum[2]


\subsection{Rosenbrock 4}
\label{Rosenbrock 4}

\lipsum[2]


\subsection{Controlled Rosenbrock 4}
\label{Controlled Rosenbrock 4}

\lipsum[2]


\subsection{Dense Output Rosenbrock 4}
\label{Dense Output Rosenbrock 4}

\lipsum[2]


\subsection{Symplectic Euler}
\label{Symplectic Euler}

\lipsum[2]


\subsection{Symplectic RKN McLachlan}
\label{Symplectic RKN McLachlan}

\lipsum[2]


\section{Integrate functions: Evolution}
\label{EvolutionODE}

Integrate functions perform the time evolution of a given ODE from some starting time t0 to a given end time t1 and starting at state x0 by subsequent calls of a given stepper's do\_step function. 

Additionally, the user can provide an observer to analyze the state during time evolution. There are five different integrate functions which have different strategies on when to call the observer function during integration. 

All of the integrate functions except integrate\_n\_steps can be called with any stepper following one of the stepper concepts: Stepper , Error Stepper , Controlled Stepper , Dense Output Stepper. Depending on the abilities of the stepper, the integrate functions make use of step-size control or dense output. 




